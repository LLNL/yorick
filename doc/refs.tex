% Language Reference Card for Yorick version 1
% Function Reference Card for Yorick version 1
% I/O Reference Card for Yorick version 1
% $Id: refs.tex,v 1.1 2005-09-18 22:05:02 dhmunro Exp $

%**start of header
\newcount\columnsperpage

% This file can be printed with 1, 2, or 3 columns per page (see below).
% Specify how many you want here.  Nothing else needs to be changed.

\columnsperpage=3

% Based on the refcard.tex file from the GNU Emacs 18.57 distribution:
% Copyright (c) 1987 Free Software Foundation, Inc.

% This file is intended to be processed by plain TeX (TeX82).
%
% The final reference card has six columns, three on each side.
% This file can be used to produce it in any of three ways:
% 1 column per page
%    produces six separate pages, each of which needs to be reduced to 80%.
%    This gives the best resolution.
% 2 columns per page
%    produces three already-reduced pages.
%    You will still need to cut and paste.
% 3 columns per page
%    produces two pages which must be printed sideways to make a
%    ready-to-use 8.5 x 11 inch reference card.
%    For this you need a dvi device driver that can print sideways.
% Which mode to use is controlled by setting \columnsperpage above.
%
% Author of GNU Emacs refcard.tex from which this is derived:
%  Stephen Gildea
%  UUCP: mit-erl!gildea
%  Internet: gildea@erl.mit.edu

\def\versionnumber{1.1}
\def\year{1995}
\def\version{March \year\ v\versionnumber}

\def\shortcopyrightnotice{\vskip 1ex plus 2 fill
  \centerline{\small \copyright\ \year\ Regents of the U. C.
  Permissions on back.  v\versionnumber}}

\def\copyrightnotice{
\vskip 1ex plus 2 fill\begingroup\small
\centerline{Copyright \copyright\ \year\
            Regents of the University of California}
\centerline{designed by David Munro, \version}
\centerline{for Yorick version 1}

Permission is granted to make and distribute copies of
this card provided the copyright notice and this permission notice
are preserved on all copies.

For copies of the Yorick manual, contact dhmunro@users.sf.net.

\endgroup}

% make \bye not \outer so that the \def\bye in the \else clause below
% can be scanned without complaint.
\def\bye{\par\vfill\supereject\end}

\newdimen\intercolumnskip
\newbox\columna
\newbox\columnb

\def\ncolumns{\the\columnsperpage}

\message{[\ncolumns\space
  column\if 1\ncolumns\else s\fi\space per page]}

\def\scaledmag#1{ scaled \magstep #1}

% This multi-way format was designed by Stephen Gildea
% October 1986.
\if 1\ncolumns
  \hsize 4in
  \vsize 10in
  \voffset -.7in
  \font\titlefont=\fontname\tenbf \scaledmag3
  \font\headingfont=\fontname\tenbf \scaledmag2
  \font\smallfont=\fontname\sevenrm
  \font\smallsy=\fontname\sevensy

  \footline{\hss\folio}
  \def\makefootline{\baselineskip10pt\hsize6.5in\line{\the\footline}}
\else
  \hsize 3.2in
  \vsize 7.95in
  \hoffset -.75in
  \voffset -.745in
  \font\titlefont=cmbx10 \scaledmag2
  \font\headingfont=cmbx10 \scaledmag1
  \font\smallfont=cmr6
  \font\smallsy=cmsy6
  \font\eightrm=cmr8
  \font\eightbf=cmbx8
  \font\eightit=cmti8
  \font\eighttt=cmtt8
  \font\eightsy=cmsy8
  \textfont0=\eightrm
  \textfont2=\eightsy
  \def\rm{\eightrm}
  \def\bf{\eightbf}
  \def\it{\eightit}
  \def\tt{\eighttt}
  \normalbaselineskip=.8\normalbaselineskip
  \normallineskip=.8\normallineskip
  \normallineskiplimit=.8\normallineskiplimit
  \normalbaselines\rm           %make definitions take effect

  \if 2\ncolumns
    \let\maxcolumn=b
    \footline{\hss\rm\folio\hss}
    \def\makefootline{\vskip 2in \hsize=6.86in\line{\the\footline}}
  \else \if 3\ncolumns
    \let\maxcolumn=c
    \nopagenumbers
  \else
    \errhelp{You must set \columnsperpage equal to 1, 2, or 3.}
    \errmessage{Illegal number of columns per page}
  \fi\fi

  \intercolumnskip=.46in
  \def\abc{a}
  \output={%
      % This next line is useful when designing the layout.
      %\immediate\write16{Column \folio\abc\space starts with \firstmark}
      \if \maxcolumn\abc \multicolumnformat \global\def\abc{a}
      \else\if a\abc
        \global\setbox\columna\columnbox \global\def\abc{b}
        %% in case we never use \columnb (two-column mode)
        \global\setbox\columnb\hbox to -\intercolumnskip{}
      \else
        \global\setbox\columnb\columnbox \global\def\abc{c}\fi\fi}
  \def\multicolumnformat{\shipout\vbox{\makeheadline
      \hbox{\box\columna\hskip\intercolumnskip
        \box\columnb\hskip\intercolumnskip\columnbox}
      \makefootline}\advancepageno}
  \def\columnbox{\leftline{\pagebody}}

  \def\bye{\par\vfill\supereject
    \if a\abc \else\null\vfill\eject\fi
    \if a\abc \else\null\vfill\eject\fi
    \end}
\fi

% we won't be using math mode much, so redefine some of the characters
% we might want to talk about
% \catcode`\^=12
\catcode`\_=12

\chardef\\=`\\
\chardef\{=`\{
\chardef\}=`\}

\hyphenation{mini-buf-fer}
\hyphenation{Yor-ick}
\hyphenation{ter-min-ated}

\parindent 0pt
\parskip 1ex plus .5ex minus .5ex

\def\small{\smallfont\textfont2=\smallsy\baselineskip=.8\baselineskip}

\outer\def\newcolumn{\vfill\eject}

\outer\def\title#1{{\titlefont\centerline{#1}}\vskip 1ex plus .5ex}

\outer\def\section#1{\par\filbreak
  \vskip 3ex plus 2ex minus 2ex {\headingfont #1}\mark{#1}%
  \vskip 2ex plus 1ex minus 1.5ex}

\newdimen\keyindent

\def\beginindentedkeys{\keyindent=1em}
\def\endindentedkeys{\keyindent=0em}
\endindentedkeys

\def\paralign{\vskip\parskip\halign}

\def\<#1>{$\langle${\rm #1}$\rangle$}

\def\kbd#1{{\tt#1}\null}        %\null so not an abbrev even if period follows

\def\beginexample{\leavevmode\begingroup
  \obeylines\obeyspaces\parskip0pt\tt}
{\obeyspaces\global\let =\ }
\def\endexample{\endgroup}

\def\key#1#2{\leavevmode\hbox to \hsize{\vtop
  {\hsize=.75\hsize\rightskip=1em
  \hskip\keyindent\relax#1}\kbd{#2}\hfil}}

\def\func#1#2{\leavevmode\hbox to \hsize{\hbox to \hsize
  {\hskip\keyindent\relax\kbd{#1}\hfil}%
  \hskip 0pt minus 1fil
  \null#2\hfil}}

\def\sample#1#2{\leavevmode\hbox to \hsize{\hbox to \hsize
  {\hskip\keyindent\relax#1\hfil}%
  \hskip 0pt minus 1fil
  \null#2\hfil}}

\def\begindemo{\leavevmode\begingroup
  \obeylines\obeyspaces\parskip0pt}
% {\obeyspaces\global\let =\ }
\def\enddemo{\endgroup}

\newif\ifpdf
\ifx\pdfoutput\undefined
  \pdffalse
  \def\mysection#1{\par\filbreak
    \vskip 3ex plus 2ex minus 2ex {\headingfont #1}\mark{#1}%
    \vskip 2ex plus 1ex minus 1.5ex}
\else
  \pdftrue
  \pdfpagewidth=11truein
  \pdfpageheight=8.5truein
  \pdfcompresslevel=9
  \outer\def\mysection#1{\par\filbreak
    \vskip 3ex plus 2ex minus 2ex\hskip -2em \pdfdest name{#1} xyz \hskip 2em {\headingfont #1}\mark{#1}%
    \vskip 2ex plus 1ex minus 1.5ex}
  \pdfcatalog{/PageMode /UseOutlines} openaction goto page 1 {/FitV}
  \pdfoutline goto page 1 {/FitV} count 9 {Language Reference-1}
    \pdfoutline goto name{Starting and Quitting Yorick}{Starting and Quitting Yorick}
    \pdfoutline goto name{Getting Help}{Getting Help}
    \pdfoutline goto name{Error Recovery}{Error Recovery}
    \pdfoutline goto name{Array Data Types}{Array Data Types}
    \pdfoutline goto name{Constants}{Constants}
    \pdfoutline goto name{Defining Variables}{Defining Variables}
    \pdfoutline goto name{Arithmetic and Comparison Operators}{Arithmetic and Comparison Operators}
    \pdfoutline goto name{Creating Arrays}{Creating Arrays}
    \pdfoutline goto name{Indexing Arrays}{Indexing Arrays}
  \pdfoutline goto page 2 {/FitV} count -9 {Language Reference-2}
    \pdfoutline goto name{Array Conformability Rules}{Array Conformability Rules}
    \pdfoutline goto name{Logical Operators}{Logical Operators}
    \pdfoutline goto name{Calling Functions}{Calling Functions}
    \pdfoutline goto name{Defining Functions}{Defining Functions}
    \pdfoutline goto name{Variable Scope}{Variable Scope}
    \pdfoutline goto name{Returning from Functions}{Returning from Functions}
    \pdfoutline goto name{Compound Statements}{Compound Statements}
    \pdfoutline goto name{Conditional Execution}{Conditional Execution}
    \pdfoutline goto name{Loops}{Loops}
  \pdfoutline goto page 3 {/FitV} count -8 {Function Reference-1}
    \pdfoutline goto name{Including Source Files}{Including Source Files}
    \pdfoutline goto name{Comments}{Comments}
    \pdfoutline goto name{Issuing Shell Commands}{Issuing Shell Commands}
    \pdfoutline goto name{Matrix Multiplication}{Matrix Multiplication}
    \pdfoutline goto name{Using Pointers}{Using Pointers}
    \pdfoutline goto name{Instancing Data Structures}{Instancing Data Structures}
    \pdfoutline goto name{Index Range Functions}{Index Range Functions}
    \pdfoutline goto name{Elementary Functions}{Elementary Functions}
  \pdfoutline goto page 4 {/FitV} count -9 {Function Reference-2}
    \pdfoutline goto name{Information About Variables}{Information About Variables}
    \pdfoutline goto name{Reshaping Arrays}{Reshaping Arrays}
    \pdfoutline goto name{Logical Functions}{Logical Functions}
    \pdfoutline goto name{Interpolation and Lookup Functions}{Interpolation and Lookup Functions}
    \pdfoutline goto name{Sorting}{Sorting}
    \pdfoutline goto name{Transposing}{Transposing}
    \pdfoutline goto name{Manipulating Strings}{Manipulating Strings}
    \pdfoutline goto name{Advanced Array Indexing}{Advanced Array Indexing}
    \pdfoutline goto name{Generating Simple Meshes}{Generating Simple Meshes}
  \pdfoutline goto page 5 {/FitV} count -7 {I/O Reference-1}
    \pdfoutline goto name{Opening and Closing Text Files}{Opening and Closing Text Files}
    \pdfoutline goto name{Reading Text}{Reading Text}
    \pdfoutline goto name{Writing Text}{Writing Text}
    \pdfoutline goto name{Positioning a Text File}{Positioning a Text File}
    \pdfoutline goto name{Opening and Closing Binary Files}{Opening and Closing Binary Files}
    \pdfoutline goto name{Saving and Restoring Variables}{Saving and Restoring Variables}
    \pdfoutline goto name{Reading History Records}{Reading History Records}
  \pdfoutline goto page 6 {/FitV} count -7 {I/O Reference-2}
    \pdfoutline goto name{Writing History Records}{Writing History Records}
    \pdfoutline goto name{Opening Non-PDB Files}{Opening Non-PDB Files}
    \pdfoutline goto name{Making Plots}{Making Plots}
    \pdfoutline goto name{Plot Paging and Hardcopy}{Plot Paging and Hardcopy}
    \pdfoutline goto name{Setting Plot Limits}{Setting Plot Limits}
    \pdfoutline goto name{Managing Graphics Windows}{Managing Graphics Windows}
    \pdfoutline goto name{Graphics Query, Edit, and Defaults}{Graphics Query, Edit, and Defaults}
  \pdfoutline goto page 1 {/FitV} count 7 {TOPICS}
    \pdfoutline goto page 1 {/FitV} count -9 {Getting Started}
      \pdfoutline goto name{Starting and Quitting Yorick}{Starting and Quitting Yorick}
      \pdfoutline goto name{Getting Help}{Getting Help}
      \pdfoutline goto name{Error Recovery}{Error Recovery}
      \pdfoutline goto name{Including Source Files}{Including Source Files}
      \pdfoutline goto name{Comments}{Comments}
      \pdfoutline goto name{Issuing Shell Commands}{Issuing Shell Commands}
      \pdfoutline goto name{Constants}{Constants}
      \pdfoutline goto name{Defining Variables}{Defining Variables}
      \pdfoutline goto name{Arithmetic and Comparison Operators}{Arithmetic and Comparison Operators}
    \pdfoutline goto page 2 {/FitV} count -13 {Programming}
      \pdfoutline goto name{Logical Operators}{Logical Operators}
      \pdfoutline goto name{Compound Statements}{Compound Statements}
      \pdfoutline goto name{Conditional Execution}{Conditional Execution}
      \pdfoutline goto name{Loops}{Loops}
      \pdfoutline goto name{Instancing Data Structures}{Instancing Data Structures}
      \pdfoutline goto name{Using Pointers}{Using Pointers}
      \pdfoutline goto name{Elementary Functions}{Elementary Functions}
      \pdfoutline goto name{Logical Functions}{Logical Functions}
      \pdfoutline goto name{Information About Variables}{Information About Variables}
      \pdfoutline goto name{Matrix Multiplication}{Matrix Multiplication}
      \pdfoutline goto name{Sorting}{Sorting}
      \pdfoutline goto name{Manipulating Strings}{Manipulating Strings}
      \pdfoutline goto name{Generating Simple Meshes}{Generating Simple Meshes}
    \pdfoutline goto page 2 {/FitV} count -4 {Functions}
      \pdfoutline goto name{Calling Functions}{Calling Functions}
      \pdfoutline goto name{Defining Functions}{Defining Functions}
      \pdfoutline goto name{Variable Scope}{Variable Scope}
      \pdfoutline goto name{Returning from Functions}{Returning from Functions}
    \pdfoutline goto page 1 {/FitV} count -8 {Arrays}
      \pdfoutline goto name{Array Data Types}{Array Data Types}
      \pdfoutline goto name{Creating Arrays}{Creating Arrays}
      \pdfoutline goto name{Indexing Arrays}{Indexing Arrays}
      \pdfoutline goto name{Array Conformability Rules}{Array Conformability Rules}
      \pdfoutline goto name{Advanced Array Indexing}{Advanced Array Indexing}
      \pdfoutline goto name{Reshaping Arrays}{Reshaping Arrays}
      \pdfoutline goto name{Transposing}{Transposing}
      \pdfoutline goto name{Index Range Functions}{Index Range Functions}
    \pdfoutline goto page 5 {/FitV} count -4 {Text Files}
      \pdfoutline goto name{Opening and Closing Text Files}{Opening and Closing Text Files}
      \pdfoutline goto name{Reading Text}{Reading Text}
      \pdfoutline goto name{Writing Text}{Writing Text}
      \pdfoutline goto name{Positioning a Text File}{Positioning a Text File}
    \pdfoutline goto page 5 {/FitV} count -5 {Binary Files}
      \pdfoutline goto name{Opening and Closing Binary Files}{Opening and Closing Binary Files}
      \pdfoutline goto name{Saving and Restoring Variables}{Saving and Restoring Variables}
      \pdfoutline goto name{Reading History Records}{Reading History Records}
      \pdfoutline goto name{Writing History Records}{Writing History Records}
      \pdfoutline goto name{Opening Non-PDB Files}{Opening Non-PDB Files}
    \pdfoutline goto page 6 {/FitV} count -6 {Making Plots}
      \pdfoutline goto name{Making Plots}{Making Plots}
      \pdfoutline goto name{Interpolation and Lookup Functions}{Interpolation and Lookup Functions}
      \pdfoutline goto name{Plot Paging and Hardcopy}{Plot Paging and Hardcopy}
      \pdfoutline goto name{Setting Plot Limits}{Setting Plot Limits}
      \pdfoutline goto name{Managing Graphics Windows}{Managing Graphics Windows}
      \pdfoutline goto name{Graphics Query, Edit, and Defaults}{Graphics Query, Edit, and Defaults}
\fi

%**end of header


%----------------------------------------------------------------------------

\title{Yorick Language Reference}

\centerline{(for version 1)}

\mysection{Starting and Quitting Yorick}

To enter Yorick, just type its name: \hglue1em \kbd{yorick}

\key{normal Yorick prompt}{>}
\key{prompt for continued line}{cont>}
\key{prompt for continued string}{quot>}
\key{prompt for continued comment}{comm>}
\key{prompt in debug mode}{dbug>}

\func{quit}{close all open files and exit Yorick}

\mysection{Getting Help}

Most Yorick functions have online documentation.

\func{help}{help on using Yorick help}
\func{help, f}{help on a specific function f}
\func{info, v}{information about a variable v}

\mysection{Error Recovery}

\key{To abort a running Yorick program type}{C-c}

To enter Yorick's debug mode after an error, type return in response
to the first prompt after the error occurs.

\mysection{Array Data Types}

The basic data types are:

\func{char}{one 8-bit byte, from 0 to 255}
\func{short}{compact integer, at least 2 bytes}
\func{int}{{\bf logical results}-- 0 false, 1 true, at least 2 bytes}
\func{long}{{\bf default integer}-- at least 4 bytes}
\func{float}{at least 5 digits, $10^{\pm38}$}
\func{double}{{\bf default real}-- 14 or 15 digits, usually $10^{\pm308}$}
\func{complex}{{\bf re} and {\bf im} parts are double}
\func{string}{0-terminated text string}
\func{pointer}{pointer to an array}

A compound data type {\it compound\_type} can be built from any combination
of basic or previously defined data types as follows:
\begindemo
\hglue1em\kbd{struct} {\it compound\_type} \kbd{\{}
\hglue2em {\it type\_name\_A memb\_name\_1} \kbd{;}
\hglue2em {\it type\_name\_B memb\_name\_2(dimlist)} \kbd{;}
\hglue2em {\it type\_name\_C memb\_name\_3,memb\_name\_4(dimlist)} \kbd{;}
\hglue2em ...
\hglue1em\kbd{\}}
\enddemo
A {\it dimlist} is a comma delimited list of dimension lengths,
or lists in the format returned by the \kbd{dimsof} function, or
ranges of the form \kbd{min\_index : max\_index}.  (By default,
\kbd{min\_index} is 1.)

For example, the \kbd{complex} data type is predefined as:
\begindemo
\hglue1em\kbd{struct complex \{ double re, im; \}}
\enddemo

\shortcopyrightnotice

\mysection{Constants}

By default, an integer number is a constant of type \kbd{long}, and
a real number is a constant of type \kbd{double}.  Constants of the
types \kbd{short}, \kbd{int}, \kbd{float}, and \kbd{complex} are
specified by means of the suffices \kbd{s}, \kbd{n}, \kbd{f}, and
\kbd{i}, respectively.  Here are some examples:

\func{char}{{\tt '\\0', '\\1', '\\x7f', '\\177', 'A', '\\t'}}
\func{short}{{\tt 0s, 1S, 0x7fs, 0177s, -32766s}}
\func{int}{{\tt 0N, 1n, 0x7Fn, 0177n, -32766n}}
\func{long}{{\tt 0, 1, 0x7f, 0177, -32766, 1234L}}
\func{float}{{\tt .0f, 1.f, 1.27e2f, 0.00127f, -32.766e3f}}
\func{double}{{\tt 0.0, 1.0, 127.0, 1.27e-3, -32.766e-33}}
\func{complex}{{\tt 0i, 1i, 127.i, 1.27e-3i, -32.766e-33i}}
\func{string}{{\tt "", "Hello, world!", "\\tTab\\n2nd line"}}

The following escape sequences are recognized in type \kbd{char} and
type \kbd{string} constants:

\keyindent=6em
\func{\\n}{newline\hglue6em}
\func{\\t}{tab\hglue6em}
\func{\\"}{double quote\hglue6em}
\func{\\'}{single quote\hglue6em}
\func{\\\\}{backslash\hglue6em}
\func{\\ooo}{octal number\hglue6em}
\func{\\xhh}{hexadecimal number\hglue6em}
\func{\\a}{alert (bell)\hglue6em}
\func{\\b}{backspace\hglue6em}
\func{\\f}{formfeed (new page)\hglue6em}
\func{\\r}{carriage return\hglue6em}
\endindentedkeys

\mysection{Defining Variables}

\sample{{\it var} {\tt =} {\it expr}}
       {redefines {\it var\/} as the value of {\it expr}}
\sample{{\it var} {\tt = [ ]}}
       {undefines {\it var}}

Any previous value or data type of {\it var} is forgotten.
The {\it expr\/} can be a data type, function, file, or any other object.

The \kbd{=} operator is a binary operator which has the side effect of
redefining its left operand.  It associates to the right, so

\sample{{\it var1} {\tt =} {\it var2} {\tt =} {\it var3} {\tt =} {\it expr}}
       {initializes all three {\it var\/} to {\it expr}}

\mysection{Arithmetic and Comparison Operators}

From highest to lowest precedence,

\keyindent=3em
\func{\^{}}{raise to power\hglue3em}
\func{* / \%}{multiply, divide, modulo\hglue3em}
\func{+ -}{add, subtract (also unary plus, minus)\hglue3em}
\func{<< >>}{shift left, shift right\hglue3em}
\func{>= < <= >}
     {(not) less, (not) greater (\kbd{int} result)\hglue3em}
\func{== !=}{equal, not equal (\kbd{int} result)\hglue3em}
\func{\&}{bitwise and\hglue3em}
\func{\~{}}{bitwise xor (also unary bitwise complement)\hglue3em}
\func{|}{bitwise or\hglue3em}
\func{=}{redefine or assign\hglue3em}
\endindentedkeys

Any binary operator may be prefixed to \kbd{=} to produce an increment
operator; thus \kbd{x*=5} is equivalent to \kbd{x=x*5}.  Also, \kbd{++x}
and \kbd{--x} are equivalent to \kbd{x+=1} and \kbd{x-=1}, respectively.
Finally, \kbd{x++} and \kbd{x--} increment or decrement \kbd{x} by 1, but
return the value of \kbd{x} {\bf before} the operation.

\mysection{Creating Arrays}

\sample{{\tt [} {\it obj1, obj2, ..., objN} {\tt ]}}
       {build an array of N objects}
The {\it objI} may be arrays to build multi-dimensional arrays.

\sample{\kbd{array(}{\it value, dimlist\/}\kbd{)}}
       {add dimensions {\it dimlist\/} to {\it value}}
\sample{\kbd{array(}{\it type\_name, dimlist\/}\kbd{)}}
       {return specified array, all zero}

\sample{\kbd{span(}{\it start, stop, n\/}\kbd{)}}
       {{\it n} equal stepped values from {\it start\/} to {\it stop}}
\sample{\kbd{spanl(}{\it start, stop, n\/}\kbd{)}}
       {{\it n} equal ratio values from {\it start\/} to {\it stop}}
\sample{\kbd{grow, }{\it var, sfx1, sfx2, ...}}
       {append {\it sfx1}, {\it sfx1}, etc. to {\it var}}
These functions may be used to generate multi-dimensional arrays;
 use \kbd{help} for details.

\mysection{Indexing Arrays}

\sample{{\it x}{\tt (}{\it index1, index2, ..., indexN\/}{\tt )}}
       {is a subarray of the array {\it x}}

Each index corresponds to one dimension of the {\it x} array, called
the {\bf ID} in this section (the two exceptions are noted below).
The {\it index1\/} varies fastest, {\it index2\/} next fastest, and so
on.  By default, Yorick indices are 1-origin.  An {\it indexI\/} may
specify multiple index values, in which case the result array will
have one or more dimensions which correspond to the {\bf ID} of
{\it x}.  Possibilities for the {\it indexI} are:

\hangindent=1em
\hangafter=1
$\bullet$ scalar index \hfil\break
Select one index.  No result dimension will correspond to {\bf ID}.

\hangindent=1em
\hangafter=1
$\bullet$ nil (or omitted) \hfil\break
Select the entire {\bf ID}.  One result dimension will match the {\bf ID}.

\hangindent=1em
\hangafter=1
$\bullet$ index range \hglue 1em \kbd{start:stop} or
\kbd{start:stop:step} \hfil\break
Select \kbd{start}, \kbd{start+step}, \kbd{start+2*step}, etc.  One
result dimension of length \kbd{1+(stop-start)/step} and origin 1
will correspond to {\bf ID}.  The default \kbd{step} is 1; it may be
negative.  In particular, \kbd{::-1} reverses the order of {\bf ID}.

\hangindent=1em
\hangafter=1
$\bullet$ index list \hfil\break
Select an arbitrary list of indices -- the index list can be any array of
integers.  The dimensions of the index list will replace the {\bf ID} in
the result.

\hangindent=1em
\hangafter=1
$\bullet$ pseudo-index \hglue 1em \kbd{-} \hfil\break
Insert a unit length dimension in the result which was not present in the
original array {\it x}.  There is no {\bf ID} for a \kbd{-} index.

\hangindent=1em
\hangafter=1
$\bullet$ rubber-index \hglue 1em \kbd{..} \hglue 1em or
                       \hglue 1em \kbd{*} \hfil\break
The {\bf ID} may be zero or more dimensions of {\it x}, forcing {it indexN\/}
to be the final actual index of {\it x}.  A \kbd{..} preserves the actual
indices, \kbd{*} collapses them to a single index.

\hangindent=1em
\hangafter=1
$\bullet$ range function {\it ifunc\/} or {\it ifunc\/}:range \hfil\break
Apply a range function to all or a subset of the {\bf ID}; the other
dimensions are ``spectators''; multiple {\it ifunc\/} are performed
successively from left to right.

Function results and expressions may be indexed directly, e.g.:
\hfil\break
\hglue0.5em{\it f\/}{\tt (}{\it a,b,c\/}{\tt )(}{\it index1,index2\/}{\tt )}
or
{\tt (}$2*x+1${\tt )(}{\it index1,index2,index3\/}{\tt )}

If the left hand operand of the \kbd{=} operator is an indexed array, the
right hand side is converted to the type of the left, and the specified
array elements are replaced.  Do not confuse this with the redefinition
operation {\it var}{\tt =}:

\sample{{\it x}{\tt (}{\it index1, index2, ..., indexN\/}{\tt )=} {\it expr}}
       {assign to a subarray of {\it x}}

\mysection{Array Conformability Rules}

Operands may be arrays, in which case the operation is performed on
each element of the array(s) to produce an array result.  Binary
operands need not have identical dimensions, but their dimensions must
be {\it conformable}.  Two arrays are conformable if their first
dimensions {\it match}, their second dimensions {\it match}, their
third dimensions {\it match}, and so on up to the number of dimensions in
the array with the fewer dimensions.  Two array dimensions {\it match\/}
if either of the following conditions is met:

$\bullet$ the dimensions have the same length

$\bullet$ one of the dimensions has unit length (1 element)

Unit length or missing dimensions are broadcast (by copying the single
value) to the length of the corresponding dimension of the other
operand.  The result of the operation has the number of dimensions of
the higher rank operand, and the length of each dimension is the
longer of the lengths in the two operands.

\mysection{Logical Operators}

Yorick supports C-style logical AND and OR operators.  Unlike
the arithmetic and comparison operators, these take only scalar
operands and return a scalar \kbd{int} result.  Their precedence
is between \kbd{|} and \kbd{=}.

The right operand is not evaluated at all if the value of the
left operand decides the result value; hence the left operand
may be used to determine whether the evaluation of the right
operand would lead to an error.

\keyindent=3em
\func{\&\&}{logical and (scalar \kbd{int} result)\hglue3em}
\func{||}{logical or (scalar \kbd{int} result)\hglue3em}
\endindentedkeys

The logical NOT operator takes an array or a scalar operand, returning
\kbd{int} 1 if the operand was zero, 0 otherwise.  Its precedence is
above \kbd{\^{}}.

\keyindent=3em
\func{!}{logical not (\kbd{int} result)\hglue3em}
\endindentedkeys

The ternary operator selects one of two values based on the value of a
scalar {\it condition}: \hfil\break
\hglue4em {\it condition} \kbd{?} {\it true\_expr} \kbd{:} {\it false\_expr}
\hfil\break
Its precedence is low, and it must be parenthesized in a function argument
list or an array index list to prevent confusion with the \kbd{:} in the
index range syntax.  Like \kbd{\&\&} and \kbd{||}, the expression which is
rejected is not evaluated at all.

\mysection{Calling Functions}

\sample{{\it f\/}\kbd{(}{\it arg1, ..., argN\/}\kbd{)}}
       {invoke {\it f} as a function}
\sample{{\it f\/}\kbd{, }{\it arg1, ..., argN\/}}
       {invoke {\it f} as a subroutine, discard return}

Arguments which are omitted are passed to the function as nil.  In addition
to positional arguments, a function (invoked by either of the above two
mechanisms).  Keyword arguments look like this:

\hglue2em
{\it f\/}\kbd{, }{\it arg1, keyA= exprA, keyB= exprB, arg2, ...\/}

where {\it keyA} and {\it keyB} are the names of keyword arguments of the
function {\it f}.  Omitted keywords are passed to {\it f} as nil values.
Keywords typically set optional values which have defaults.

\mysection{Defining Functions}

\def\indf{\hglue1em\kbd{func}}

A function of N dummy arguments is defined by:
\begindemo
\indf {\it func\_name\/}\kbd{(}{\it dummy1, dummy2, ..., dummyN\/}\kbd{)}
\hglue1em\kbd{\{}
\hglue2em {\it body\_statements}
\hglue1em\kbd{\}}
\enddemo

If the function has no dummy arguments, the first line of the
definition should read:
\begindemo
\indf {\it func\_name\/}
\enddemo

Mark output parameters with a \kbd{\&}, as {\it dummy2} here:
\begindemo
\indf {\it func\_name\/}\kbd{(}{\it dummy1, \kbd{\&}dummy2, dummy3\/}\kbd{)}
\enddemo

If the function will take keyword arguments, they must be listed after all
positional arguments and marked by a \kbd{=}:
\begindemo
\indf {\it func\_name\/}\kbd{(}{\it ..., dummyN, key1=, ..., keyN=}\kbd{)}
\enddemo

If the function allows an indeterminate number of positional arguments
(beyond those which can be named), place the special symbol \kbd{..} after
the final dummy argument, but before the first keyword.  For example, to
define a function which takes one positional argument, followed by an
indeterminate number of positional arguments, and one keyword, use:
\hfil\break
\indf\hglue.7em {\it func\_name\/}\kbd{(}{\it dummy1, \kbd{..}, key1=}\kbd{)}
\hfil\break
The function \kbd{more_args()} returns the number of unread actual
arguments corresponding to the \kbd{..} indeterminate dummy argument.
The function \kbd{next_arg()} reads and returns the next unread actual
argument, or nil if all have been read.

\mysection{Variable Scope}

\sample{\kbd{local} {\it var1, var2, ..., varN}}
       {give the {\it varI} local scope}
\sample{\kbd{extern} {\it var1, var2, ..., varN}}
       {give the {\it varI} external scope}

If a variable {\it var} has local scope within a function, any value
associated with {\it var} is temporarily replaced by nil on entry to
the function.  On return from the function, the external value of {\it
var} is restored, and the local value is discarded.

If a variable {\it var} has external scope within a function,
references to {\it var} within the function refer to the {\it var} in
the ``nearest'' calling function for which {\it var} has local scope
(that is, to the most recently created {\it var}).

The \kbd{*main*} function has no variables of local scope; all
variables created at this outermost level persist until they are
explicitly undefined or redefined.

Dummy or keyword arguments always have local scope.

In the absence of a \kbd{extern} or \kbd{local} declaration, a variable
{\it var} has local scope if, and only if, its first use within the
function is as the left operand of a redefinition, {\it var\/}\kbd{=}
{\it expr}.

\mysection{Returning from Functions}

\sample{\kbd{return} {\it expr}}{return {\it expr} from current function}
The {\it expr} may be omitted to return nil, which is the default
return value if no \kbd{return} statement is encountered.

\sample{\kbd{exit,} {\it msg}}{return from all functions, printing {\it msg}}
\sample{\kbd{error,} {\it msg}}{halt with error, printing {\it msg}}

\mysection{Compound Statements}

Yorick statements end with a \kbd{;} or end-of-line if the resulting
statement would make sense.

Several Yorick statements can be combined into a single compound
statement by enclosing them in curly braces:
\begindemo
\hglue1em \kbd{\{}
\hglue2em {\it statement1}
\hglue2em {\it statement2}
\hglue2em ...
\hglue1em \kbd{\}}
\enddemo
The bodies of most loops and \kbd{if} statements are compound.

\mysection{Conditional Execution}

A Yorick statement can be executed or not based on the value of a
scalar {\it condition} (0 means don't execute, non-0 means execute):
\begindemo
\hglue1em \kbd{if (}{\it condition\/}\kbd{)} {\it statementT}
\enddemo
or, more generally,
\begindemo
\hglue1em \kbd{if (}{\it condition\/}\kbd{)} {\it statementT}
\hglue1em \kbd{else} {\it statementF}
\enddemo

Several \kbd{if} statements may be chained as follows:
\begindemo
\hglue1em \kbd{if (}{\it condition1\/}\kbd{)} {\it statement1}
\hglue1em \kbd{else if (}{\it condition2\/}\kbd{)} {\it statement2}
\hglue1em \kbd{else if (}{\it condition3\/}\kbd{)} {\it statement3}
\hglue1em ...
\hglue1em \kbd{else} {\it statementF}
\enddemo

\mysection{Loops}

Yorick has three types of loops:
\begindemo
\hglue1em\kbd{while (}{\it condition\/}\kbd{)} {\it body\_statement}
\hglue1em\kbd{do} {\it body\_statement\/} \kbd{while (}{\it condition\/}\kbd{)}
\hglue1em\kbd{for (}{\it init\_expr \kbd{;} test\_expr \kbd{;} inc\_expr\/}\kbd{)} {\it body\_statement}
\enddemo
The {\it init\_expr} and {\it inc\_expr} of a \kbd{for} loop may be comma
delimited lists of expressions.  They or the {\it test\_expr} may be
omitted.  In particular, \kbd{for (;;) ...} means ``do forever''.  If
there is a {\it test\_expr}, the {\it body\_statement} of the \kbd{for} loop
will execute until it becomes false (possibly never executing).  After
each pass, but before the {\it test\_expr}, the {\it inc\_expr} executes.
A \kbd{for} loop to make N passes through its {\it body\_statement} might
look like this: \hfil\break
\hglue1em\kbd{for (i=1 ; i<=N ; i++)} {\it body\_statement}


Within a loop body, the following statements are legal:

\func{break}{exit the current loop now}
\func{continue}{abort the current pass through the current loop}

For more complex flow control, Yorick supports a \kbd{goto}:

\sample{\kbd{goto} {\it label}}{go to the statement after {\it label}}
\sample{{\it label\/}\kbd{:} {\it statement}}
       {mark {\it statement} as a \kbd{goto} target}

\copyrightnotice

\par\vfill\supereject

%----------------------------------------------------------------------------

\title{Yorick Function Reference}

\centerline{(for version 1)}

\mysection{Including Source Files}

\func{\#include "{\it filename.i\/}"}{insert contents of {\it filename}}

This is a parser directive, NOT an executable statement.  Yorick also
provides two forms of executable include statements:

\func{include, "{\it filename.i\/}"}{parse contents of {\it filename.i}}
\func{require, "{\it filename.i\/}"}{parse {\it filename.i} if not yet parsed}

The effect of the \kbd{include} function is not quite immediate, since
any tasks (\kbd{*main*} programs) generated cannot execute until the
task which called \kbd{include} finishes.

The \kbd{require} function should be placed at the top of a file which
represents a package of Yorick routines that depends on functions or
variables defined in another package {\it filename.i}.

The {\it filename.i\/} ends with a \kbd{.i} suffix by convention.

\mysection{Comments}

\hglue1em \kbd{/* Yorick comments begin with slash-asterisk, \hfil\break
\hglue2.5em  and end with asterisk-slash.  A comment \hfil\break
\hglue2.5em  of any size is treated as a single blank.  */}

Since \kbd{/* ... */} comments do not nest properly, Yorick supports
C++ style comments as well:
\begindemo
\hglue1em \kbd{statement   //} remainder of line is comment (C++)
\hglue1em \kbd{//} Prefix a double slash to each line to comment out
\hglue1em \kbd{//} a block of lines, which may contain comments.
\enddemo

\mysection{Issuing Shell Commands}

You can execute a system command, returning to Yorick when
the command completes, by prefixing the command line with \kbd{\$}:

\hglue1em \kbd{\$}any shell command line

This is a shorthand for the \kbd{system} function:

\func{system, {\it shell\_string}}{pass {\it shell\_string} to a system shell}

You need to use the \kbd{system} function if you want to compute the
{\it shell\_string}; otherwise \kbd{\$} is more convenient.

Note that the cd (change directory) shell command and its relatives
will not have any effect on Yorick's working directory.  Instead, use
Yorick's \kbd{cd} function to change it's working directory:

\func{cd, {\it path\_name}}{change Yorick's default directory}
\func{get_cwd()}{return Yorick's current working directory}

The following functions also relate to the operating system:

\func{get_home()}{return your home directory}
\func{get\_env({\it env\_string\/})}
     {return environment variable {\it env\_string\/}}
\func{get\_argv()}{return the command line arguments}

\shortcopyrightnotice

\mysection{Matrix Multiplication}

The \kbd{*} binary operator normally represents the product of its
operands element-by-element, following the same conformability rules
as the other binary operators.  However, by marking one dimension of
its left operand and one dimension of its right operand with \kbd{+},
\kbd{*} will be interpreted as a matrix multiply along the marked
dimensions.  The marked dimensions must have the same length.  The result
will have the unmarked dimensions of the left operand, followed by the
unmarked dimensions of the right operand.

For example, if \kbd{x} is a 12-by-25-by-35 array, \kbd{y} and \kbd{z} are
vectors of length 35, and \kbd{w} is a 9-by-12-by-7 array, then:

\func{x(,,+)*y(+)}{is a 12-by-25 array}
\func{y(+)*z(+)}{is the inner product of \kbd{y} and \kbd{z}}
\func{x(+,,)*w(,+,)}{is a 25-by-35-by-9-by-7 array}

\mysection{Using Pointers}

A scalar of type \kbd{pointer} points to a Yorick array of any data
type or dimensions.  Unary \kbd{\&} returns a \kbd{pointer} to its
argument, which can be any array valued expression.  Unary \kbd{*}
dereferences its argument, which must be a scalar of type
\kbd{pointer}, returning the original array.  A dereferenced pointer
may itself be an array of type \kbd{pointer}.  The unary \kbd{\&} and
\kbd{*} bind more tightly than any other Yorick operator except
\kbd{.} and \kbd{->} (the member extraction operators), and array
indexing {\it x\/}\kbd{(..)}:

\keyindent=2em
\sample{\kbd{\&}{\it expr}}
       {return a scalar \kbd{pointer} to {\it expr}\hglue2em}
\sample{\kbd{*}{\it expr}}
       {dereference {\it expr}, a scalar pointer\hglue2em}
\endindentedkeys

Since a \kbd{pointer} always points to a Yorick array, Yorick can
handle all necessary memory management.  Dereference \kbd{*} or
\kbd{->}, copy by assignment \kbd{=}, or compare to another pointer
with \kbd{==} or \kbd{!=} are the only legal operations on a \kbd{pointer}.
A pointer to a temporary {\it expr} makes sense and may be useful.

The purpose of the \kbd{pointer} data type is to deal with several related
objects of different types or shapes, where the type or shape changes,
making \kbd{struct} inapplicable.

\mysection{Instancing Data Structures}

Any data type {\it type\_name} --- basic or defined by \kbd{struct}
--- serves as a type converter to that data type.  A nil argument is
converted to a scalar zero of the specified type.  Keywords matching
the member names can be used to assign non-zero values to individual
members:

\func{{\it type\_name}()}{scalar instance of {\it type\_name}, zero value}
\func{{\it type\_name}({\it memb\_name\_1}={\it expr\_1},...)}
     {scalar {\it type\_name}}

The \kbd{.} operator extracts a member of a data structure.  The
\kbd{->} operator dereferences a pointer to the data structure before
extracting the member.  For example:
\beginexample
struct Mesh \{ pointer x, y; long imax, jmax; \}
mesh= Mesh(x=\&xm,
           imax=dimsof(xm)(1), jmax=dimsof(xm)(2));
mesh.y= \&ym;        mptr= \&mesh;
print, mesh.x(2,1:10), mptr->y(2,1:10);
\endexample

\mysection{Index Range Functions}

Range functions are executed from left to right if more than one
appears in a single index list.  The following range functions reduce
the rank of the result, like a scalar index:

\beginindentedkeys
\func{min}{minimum of values along index}
\func{max}{maximum of values along index}
\func{sum}{sum of values along index}
\func{avg}{average of values along index}
\func{rms}{root mean square of values along index}
\func{ptp}{peak-to-peak of values along index}
\func{mnx}{index at which minimum occurs}
\func{mxx}{index at which maximum occurs}
\endindentedkeys

The following functions do not change the rank of the result, like an
index range.  However, the length of the index is changed as indicated
by +1, -1, or 0 (no change):

\beginindentedkeys
\func{cum, psum}{+1, 0, partial sums of values along index}
\func{dif}{-1, pairwise differences of adjacent values}
\func{zcen}{-1, pairwise averages of adjacent values}
\func{pcen}{+1, pairwise averages of adjacent interior values}
\func{uncp}{-1, inverse of pcen (point center) operation}
\endindentedkeys

For example, given a two-dimensional array \kbd{x}, \kbd{x(min, max)}
returns the largest of the smallest elements along the first dimension.
To get the smallest of the largest elements along the second dimension,
use \kbd{x(, max)(min)}.

\mysection{Elementary Functions}

\func{abs, sign}{absolute value, arithmetic sign}
\func{sqrt}{square root}
\func{floor, ceil}{round down, round up to integer}
\func{conj}{complex conjugation}
\func{pi}{the constant 3.14159265358979323846...}
\func{sin, cos, tan}{trigonometric functions (of radians)}
\func{asin, acos, atan}{inverse trigonometric functions}
\func{sinh, cosh, tanh, sech, csch}{hyperbolic functions}
\func{asinh, acosh, atanh}{inverse hyperbolic functions}
\func{exp, log, log10}{exponential and logarithmic functions}
\func{min, max}{find minimum, maximum of array}
\func{sum, avg}{find sum, average of array}
\func{random}{random number generator}

The \kbd{atan} function takes one or two arguments; \kbd{atan(t)}
returns a value in the range $(-\pi/2,\pi/2])$, while \kbd{atan(y,x)}
returns the counterclockwise angle from $(1,0)$ to $(x,y)$ in the
range $(-\pi,\pi])$.

The \kbd{abs} function allows any number of arguments; for example,
\kbd{abs(x, y, z)} is the same as \kbd{sqrt(x\^{}2 + y\^{}2 + z\^{}2)}.
The \kbd{sign} satisfies \kbd{sign(0)==1} and \kbd{abs(z)*sign(z)==z}
always (even when \kbd{z} is \kbd{complex}).

The \kbd{min} and \kbd{max} functions return a scalar result when
presented with a single argument, but the pointwise minimum or maximum
when presented with multiple arguments.

The \kbd{min}, \kbd{max}, \kbd{sum}, and single argument \kbd{abs}
functions return integer results when presented integer arguments; the
other functions will promote their arguments to a real type and return
reals.

\mysection{Information About Variables}

\func{print, {\it var1, var2, ...}}{print the values of the {\it varI}}
\func{info, {\it var}}{print a description of {\it var}}
\func{dimsof({\it x\/})}{returns [\# dimensions, length1, length2, ...]}
\func{orgsof({\it x\/})}{returns [\# dimensions, origin1, origin2, ...]}
\func{numberof({\it x\/})}
     {returns number of elements (product of \kbd{dimsof})}
\func{typeof({\it x\/})}{returns name of data type of {\it x}}
\func{structof({\it x\/})}{returns data type of {\it x}}

\func{is_array({\it x\/})}{returns 1 if {\it x} is an array, else 0}
\func{is_func({\it x\/})}{returns 1 or 2 if {\it x} is an function, else 0}
\func{is_void({\it x\/})}{returns 1 if {\it x} is nil, else 0}
\func{is_range({\it x\/})}{returns 1 if {\it x} is an index range, else 0}
\func{is_stream({\it x\/})}{returns 1 if {\it x} is a binary file, else 0}
\func{am_subroutine()}{1 if current function invoked as subroutine}

The \kbd{print} function returns a string array of one string per line if
it is invoked as a function.  Using \kbd{print} on files, bookmarks, and
other objects usually produces some sort of useful description.  Also,
\kbd{print} is the default function, so that
\begindemo
\hglue2em{\it expr}
\enddemo

is equivalent to \kbd{print,} {\it expr\/} (if {\it expr\/} is not a function).

\mysection{Reshaping Arrays}

\func{reshape, {\it x, type\_name, dimlist}}
     {masks shape of {\it x}}

Don't try to use this unless (1) you're an expert, and (2) you're desperate.
It is intended mainly for recovering from misfeatures of other programs,
although there are a few legitimate uses within Yorick.

\mysection{Logical Functions}

\func{allof({\it x\/})}{returns 1 if every element of {\it x} is non-zero}
\func{anyof({\it x\/})}{returns 1 if any element of {\it x} is non-zero}
\func{noneof({\it x\/})}{returns 1 if no element of {\it x} is non-zero}
\func{nallof({\it x\/})}{returns 1 if any element of {\it x} is zero}

\func{where({\it x\/})}{returns list of indices where {\it x} is non-zero}
\func{where2({\it x\/})}{human-readable variant of \kbd{where}}

\mysection{Interpolation and Lookup Functions}

In the following function, {\it y\/} and {\it x\/} are one-dimensional
arrays which determine a piecewise linear function {\it y(x)}.  The {\it x\/}
must be monotonic.  The {\it xp\/} (for {\it x\/}-prime) can be an
array of any dimensionality; the dimensions of the result will be the
same as the dimensions of {\it xp}.

\func{digitize({\it xp, x\/})}
     {returns indices of {\it xp\/} values in {\it x}}
\func{interp({\it y, x, xp\/})}
     {returns {\it yp}, {\it xp\/} interpolated into {\it y(x)}}
\func{integ({\it y, x, xp\/})}
     {returns the integrals of {\it y(x)\/} from {\it x\/}(1) to {\it xp}}

Note that \kbd{integ} is really an area-conserving interpolator.  If
the {\it xp} coincide with {\it x}, you probably want to use \hfil\break
\hglue2em \kbd{(}{\it y\/}\kbd{(zcen)*}{\it x\/}\kbd{(dif))(cum)} \hfil\break
instead.

The on-line \kbd{help} documentation for \kbd{interp} describes how to
use \kbd{interp} and \kbd{integ} with multidimensional {\it y} arrays.

\mysection{Sorting}

\func{sort({\it x\/})}{return index list which sorts {\it x}}

That is, {\it x\/}\kbd{(sort({\it x\/}))} will be in non-decreasing order
({\it x\/} can be an integer, real, or string array).  The on-line
\kbd{help} documentation for \kbd{sort} explains how to sort
multidimensional arrays.

\func{median({\it x\/})}{return the median of the {\it x\/} array}

Consult the on-line \kbd{help} documentation for \kbd{median} for use
with multidimensional arrays.

\mysection{Transposing}

\func{transpose({\it x\/})}{transpose the 2-D array {\it x}}
\func{transpose({\it x, permutation\/})}{general transpose}

The {\it permutation\/} is a comma delimited list of cyclic permutations
to be applied to the indices of {\it x}.  Each cyclic permutation may
be:

\hangindent=1em
\hangafter=1
$\bullet$ a list of dimension numbers [{\it n1, n2, ..., nN\/}] \hfil\break
to move dimension number {\it n1\/} (the first dimension is number 1, the
second number 2, and so on) to dimension number {\it n2\/}, {\it n2\/} to
{\it n3\/}, and so on, until finally {\it nN\/} is moved to {\it n1\/}.

\hangindent=1em
\hangafter=1
$\bullet$ a scalar integer {\it n} \hfil\break
to move dimension number 1 to dimension number {\it n\/}, 2 to {\it n\/}+1,
and so on, cyclically permuting all of the indices of {\it x}.

In either case, {\it n\/} or {\it nI\/} can be non-positive to refer to
indices relative to the {\bf final} dimension of {\it x}.  That is, 0 refers
to the final dimension of {\it x}, -1 to the next to last dimension, and so
on.  Thus, \hfil\break
\hglue2em \kbd{transpose({\it x}, [1,0])} \hfil\break
swaps the first and last dimensions of {\it x}.

\mysection{Manipulating Strings}

Yorick type \kbd{string} is a pointer to a 0-terminated
array of \kbd{char}.  A string with zero characters -- \kbd{""} -- differs
from a zero pointer -- \kbd{string(0)}.  A \kbd{string} variable {\it s\/}
can be converted to a \kbd{pointer} to a 1-D array of \kbd{char}, and such
a \kbd{pointer} {\it p\/} can be converted back to a \kbd{string}:
\begindemo
\hglue2em \kbd{{\it p\/}= pointer({\it s\/});}
\hglue2em \kbd{{\it s\/}= string({\it p\/});}
\enddemo
These conversions copy the characters, so you can't use the pointer {\it p\/}
to alter the characters of {\it s}.  The \kbd{strchar} function directly
converts between \kbd{string} and  \kbd{char} data.

Given a string or an array of strings {\it s\/}:

\func{strlen({\it s\/})}{number of characters in each element of {\it s\/}}
\func{strpart({\it s, rng})}{returns substring {\it rng\/} of {\it s\/}}
\func{strfind({\it pat, s\/})}
          {returns {\it rng\/} where {\it pat} occurs in {\it s\/}}
\func{strgrep({\it pat, s\/})}{regular expression version of \kbd{strfind}}
\func{strword({\it s, delims\/})}{returns {\it rng\/} of words in {\it s}}
\func{streplace({\it s, rng, \it to})}
          {replaces {\it rng\/} of {\it s\/} by {\it to}}

The \kbd{strfind}, \kbd{strgrep}, and \kbd{strword} functions return
{\it rng} lists suitable as inputs to \kbd{strpart} or \kbd{streplace}.
Other string manipulation functions include \kbd{strmatch}, \kbd{strglob},
\kbd{strcase}, and \kbd{strtrim}.  Use \kbd{help} for details.

\mysection{Advanced Array Indexing}

\hangindent=1em
\hangafter=1
$\bullet$ A scalar index or the \kbd{start} and \kbd{stop} of an index
range may be non-positive to reference the elements near the end of a
dimension.  Hence, 0 refers to the final element, -1 refers to the next
to last element, -2 to the element before that, and so on.  For example,
{\it x\/}\kbd{(2:-1)} refers to all but the first and last elements of
the 1-D array {\it x}.  This convention does {\bf NOT} work for an index
list.

\hangindent=1em
\hangafter=1
$\bullet$ A range function {\it ifunc} may be followed by a colon and an
index range \kbd{start:stop} or \kbd{start:stop:step} in order to restrict
the indices to which the range function applies to a subset of the entire
dimension.  Hence, {\it x\/}\kbd{(min:2:-1)} returns the minimum of all
the elements of the 1-D array {\it x}, excluding the first and last elements.

\hangindent=1em
\hangafter=1
$\bullet$ An index specified as a scalar, the \kbd{start} or \kbd{stop} of
an index range, or an element of an index list may exceed the length of the
indexed dimension {\bf ID}, provided that the entire indexing operation does
not overreach the bounds of the array.  Thus, if {\it y\/} is a 5-by-6 array,
then {\it y\/}\kbd{(22)} refers to the same datum as  {\it y\/}\kbd{(2,5)}.

\hangindent=1em
\hangafter=1
$\bullet$ The expression {\it z\/}\kbd{(..)} --- using the rubber-index
operator \kbd{..} --- refers to the entire array {\it z\/}.  This is
occasionally useful as the left hand side of an assignment statement in
order to force broadcasting and type conversion of the right hand
expression to the preallocated type and shape {\it z\/}.

\hangindent=1em
\hangafter=1
$\bullet$ The expression {\it z\/}\kbd{(*)} --- using the rubber-index
operator \kbd{*} --- collapses a multidimensional array {\it z\/} into
a one-dimensional array.  Even more useful as {\it z\/}\kbd{(*,)} to
preserve the final index of an array and force a two-dimensional result.

\mysection{Generating Simple Meshes}

Many Yorick calculations begin by defining an array of {\it x\/} values
which will be used as the argument to functions of a single variable.
The easiest way to do this is with the \kbd{span} or \kbd{spanl} function:
\begindemo
\hglue2em \kbd{x= span(x\_min, x\_max, 200);}
\enddemo
This gives 200 points equally spaced from \kbd{x\_min} to \kbd{x\_max}.

A two dimensional rectangular grid is most easily obtained as follows:
\begindemo
\hglue1em \kbd{x= span(x\_min, x\_max, 50)(, -:1:40);}
\hglue1em \kbd{y= span(y\_min, y\_max, 40)(-:1:50, );}
\enddemo
This gives a 50-by-40 rectangular grid with \kbd{x} varying fastest.
Such a grid is appropriate for exploring the behavior of a function of
two variables.  Higher dimensional meshes can be built in this way, too.

\copyrightnotice

\par\vfill\supereject

%----------------------------------------------------------------------------

\title{Yorick I/O Reference}

\centerline{(for version 1)}

\mysection{Opening and Closing Text Files}

\func{{\it f\/}= open({\it filename, mode\/})}
     {open {\it filename\/} in {\it mode}}
\func{close, {\it f}}
     {close file {\it f} (automatic if {\it f\/} redefined)}

The {\it mode} is a string which announces the type of operations you
intend to perform: \kbd{"r"} (the default if {\it mode\/} is omitted)
means read operations only, \kbd{"w"} means write only, and destroy any
existing file {\it filename}, \kbd{"r+"} means read/write, leaving any
existing file {\it filename\/} intact.  Other {\it mode\/} values are
also meaningful; see \kbd{help}.

The file variable {\it f\/} is a distinct data type in Yorick; text
files have a different data type than binary files.  The \kbd{print}
or \kbd{info} function will describe the file.  The \kbd{close}
function is called implicitly when the last reference to a file disappears.

\mysection{Reading Text}

\func{read, {\it f, var1, var2, ..., varN}}
     {reads the {\it varI} from file {\it f}}
\func{read, {\it var1, var2, ..., varN}}
     {reads the {\it varI\/} from keyboard}
\func{read\_n, {\it f, var1, var2, ..., varN}}
     {read, skip non-numeric tokens}
\func{rdline({\it f\/})}
     {returns next line from file {\it f}}
\func{rdline({\it f, n\/})}
     {returns next {\it n\/} lines from file {\it f}}
\func{sread, {\it s, var1, var2, ..., varN}}
     {reads the {\it varI\/} from string {\it s}}

The data type and dimensions of the {\it varI\/} determine how the text
is converted as it is read.  The {\it varI\/} may be arrays, provided the
arrays have identical dimensions.  If the {\it varI\/} have length {\bf L},
then the \kbd{read} is applied as if  called {\bf L} times, with successive
elements of each of the {\it varI\/} read on each call.

The \kbd{read} function takes the \kbd{prompt} keyword to set the prompt
string, which defaults to \kbd{"read> "}.

Both \kbd{read} and \kbd{sread} accept the \kbd{format} keyword.  The
format is a string containing conversion specifiers for the {\it varI}.
The number of conversion specifiers should match the number of {\it varI}.
If the {\it varI\/} are arrays, the format string is applied repeatedly
until the arrays are filled.

Read format strings in Yorick have (nearly) the same meaning as the
format strings for the ANSI standard C library \kbd{scanf} routine.  In
brief, a format string consists of:

\hangindent=1em
\hangafter=1
$\bullet$ whitespace \hfil\break
means to skip any number of whitespace characters in the source

\hangindent=1em
\hangafter=1
$\bullet$ characters other than whitespace and \kbd{\%} \hfil\break
must match characters in the source exactly or the read operation stops

\hangindent=1em
\hangafter=1
$\bullet$ conversion specifiers beginning with \kbd{\%} \hfil\break
each specifier ends with one of the characters \kbd{d} (decimal integer),
\kbd{i} (decimal, octal, or hex integer), \kbd{o} (octal integer),
\kbd{x} (hex integer), \kbd{s} (whitespace delimited string),
any of \kbd{e}, \kbd{f}, or \kbd{g} (real),
\kbd{[{\it xxx}]} to match the longest string of characters in the list,
\kbd{[\^{}{\it xxx}]} to match the longest string of characters not in
the list, or \kbd{\%} (the \kbd{\%} character -- not a conversion)

\shortcopyrightnotice

\mysection{Writing Text}

\func{write, {\it f, expr1, expr2, .., exprN}}
     {writes the {\it exprI\/} to file {\it f}}
\func{write, {\it expr1, expr2, .., exprN}}
     {writes the {\it exprI\/} to terminal}
\func{swrite({\it expr1, expr2, .., exprN\/})}
     {returns the {\it exprI\/} as a string}

The \kbd{swrite} function returns an array of strings --- one string
for each line that would have been produced by the \kbd{write} function.

The {\it exprI\/} may be arrays, provided the arrays are conformable.
In this case, the {\it exprI\/} are broadcast to the same length {\bf L},
then the \kbd{write} is applied as if called {\bf L} times, with
successive elements of the {\it exprI\/} written on each call.

Both functions accept an optional \kbd{format} keyword.  Write format
strings in Yorick have (nearly) the same meaning as the format strings
for the ANSI stacndard C library \kbd{printf} routine.  In brief, a format
string consists of:

\hangindent=1em
\hangafter=1
$\bullet$ characters other than \kbd{\%} \hfil\break
which are copied directly to output

\hangindent=1em
\hangafter=1
$\bullet$ conversion specifiers beginning with \kbd{\%} \hfil\break
of the general format \kbd{\%}{\it FW.PSC\/} where: \hfil\break
{\it F\/} is zero or more of the optional flags \kbd{-} (left justify),
\kbd{+} (always print sign), (space) (leave space if +), \kbd{0}
(leading zeroes) \hfil\break
{\it W\/} is an optional decimal integer specifying the minimum number
of characters to output \hfil\break
{\it .P\/} is an optional decimal integer specifying the number of
digits of precision \hfil\break
{\it S\/} is one of \kbd{h}, \kbd{l}, or \kbd{L}, ignored by Yorick \hfil\break
{\it C\/} is \kbd{d} or \kbd{i} (decimal integer), \kbd{o} (octal integer),
\kbd{x} (hex integer), \kbd{f} (fixed point real), \kbd{e} (scientific
real), \kbd{g} (fixed or scientific real), \kbd{s} (string), \kbd{c}
(ASCII character), or \kbd{\%} (the \kbd{\%} character -- not a conversion)

For example,
\beginexample
> write, format="   tp \%7.4f \%e\\n", [1.,2.], [.5,.6]
   tp  1.0000 5.000000e-01
   tp  2.0000 6.000000e-01
>
\endexample

\mysection{Positioning a Text File}

The \kbd{write} function always appends to the end of a file.

A sequence of \kbd{read} operations may be intermixed with \kbd{write}
operations on the same file.  The two types of operations do not
interact.

The \kbd{read} and \kbd{rdline} functions read the file in complete
lines; a file cannot be positioned in the middle of a line -- although
the \kbd{read} function may ignore a part of the last line read,
subsequent \kbd{read} operations will begin with the next full line.
The following functions allow the file to be reset to a previously
read line.

\func{backup, {\it f}}{back up file {\it f\/} one line}
\func{{\it m\/}= bookmark({\it f\/})}
     {record position of file {\it f\/} in {\it m}}
\func{backup, {\it f}, {\it m}}{back up file {\it f\/} to {\it m}}

The bookmark {\it m\/} records the current position of the file; it
has a distinct Yorick data type, and the \kbd{info} or \kbd{print}
function can be used to examine it.  Without a bookmark, the \kbd{backup}
function can back up only a single line.

\mysection{Opening and Closing Binary Files}

\func{{\it f\/}= openb({\it filename\/})}
     {open {\it filename\/} read-only}
\func{{\it f\/}= updateb({\it filename\/})}
     {open {\it filename\/} read-write}
\func{{\it f\/}= createb({\it filename\/})}
     {create the binary file {\it filename\/}}
\func{close, {\it f}}
     {close file {\it f}}

A binary file {\it f\/} has a Yorick data type which is distinct from a text
file.  The \kbd{info} and \kbd{print} functions describe {\it f\/}.  The
\kbd{close} function will be called implicitly when the last reference to
a file disappears, e.g.-- if {\it f} is redefined.

The data in a binary file is organized into named variables, each of
which has a data type and dimensions.  The \kbd{.} operator, which extracts
members from a structure instance, accepts binary files for its left
operand.  Thus:
\beginexample
\hglue2em f= updateb("foo.bar");
\hglue2em print, f.var1, f.var2(2:8,::4);
\hglue2em f.var3(2,5)= 3.14;
\hglue2em close, f;
\endexample
Opens a file, prints \kbd{var1} and a subarray of \kbd{var2}, sets one
element of \kbd{var3}, then closes the file.

The \kbd{show} command prints an alphabetical list of the variables
contained in a file:

\func{show, {\it f}}{shows the variables in file {\it f}}
\func{show, {\it f, pat}}{show only names starting with {\it pat}}
\func{get_vars({\it f\/})}
     {returns pointers to complete name lists for {\it f}}

\mysection{Saving and Restoring Variables}

\func{save, {\it f, var1, var2, ..., varN}}
     {saves the {\it varI\/} in binary file {\it f}}
\func{restore, {\it f, var1, var2, ..., varN}}
     {restores the {\it varI\/} from {\it f}}
\func{save, {\it f}}
     {saves all array variables in binary file {\it f}}
\func{restore, {\it f}}
     {restores all variables from binary file {\it f}}

Unlike {\it f.varI\/}\kbd{=} {\it expr}, the \kbd{save} function will
create the variable {\it varI\/} in the file {\it f\/} if it does not
already exist.

The \kbd{restore} function redefines the in-memory
{\it varI\/}.  If several binary
files are open simultaneously, the {\it f.varI\/} syntax will be more
useful for reading variables than the \kbd{restore} function.

Note that a single command can be used to create a binary file, save
variables {\it varI\/} in it, and close the file:
\beginexample
\hglue2em save, createb({\it filename\/}), {\it var1, var2, ..., varN}
\endexample
A similar construction using \kbd{restore} and \kbd{openb} is also useful.

\mysection{Reading History Records}

A binary file may have two groups of variables: those belonging to a set
of history records, and non-record variables.
The record variables may have different values in each record.  The records
are labeled by (optional) time and cycle numbers:

\func{jt, {\it time}}
     {advance all open record files to record nearest {\it time}}
\func{jt, {\it f, time}}
     {advance file {\it f\/} to record nearest {\it time}}
\func{jc, {\it f, ncyc}}
     {advance file {\it f\/} to record nearest {\it ncyc}}
\func{get\_times({\it f\/})}{return list of record times}
\func{get\_ncycs({\it f\/})}{return list of record cycles}

\mysection{Writing History Records}

To write a family of files containing history records:

\hangindent=1em
\hangafter=1
1.\hglue.5emCreate the file using \kbd{createb}.

\hangindent=1em
\hangafter=1
2.\hglue.5emWrite all of the non-record (time independent) variables to the
file using \kbd{save}.

\hangindent=1em
\hangafter=1
3.\hglue.5emCreate a record which will correspond to time {\it time} and cycle
{\it ncyc} for future \kbd{jt} and \kbd{jc} commands.  Use:

\func{add_record, {\it f, time, ncyc}}{make new record at {\it time, ncyc}}

\hangindent=1em
\hangafter=1
4.\hglue.5emWrite all record (time dependent) variables to the file using
\kbd{save}.  After the first \kbd{add_record}, \kbd{save} will create and
store record variables instead of non-record variables as in step 2.

\hangindent=1em
\hangafter=1
5.\hglue.5emRepeat steps 3 and 4 for each new record you wish to add to the
file.  For the second and subsequent records, \kbd{save} will not allow
variables which were not written to the first record, or whose data type or
shape has changed since the first record.  That is, the structure of all
history records in a file must be identical.  Use type \kbd{pointer} variables
to deal with data which changes in size, shape, or data type.

After each \kbd{add\_record}, any number of \kbd{save} commands may be used
to write the record.

If the current member of a history record file family has at least one
record, and if the next record would cause the file to exceed the maximum
allowed file size, \kbd{add_record} will automatically form the next member
of the family.  The maximum family member file size defaults to 4 MBytes, but:

\func{set\_filesize, {\it f, n\_bytes}}{set family member size}

\mysection{Opening Non-PDB Files}

Yorick expects binary files to be in PDB format, but it can be trained to
recognize any file whose format can be described using its Contents Log
file description language.  The basic idea is that if you can figure out
how to compute the names, data types, dimensions, and disk addresses of the
data in the file, you can train Yorick to open the file; once open, all of
Yorick's machinery to manipulate the data will grind away as usual.

The following functions can be used to teach Yorick about a non-PDB file;
use \kbd{help} to get complete details:

\func{\_read, {\it f, address, var}}{raw binary read}
\func{install\_struct, {\it f, struct\_name, size, align, order, layout}}{}
\func{}{define a primitive data type}
\func{add_variable, {\it f, address, name, type, dimlist}}
     {add a variable}
\func{add_member, {\it f, struct\_name, offset, name, type, dimlist}}{}
\func{}{build up a data structure}
\func{install\_struct, {\it f, struct\_name}}{finish \kbd{add_member} struct}
\func{data\_align, {\it f, alignment}}{specify default data alignment}
\func{struct\_align, {\it f, alignment}}{specify default struct alignment}
\func{add\_record, {\it f, time, ncyc, address}}{declare record}
\func{add\_next\_file, {\it f, filename}}{open new family member}

To write a plain text description of any binary file, use:

\func{dump\_clog, {\it f, clogname}}{write Contents Log for {\it f}}

\mysection{Making Plots}

\func{plg, {\it y, x}}{plot graph of 1-D {\it y} vs.\ {\it x}}
\func{plm, {\it mesh\_args}}{plot quadrilateral mesh}
\func{plc, {\it z, mesh\_args}}{plot contours of {\it z}}
\func{plf, {\it z, mesh\_args}}{plot filled mesh, filling with {\it z}}
\func{plv, {\it v, u, mesh\_args}}{plot vector field ({\it u,v)\/})}
\func{pli, {\it z, x0, y0, x1, y1}}{plot image {\it z}}
\func{pldj, {\it x0, y0, x1, y1}}{plot disjoint lines}
\func{plt, {\it text, x, y}}{plot {\it text\/} at ({\it x,y\/})}

The {\it mesh\_args\/} may be zero, two, or three arguments as follows:

\hangindent=1em
\hangafter=1
$\bullet$ omitted to use the current default mesh set by:

\func{plmesh, {\it mesh\_args}}{set default quadrilateral mesh}
\func{plmesh}{delete current default quadrilateral mesh}

\hangindent=1em
\hangafter=1
$\bullet$ {\it y, x} \hfil\break
To set mesh points to ({\it x, y\/}), which must be 2-D arrays of the same
shape, with at least two elements in each dimension.

\hangindent=1em
\hangafter=1
$\bullet$ {\it y, x, ireg} \hfil\break
To set mesh points to ({\it x, y\/}), as above, with a region number array
{\it ireg}.  The {\it ireg\/} should be an integer array of the same
shape as {\it y\/} and {\it x}, which has a non-zero ``region number'' for
every meaningful zone in the problem.  The first row and column of
{\it ireg\/} do not correspond to any zone, since there are one fewer zones
along each dimension than points in {\it y\/} and {\it x}.

The \kbd{plc} command accepts the \kbd{levs} keyword to specify the list of
{\it z\/} values to be contoured; by default, eight linearly spaced levels
are generated.

The \kbd{plc} and \kbd{plmesh} commands accept the \kbd{triangle} keyword
to specify a detailed triangulation map for the contouring algorithm.  Use
the \kbd{help, triangle} for details.

The \kbd{plv} command accepts the \kbd{scale} keyword to specify the
scaling factor to be applied to ({\it u, v\/}) before rendering the vectors
in ({\it x, y\/}) space; by default, the vector lengths are chosen to be
comparable to typical zone dimensions.

The \kbd{plm} command accepts the \kbd{boundary} keyword, which should be
set to 1 if only the mesh boundary, rather than the mesh interior, is to
be plotted.

The \kbd{plm}, \kbd{plc}, \kbd{plf}, and \kbd{plv} commands accept the
\kbd{region} keyword to restrict the plot to only one region of the
mesh, as numbered by the {\it ireg\/} argument.  The default \kbd{region}
is 0, which is interpreted to mean the every non-0 region of the mesh.

The \kbd{pli} command produces a cell array; the {\it x0, y0, x1, y1\/},
which are optional, specify the coordinates of the opposite corners of the
cell array.

Numerous other keywords adjust the style of lines, text, etc.

\mysection{Plot Paging and Hardcopy}

\func{fma}{frame advance --- next plot command will clear picture}
\func{hcp}{send current picture to hardcopy file}
\func{hcpon}{do automatic \kbd{hcp} at each \kbd{fma}}
\func{hcpoff}{require explicit \kbd{hcp} for hardcopy}
\func{hcp\_out}{print and destroy current hardcopy file}
\func{animate}{toggle animation mode (see \kbd{help})}

\mysection{Setting Plot Limits}

\func{logxy, {\it xflag, yflag}}{set log or linear axis scaling}

\func{limits, {\it xmin, xmax, ymin, ymax}}{set plot limits}
\func{limits, {\it xmin, xmax}}{set plot x-limits}
\func{range, {\it ymin, ymax}}{set plot y-limits}
\func{{\it l\/}= limits()}{save current plot limits in {\it l\/}}
\func{limits, {\it l}}{restore plot limits saved in {\it l\/}}

The four plot limits can be numbers to fix them at specific values, or
the string \kbd{"e"} to specify extreme values.  The \kbd{limits}
command accepts the keywords \kbd{square}, \kbd{nice}, and
\kbd{restrict}, which control how extreme values are computed.

Plot limits may also be set by point-and-click in the X window.  The
left button zooms in, middle button pans, and right button zooms out.
Refer \kbd{help} on \kbd{limits} for details.

\mysection{Managing Graphics Windows}

\func{window, {\it n}}{switch to window {\it n\/} (0-7)}
\func{winkill, {\it n}}{delete window {\it n\/} (0-7)}

The \kbd{window} command takes several keywords, for example:
\kbd{dpi=75} makes a smaller X window than the default \kbd{dpi=100},
\kbd{private=1} forces use of private instead of shared colors,
\kbd{dump=1} forces the palette to be dumped to the \kbd{hcp} file,
and \kbd{style} specifies an alternative style sheet for
tick and label style (\kbd{"work.gs"} and \kbd{"boxed.gs"} are two
predefined style sheets).

The \kbd{plf} and \kbd{pli} commands require a color palette:

\func{palette, {\it name}}{load the standard palette {\it name}}
\func{palette, {\it r, g, b}}{load a custom palette}
\func{palette, query=1, {\it r, g, b}}{retrieve current palette}

Standard palette names: \kbd{"earth.gp"} (the default), \kbd{"gray.gp"},
\kbd{"yarg.gp"}, \kbd{"stern.gp"}, \kbd{"heat.gp"}, and \kbd{"rainbow.gp"}.

\mysection{Graphics Query, Edit, and Defaults}

\func{plq}{query (print) legends for current window}
\func{plq, {\it i}}{query properties of element {\it i}}
\func{pledit, {\it key\_list}}{change properties of queried element}
\func{pldefault, {\it key\_list}}{set default window and element properties}

The keywords which regulate the appearance of graphical primitives
include (each has a \kbd{help} entry):

\func{legend}{string to use for legend}
\func{hide}{non-zero to skip element}
\func{type}{\kbd{"solid", "dash", "dot", "dashdot"}, etc.}
\func{width}{line width, default 1.0}
\func{color}{\kbd{"fg"} (default), \kbd{"red", "green", "blue"}, etc.}
\func{marks, marker, mspace, mphase, msize}{line markers}
\func{rays, rspace, rphase, arroww, arrowl}{line ray arrows}
\func{closed, smooth}{more line properties}
\func{font, height, opaque, path, justify}{text properties}
\func{hollow, aspect}{vector properties for \kbd{plv}}

\copyrightnotice

\bye

% Local variables:
% compile-command: "tex refs"
% End:
