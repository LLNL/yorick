\input texinfo.tex    @c -*-texinfo-*-
@comment %**start of header (This is for running Texinfo on a region.)
@setfilename yorick.info
@settitle Yorick
@dircategory Math
@direntry
* Yorick: (yorick).       Yorick interpreted language.
@end direntry
@setchapternewpage odd
@comment %**end of header (This is for running Texinfo on a region.)

@comment $Id: yorick.tex,v 1.4 2009-12-05 22:11:14 dhmunro Exp $
@comment -----------------------------------------------------------------

@copying
This file documents Yorick, an interpreted language specialized for
numerical analysis.  Yorick can prepare input to or process output from
another physics simulation code.  New simulation codes can also be
constructed as a part of the Yorick interpreted environment.

@b{BSD License}

Copyright @copyright{} 2005, The Regents of the University of California.
Produced at the Lawrence Livermore National Laboratory.
Written by David H. Munro <dhmunro at users.sourceforge.net>.
UCRL-CODE-155996
All rights reserved.

Redistribution and use in source and binary forms, with or without
modification, are permitted provided that the following conditions are
met:

@itemize @bullet
@item
Redistributions of source code must retain the above copyright
notice, this list of conditions and the disclaimer below.
@item
Redistributions in binary form must reproduce the above copyright
notice, this list of conditions and the disclaimer (as noted below)
in the documentation and/or other materials provided with the
distribution.
@item
Neither the name of the UC/LLNL nor the names of its contributors
may be used to endorse or promote products derived from this software
without specific prior written permission.
@end itemize

THIS SOFTWARE IS PROVIDED BY THE COPYRIGHT HOLDERS AND CONTRIBUTORS
"AS IS" AND ANY EXPRESS OR IMPLIED WARRANTIES, INCLUDING, BUT NOT
LIMITED TO, THE IMPLIED WARRANTIES OF MERCHANTABILITY AND FITNESS FOR
A PARTICULAR PURPOSE ARE DISCLAIMED. IN NO EVENT SHALL THE REGENTS OF
THE UNIVERSITY OF CALIFORNIA, THE U.S. DEPARTMENT OF ENERGY OR
CONTRIBUTORS BE LIABLE FOR ANY DIRECT, INDIRECT, INCIDENTAL, SPECIAL,
EXEMPLARY, OR CONSEQUENTIAL DAMAGES (INCLUDING, BUT NOT LIMITED TO,
PROCUREMENT OF SUBSTITUTE GOODS OR SERVICES; LOSS OF USE, DATA, OR
PROFITS; OR BUSINESS INTERRUPTION) HOWEVER CAUSED AND ON ANY THEORY OF
LIABILITY, WHETHER IN CONTRACT, STRICT LIABILITY, OR TORT (INCLUDING
NEGLIGENCE OR OTHERWISE) ARISING IN ANY WAY OUT OF THE USE OF THIS
SOFTWARE, EVEN IF ADVISED OF THE POSSIBILITY OF SUCH DAMAGE.

@b{Additional BSD Notice}

@enumerate
@item
This notice is required to be provided under our contract with the
U.S. Department of Energy (DOE). This work was produced at the
University of California, Lawrence Livermore National Laboratory under
Contract No. W-7405-ENG-48 with the DOE.
@item
Neither the United States Government nor the University of
California nor any of their employees, makes any warranty, express or
implied, or assumes any liability or responsibility for the accuracy,
completeness, or usefulness of any information, apparatus, product, or
process disclosed, or represents that its use would not infringe
privately-owned rights.
@item
Also, reference herein to any specific commercial products,
process, or services by trade name, trademark, manufacturer or
otherwise does not necessarily constitute or imply its endorsement,
recommendation, or favoring by the United States Government or the
University of California. The views and opinions of authors expressed
herein do not necessarily state or reflect those of the United States
Government or the University of California, and shall not be used for
advertising or product endorsement purposes.
@end enumerate
@end copying

@comment -----------------------------------------------------------------

@titlepage
@title Yorick: An Interpreted Language
@author David H. Munro

@comment  The following two commands start the copyright page.
@page
@vskip 0pt plus 1filll
@insertcopying
@end titlepage

@contents

@comment -----------------------------------------------------------------

@ifnottex
@node Top
@top Yorick: An Interpreted Language

@comment @insertcopying
@end ifnottex

@menu
* Basics::                      A tutorial-like introductory chapter.
* Arrays::                      Using array syntax.
* Graphics::                    How to plot things.
* Embedding::                   Embedding compiled code in a custom Yorick.
@comment * Concept Index::

 --- The Detailed Node Listing ---

Basic Ideas

* Simple Statements::
* Flow control::                Defining functions, conditionals, and loops.
* Environment::                 How to use Yorick's interpreted environment.

Simple Statements

* Define variable::
* Call procedure::
* Print expression::

Flow Control Statements

* Define function::
* Define procedure::
* Conditionals::                Conditionally executing statements.
* Loops::                       Repeatedly executing statements.
* Scoping::                     Local and external variables.

Conditional Execution

* if-else constructs::
* logical operators::

Loops

* while and do::
* for::
* goto::                        How to break, continue, and goto from a loop body.

Variable scope

* extern::
* local::

The Interpreted Environment

* Starting::                    Starting, stopping, and interrupting Yorick.
* Include::                     How to read Yorick statements from a file.
* Help::                        Using the help command.
* Info::                        Getting information about a variable.
* Prompts::                     What Yorick prompts mean.
* Shell commands::              Issuing shell commands from within Yorick.
* Errors::                      What to do when Yorick detects an error.

Include files

* Sample include::
* Comments::
* DOCUMENT::                    The help command recognizes special comments.
* Include path::                Directories Yorick searches for include files.
* Customizing::                 How to execute Yorick statements at startup.

Error Messages

* Runtime errors::
* Simple debugging::

Using Array Syntax

* Creating Arrays::             How to originate arrays.
* Interpolating::               Interpolation functions.
* Indexing::                    How to reference array elements.
* Sorting::                     How to sort an array.
* Transposing::                 How to change the order of array dimensions.
* Broadcasting::                Making arrays conformable.
* Dimension Lists::

Indexing

* Scalar Index::                Scalar indices and array order.
* Index Range::                 Selecting a range of indices.
* Nil Index::                   Nil index refers to an entire dimension.
* Index List::                  Selecting an arbitrary list of indices.
* Pseudo-Index::                Creating a pseudo-index.
* Negative Index::              Numbering a dimension from its last element.
* Rubber-Index::                Using a rubber index.
* Matrix Multiply::             Marking an index for matrix multiplication.
* Statistical::                 Rank reducing (statistical) range functions.
* Finite Difference::           Rank preserving (finite difference) range functions.

Graphics

* Plotting primitives::         The basic drawing functions.
* Plot limits::                 Setting plot limits, log scaling, etc.
* Display list::                The display list model.
* Hardcopy::                    How to get it.
* Graphics style::              How to change it.
* Query and edit::              Seeing legends and making minor changes.
* pldefault::                   Setting (non-default) defaults.
* Custom plot functions::       Combining the plotting primitives.
* Animation::                   Spielberg look out.
* 3D graphics::                 An experimental interface.

Primitive plotting functions

* plg::                         Plot graph.
* pldj::                        Plot disjoint lines.
* plm::                         Plot quadrilateral mesh.
* plc::                         Plot contours.
* plf::                         Plot filled quadrilateral mesh.
* pli::                         Plot image.
* plfp::                        Plot filled polygons.
* plv::                         Plot vectors.
* plt::                         Plot text.

Plot limits and relatives

* limits::                      Set plot limits.
* logxy::                       Set log axis scaling.
* gridxy::                      Set grid lines.
* palette::                     Set color palette.
* Color model::                 More about color.

limits

* mouse zooming::               How to zoom by mouse clicks.
* saving limits::               Save and restore plot limits.
* square limits::               Assure that circles are not ellipses.

Managing a display list

* fma::                         Frame advance (begin next picture).
* multiple windows::            How to get them.

Getting hardcopy

* Color hardcopy::              Dumping palettes into hardcopy files.
* CGM hardcopy::                Caveats about binary CGM format.
* EPS hardcopy::                Encapsulated PostScript output.

Graphics style

* style keyword::               Accessing predefined graphics styles.
* style.i::                     Bypassing predefined graphics styles.
* plsys::                       Multiple coordinate systems.
* ticks and labels::            How to change them.

Queries, edits, and legends

* legends::                     Setting plot legends.
* plq and pledit::              The plot query and edit functions.

3D graphics interfaces

* 3D mapping::                  Changing your viewpoint.
* 3D lighting::                 The 3D lighting model.
* 3D gnomon::                   Gnomon indicates axis orientation.
* plwf::                        The plot wire frame interface.
* slice3::                      The slice and isosurface interface.

Embedding

* Mechanics of embedding::      How to use the make utility.
* API for embedding::           Using the yapi.h interface.
* Embedding example::           Three ways to implement a function.

Embedding example

* Using PROTOTYPE::             Example of PROTOTYPE comment.
* Using yapi.h::                Example of @file{yapi.h} API.
@end menu

@comment -----------------------------------------------------------------

@node Basics, Arrays, Top, Top
@chapter Basic Ideas

Yorick is an interpreted programming language.  With Yorick you can
read input (usually lists of numbers) from virtually any text or
binary file, process it, and write the results back to another file.
You can also plot results on your screen.

Yorick expression and flow control syntax is similar to the C
programming language, but the Yorick language lacks declaration
statements.  Also, Yorick array index syntax more closely resembles
Fortran than C.


@menu
* Simple Statements::
* Flow control::                Defining functions, conditionals, and loops.
* Environment::                 How to use Yorick's interpreted environment.
@end menu

@node Simple Statements, Flow control, Basics, Basics
@section Simple Statements

An interpreter immediately executes each line you type.  Most Yorick
input lines define a variable, invoke a procedure, or print the value
of an expression.

@menu
* Define variable::
* Call procedure::
* Print expression::
@end menu

@node Define variable, Call procedure, Simple Statements, Simple Statements
@subsection Defining a variable

The following five Yorick statements define the five variables @kbd{c},
@kbd{m}, @kbd{E}, @kbd{theta}, and @kbd{file}:

@example
c = 3.00e10;  m = 9.11e-28
E = m*c^2
theta = span(0, 6*pi, 200)
file = create("damped.txt")
@end example

Variable names are case sensitive, so @kbd{E} is not the same variable
as @kbd{e}.  A variable name must begin with a letter of the alphabet
(either upper or lower case) or with an underscore (@kbd{_});
subsequent characters may additionally include digits.

A semicolon terminates a Yorick statement, so the first line contains
two statements.  To make Yorick statements easier to type, you don't
need to put a semicolon at the end of most lines.  However, if you are
composing a Yorick program in a text file (as opposed to typing
directly to Yorick itself, a semicolon at the end of every line
reduces the chances of a misinterpretation, and makes your program
easier to read.

Conversely, a new line need not represent the end of a statement.  If
the line is incomplete, the statement automatically continues on the
following line.  Hence, the second and third lines above could have
been typed as:

@example
E=
  m *
  c^2
theta= span(0, 6*pi,
            200)
@end example

In the second line, @kbd{*} and @kbd{^} represent multiplication and
raising to a power.  The other common arithmetic operators are
@kbd{+}, @kbd{-}, @kbd{/} (division), and @kbd{%} (remainder or
modulo).  The rules for forming arithmetic expressions with these
operators and parentheses are the same in Yorick as in Fortran or C
(but note that @kbd{^} does not mean raise to a power in C, and
Fortran uses the symbol @kbd{**} for that operation).

The @kbd{span} function returns 200 equally spaced values beginning
with @kbd{0} and ending with @kbd{6*pi}.  The variable @kbd{pi} is
predefined as 3.14159...

The @kbd{create} function returns an object representing the new file.
The variable @kbd{file} specifies where output functions should write
their data.  Besides numbers like @kbd{c}, @kbd{m}, and @kbd{E}, or
arrays of numbers like @kbd{theta}, or files like @kbd{file}, Yorick
variables may represent several other sorts of objects, taken up in
later chapters.

The @kbd{=} operator is itself a binary operator, which has the side
effect of redefining its left operand.  It associates from right to
left, that is, the rightmost @kbd{=} operation is performed first (all
other binary operators except @kbd{^} associate from left to right).
Hence, several variables may be set to a single value with a single
statement:

@example
psi = phi = theta = span(0, 6*pi, 200)
@end example

When you define a variable, Yorick forgets any previous value and data
type:

@example
phi = create("junk.txt")
@end example


@node Call procedure, Print expression, Define variable, Simple Statements
@subsection Invoking a procedure

A Yorick function which has side effects may sensibly be invoked as a
procedure, discarding the value returned by the function, if any:

@example
plg, sin(theta)*exp(-theta/6), theta
write, file, theta, sin(theta)*exp(-theta/6)
close, file
@end example

The @kbd{plg} function plots a graph on your screen --- in this case,
three cycles of a damped sine wave.  The graph is made by connecting
200 closely spaced points by straight lines.

The @kbd{write} function writes a two column, 200 line table of values
of the same damped sine wave to the file @file{damped.txt}.  Then
@kbd{close} closes the file, making it unavailable for any further
@kbd{write} operations.

A line which ends with a comma will be continued, to allow procedures
with long argument lists.  For example, the @kbd{write} statement could
have been written:

@example
write, file, theta,
             sin(theta)*exp(-theta/6)
@end example

A procedure may be invoked with zero arguments; several graphics
functions are often used in this way:

@example
hcp
fma
@end example

The @kbd{hcp} function writes the current graphics window contents to
a ``hardcopy file'' for later retrieval or printing.  The @kbd{fma}
function stands for ``frame advance'' --- subsequent plotting commands
will draw on a fresh page.  Normally, plotting commands such as
@kbd{plg} draw on top of whatever has been drawn since the previous
@kbd{fma}.


@node Print expression,  , Call procedure, Simple Statements
@subsection Printing an expression

An unadorned expression is also a legal Yorick statement; Yorick prints
its value.  In the preceding examples, only the characters you would
type have been shown; to exhibit the @kbd{print} function and its
output, I need to show you what your screen would look like --- not only
what you type, but what Yorick prints.  To begin with, Yorick prompts
you for input with a @kbd{>} followed by a space (@pxref{Prompts}).  In
the examples in this section, therefore, the lines which begin with
@kbd{>} are what you typed; the other line(s) are Yorick's
responses.

@example
> E
8.199e-07
> print, E
8.199e-07
> m;c;m*c^2
9.11e-28
3.e+10
8.199e-07
> span(0,2,5)
[0,0.5,1,1.5,2]
> max(sin(theta)*exp(-theta/6))
0.780288
@end example

In the @kbd{span} example, notice that an array is printed as a comma
delimited list enclosed in square brackets.  This is also a legal
syntax for an array in a Yorick statement.  For numeric data, the
@kbd{print} function always displays its output in a format which
would be legal in a Yorick statement; you must use the @kbd{write}
function if you want ``prettier'' output.  Beware of typing an
expression or variable name which is a large array; it is easy to
generate lots of output (you can interrupt Yorick by typing control-c
if you do this accidentally, @pxref{Starting}).

Most non-numeric objects print some useful descriptive information;
for example, before it was closed, the @kbd{file} variable above would
have printed:

@example
> file
write-only text stream at:
  LINE: 201  FILE: /home/icf/munro/damped.txt
@end example

As you may have noticed, printing a variable and invoking a procedure
with no arguments are syntactically indistinguishable.  Yorick decides
which operation is appropriate at run time.  Thus, if @kbd{file} had
been a function, the previous input line would have invoked the
@kbd{file} function; as it was not a function, Yorick printed the
@kbd{file} variable.  Only an explicit use of the @kbd{print} function
will print a function:

@example
> print, fma
builtin fma()
@end example


@node Flow control, Environment, Simple Statements, Basics
@section Flow Control Statements

Almost everything you type at Yorick during an interactive session
will be one of the simple Yorick statements described in the previous
section -- defining variables, invoking procedures, and printing
expressions.  However, in order to actually use Yorick you need to
write your own functions and procedures.  You can do this by typing
Yorick statements at the keyboard, but you should usually put function
definitions in a text file.  Suggestions for how to organize such
``include'' files will be the topic of the next section.  This section
introduces the Yorick statements which define functions, conditionals,
and loops.


@menu
* Define function::
* Define procedure::
* Conditionals::                Conditionally executing statements.
* Loops::                       Repeatedly executing statements.
* Scoping::                     Local and external variables.
@end menu

@node Define function, Define procedure, Flow control, Flow control
@subsection Defining a function

Consider a damped sine wave.  It describes the time evolution of an
oscillator, such as a weight on a spring, which bobs up and down for a
while after you whack it.  The basic shape of the wave is determined
by the Q of the oscillator; a high Q means there is little friction,
and the weight will continue bobbing for many cycles, while a low Q
means a lot of friction and few cycles.  The amplitude of the
oscillation is therefore a function of two parameters -- the phase
(time in units of the natural period of the oscillator), and the Q:

@example
func damped_wave(phase, Q)
@{
  nu = 0.5/Q;
  omega = sqrt(1.-nu*nu);
  return sin(omega*phase)*exp(-nu*phase);
@}
@end example

Within a function body, I terminate every Yorick statement with a
semicolon (@pxref{Define variable}).

The variables @kbd{phase} and @kbd{Q} are called the parameters of the
function.  They and the variables @kbd{nu} and @kbd{omega} defined in
the first two lines of the function body are @emph{local} to the
function.  That is, calling @kbd{damped_wave} will not change the
values of any variables named @kbd{phase}, @kbd{Q}, @kbd{nu}, or
@kbd{omega} in the calling environment.

In fact, the only effect of calling @kbd{damped_wave} is to return its
result, which is accomplished by the @kbd{return} statement in the
third line of the body.  That is, calling @kbd{damped_wave} has no side
effects.  You can use @kbd{damped_wave} in expressions like this:

@example
> damped_wave(1.5, 3)
0.775523
> damped_wave([.5,1,1.5,2,2.5], 3)
[0.435436,0.705823,0.775523,0.659625,0.41276]
> q5 = damped_wave(theta,5)
> fma; plg, damped_wave(theta,3), theta
> plg, damped_wave(theta,1), theta
@end example

The last two lines graphically compare Q=3 oscillations to Q=1
oscillations.

Notice that the arguments to @kbd{damped_wave} may be arrays.  In this
case the result will be the array of results for each element of
input; hence @kbd{q5} will be an array of 200 numbers.  Nor is Yorick
confused by the fact that the @kbd{phase} argument (@kbd{theta}) is an
array, while the @kbd{Q} argument (@kbd{5}) is a scalar.  The precise
rules for ``conformability'' between two arrays will be described
later (@pxref{Broadcasting}); usually you get what you expected.

In this case, as Yorick evaluates @kbd{damped_wave}, @kbd{nu} and
@kbd{omega} will both be scalars, since @kbd{Q} is a scalar.  On the
other hand, @kbd{omega*phase} and @kbd{nu*phase} become arrays, since
@kbd{phase} is an array.  Whenever an operand is an array, an
arithmetic operation produces an array as its result.


@node Define procedure, Conditionals, Define function, Flow control
@subsection Defining Procedures

Any Yorick function may be invoked as a procedure; the return value is
simply discarded.  Calling @kbd{damped_wave} as a procedure would be
pointless, since it has no side effects.  The converse case is a
function to be called @emph{only} for its side effects.  Such a
function need not have an explicit @kbd{return} statement:

@example
func q_out(Q, file_name)
@{
  file = create(file_name);
  write, file, "Q = "+pr1(Q);
  write, file, "   theta       amplitude";
  write, file, theta, damped_wave(theta,Q);
@}
@end example

The @kbd{pr1} function (for ``print one value'') returns a string
representation of its numeric argument, and the @kbd{+} operator
between two strings concatenates them.

You would invoke @kbd{q_out} with the line:

@example
q_out, 3, "q.out"
@end example

Besides lacking an explicit @kbd{return} statement, the @kbd{q_out}
function has two other peculiarities worth mentioning:

First, the variable @kbd{theta} is never defined.  But neither are the
functions @kbd{create} and @kbd{write}.  Any symbol not defined within
a function is @emph{external} to the function.  As already noted,
parameters and variables defined within the function are @emph{local}
to the function.  Here, @kbd{theta} is the 200 element array of phase
angles defined before q_out was called.  You can change theta (both
its dimensions and its values) before the next call to @kbd{q_out};
that is, an external reference may change between calls to a function
that uses it.

Second, @kbd{file} is never explicitly closed.  When a function
returns, its local variables (such as @kbd{file}) disappear; when a
file object disappears, the associated file automatically closes.


@node Conditionals, Loops, Define procedure, Flow control
@subsection Conditional Execution

The design of the @kbd{q_out} function can be improved.  As written,
each output file will contain only a single wave.  You might want the
option of writing several waves into a single file.  Consider this
alternative:

@example
func q_out(Q, file)
@{
  if (structof(file)==string) file = create(file);
  write, file, "Q = "+pr1(Q);
  write, file, "   theta       amplitude";
  write, file, theta, damped_wave(theta,Q);
  return file;
@}
@end example

The @kbd{if} statement executes its body (the redefinition of
@kbd{file}) if and only if its condition
(@kbd{structof(file)==string}) is true.  Any scalar number may serve
as a condition -- a non-zero value is ``true'', and the value zero is
``false''.

The @kbd{structof()} function returns the data type object when its
argument is an array, and nil (@kbd{[]}) otherwise.  In particular, if
@kbd{file} is a text string (like @kbd{"q.out"}), @kbd{structof}
returns string, which is the data type of strings.  The binary
operator @kbd{==} (as distinguished from assignment @kbd{=}) tests for
equality, returning ``true'' or ``false''.  Note that the test works
even for non-numeric objects, like the data type string.

Hence, if the @kbd{file} argument is a string, the new @kbd{q_out}
presumes it is the name of a file, which it creates, redefining
@kbd{file} as the associated file object.  Thus, after the first line
of the function body, @kbd{file} will be a file object, even if a file
name was passed into the function.  Furthermore, since the parameter
@kbd{file} is local to @kbd{q_out}, none of this hocus pocus will have
any effect outside @kbd{q_out}.

The second trick in the new @kbd{q_out} is the reappearance of a
@kbd{return} statement.  The original calling sequence:
@example
q_out, 3, "q.out"
@end example
@noindent
has the same result as before -- the @kbd{if} condition is true, so
the file is created, then the wave data is written.  This time the
file object is returned, only to be discarded because @kbd{q_out}
was invoked as a procedure.  When the file object disappears, the
file closes.  But if @kbd{q_out} were invoked as a function, the file
object can be saved, which keeps the file open:

@example
f = q_out(3,"q.out")
q_out,2,f
q_out,1,f
close,f
@end example

Now the file @file{q.out} contains the Q=3 wave, followed by the Q=2
and Q=1 waves.  In the second and third calls to @kbd{q_out}, the
@kbd{file} parameter is already a file object, so the @kbd{if}
condition is false, and @kbd{create} is not called.  Notice that the
file does not close when the return value from the second (or third)
call is discarded; the variable @kbd{f} refers to the same file object
as the discarded return value.  Without an explicit call to
@kbd{close}, a file only closes when the @emph{final} reference to it
disappears.


@menu
* if-else constructs::
* logical operators::
@end menu

@node if-else constructs, logical operators, Conditionals, Conditionals
@subsubsection General @kbd{if} and @kbd{else} constructs

The most general form of the @kbd{if} statement is:

@example
if (condition) statement_if_true;
else statement_if_false;
@end example

When you need to choose among several alternatives, make the @kbd{else}
clause itself an @kbd{if} statement:

@example
if (condition_1) statement_if_1;
else if (condition_2) statement_if_2;
else if (condition_3) statement_if_3;
...
else statement_if_none;
@end example

The final @kbd{else} is always optional -- if it is not present,
nothing at all happens if none of the conditions is satisfied.

Often you need to execute more than one statement conditionally.
Quite generally in Yorick, you may group several statements into a
single compound statement by means of curly braces:

@example
@{ statement_1; statement_2; statement_3; statement_4; @}
@end example

Ordinarily, both curly braces and each of the statements should be
written on a separate line, with the statements indented to make their
membership in the compound more obvious.  Further, in an if-else
construction, when one branch requires more than one statement, you
should write every branch as a compound statement with curly braces.
Therefore, for a four branch @kbd{if} in which the second @kbd{if}
requires two statements, and the @kbd{else} three, write this:

@example
if (condition_1) @{
  statement_if_1;
@} else if (condition_2) @{
  statement_if_2_A;
  statement_if_2_B;
@} else if (condition_3) @{
  statement_if_3;
@} else @{
  statement_if_none_A;
  statement_if_none_B;
  statement_if_none_C;
@}
@end example

The @kbd{else} statement has a very unusual syntax: It is the only
statement in Yorick which depends on the form of the @emph{previous}
statement; an @kbd{else} must follow an @kbd{if}.  Because Yorick
statements outside functions (and compound statements) are executed
immediately, you must be very careful using @kbd{else} in such a
situation.  If the @kbd{if} has already been executed (as it would be
in the examples without curly braces), the following @kbd{else} leads
to a syntax error!

The best way to avoid this puzzling problem is to always use the curly
brace syntax of the latest example when you use @kbd{else} outside a
function.  (You don't often need it in such a context anyway.)


@node logical operators,  , if-else constructs, Conditionals
@subsubsection Combining conditions with @kbd{&&} and @kbd{||}

The logical operators @kbd{&&} (``and'') and @kbd{||} (``or'') combine
conditional expressions; the @kbd{!} (``not'') operator negates them.
The @kbd{!} has the highest precedence, then @kbd{&&}, then @kbd{||};
you will need parentheses to force a different order.

(Beware of the bitwise ``and'' and ``or'' operators @kbd{&} and
@kbd{|} -- these should @emph{never} be used to combine conditions;
they are for set-and-mask bit fiddling.)

The operators for comparing numeric values are @kbd{==} (equal),
@kbd{!=} (not equal), @kbd{>} (greater), @kbd{<} (less), @kbd{>=}
(greater or equal), and @kbd{<=} (less or equal).  These all have
higher precedence than @kbd{&&}, but lower than @kbd{!} or any
arithmetic operators.

@example
if ((a<b && x>a && x<b) || (a>b && x<a && x>b))
  write, "x is between a and b"
@end example

Here, the expression for the right operand to @kbd{&&} will execute
only if its left operand is actually true.  Similarly, the right
operand to @kbd{||} executes only if its left proves false.
Therefore, it is important to order the operands of @kbd{&&} and
@kbd{||} to put the most computationally expensive expression on the
right --- even though the logical ``and'' and ``or'' functions are
commutative, the order of the operands to @kbd{&&} and @kbd{||} can be
critical.

In the example, if @kbd{a>b}, the @kbd{x>a} and @kbd{x<b}
subexpressions will not actually execute since @kbd{a<b} proved false.
Since the left operand to @kbd{||} was false, its right operand will
be evaluated.

Despite the cleverness of the @kbd{&&} and @kbd{||} operators in not
executing the expression for their right operands unless absolutely
necessary, the example has obvious inefficiencies: First, if
@kbd{a>=b}, then both @kbd{a<b} and @kbd{a>b} are checked.  Second, if
@kbd{a<b}, but @kbd{x} is not between @kbd{a} and @kbd{b}, the right
operand to @kbd{||} is evaluated anyway.  Yorick has a ternary operator
to avoid this type of inefficiency:

@example
expr_A_or_B= (condition? expr_A_if_true : expr_B_if_false);
@end example

The @kbd{?:} operator evaluates the middle expression if the condition
is true, the right expression otherwise.  Is that so? Yes : No.  The
efficient betweeness test reads:

@example
if (a<b? (x>a && x<b) : (x<a && x>b))
  write, "x is between a and b";
@end example


@node Loops, Scoping, Conditionals, Flow control
@subsection Loops

Most loops in Yorick programs are implicit; remember that operations
between array arguments produce array results.  Whenever you write a
Yorick program, you should be suspicious of @emph{all} explicit loops.
Always ask yourself whether a clever use of array syntax could have
avoided the loop.

To illustrate an appropriate Yorick loop, let's revise @kbd{q_out} to
write several values of Q in a single call.  Incidentally, this sort
of incremental revision of a function is very common in Yorick program
development.  As you use a function, you notice that the surrounding
code is often the same, suggesting a savings if it were incorporated
into the function.  Again, a careful job leaves all of the previous
behavior of @kbd{q_out} intact:

@example
func q_out(Q, file)
@{
  if (structof(file)==string) file = create(file);
  n = numberof(Q);
  for (i=1 ; i<=n ; ++i) @{
    write, file, "Q = "+pr1(Q(i));
    write, file, "   theta       amplitude";
    write, file, theta, damped_wave(theta,Q(i));
  @}
  return file;
@}
@end example

Two new features here are the @kbd{numberof} function, which returns
the length of the array @kbd{Q}, and the array indexing syntax
@kbd{Q(i)}, which extracts the @kbd{i}-th element of the array
@kbd{Q}.  If @kbd{Q} is scalar (as it had to be before this latest
revision), @kbd{numberof} returns 1, and @kbd{Q(1)} is the same as
@kbd{Q}, so scalar @kbd{Q} works as before.

But the most important new feature is the @kbd{for} loop.  It says to
initialize @kbd{i} to 1, then, as long as @kbd{i} remains less than or
equal to @kbd{n}, to execute the loop body, which is a compound of
three write statements.  Finally, after each pass, the increment
expression @kbd{++i} is executed before the @kbd{i<=n} condition is
evaluated.  @kbd{++i} is equivalent to @kbd{i=i+1}; @kbd{--i} means
@kbd{i=i-1}.

A loop body may also be a single statement, in which case the curly
braces are unecessary.  For example, the following lines will write
the Q=3, Q=2, and Q=1 curves to the file @file{q.out}, then plot
them:

@example
q_list = [3,2,1]
q_out, q_list, "q.out"
fma; for(i=1;i<=3;++i) plg, damped_wave(theta,q_list(i)), theta
@end example

A @kbd{for} loop with a @kbd{plg} body is the easiest way to overlay a
series of curves.  After the three simple statement types, @kbd{for}
statements see the most frequent direct use from the keyboard.


@menu
* while and do::
* for::
* goto::                        How to break, continue, and goto from a loop body.
@end menu

@node while and do, for, Loops, Loops
@subsubsection The @kbd{while} and @kbd{do while} statements

The @kbd{while} loop is simpler than the @kbd{for} loop:

@example
while (condition) body_statement
@end example

The @kbd{body_statement} --- very oten a compound statement enclosed
in curly braces --- executes over and over until the @kbd{condition}
becomes false.  If the @kbd{condition} is false to begin with,
@kbd{body_statement} never executes.

Occasionally, you want the @kbd{body_statement} to execute and set the
@kbd{condition} @emph{before} it is tested for the first time.  In
this case, use the @kbd{do while} statement:

@example
do @{
  body_statement_A;
  body_statement_B;
  ...etc...
@} while (condition);
@end example


@node for, goto, while and do, Loops
@subsubsection The @kbd{for} statement

The @kbd{for} statement is a ``packaged'' form of a @kbd{while} loop.
The meaning of this generic @kbd{for} statement is the same as the
following @kbd{while}:

@example
for (start_statements ; condition ; step_statements) body_statements

start_statments
while (condition) @{
  body_statements
  step_statements
@}
@end example

There only two reasons to prefer @kbd{for} over @kbd{while}: Most
importantly, @kbd{for} can show ``up front'' how a loop index is
initialized and incremented, rather than relegating the increment
operation (@kbd{step_statements}) to the end of the loop.  Secondly,
the @kbd{continue} statement (@pxref{goto}) branches just past the body of
the loop, which would include the @kbd{step_statements} in a
@kbd{while} loop, but not in a @kbd{for} loop.

In order to make Yorick loop syntax agree with C, Yorick's @kbd{for}
statement has a syntactic irregularity: If the @kbd{start_statements}
and @kbd{step_statements} consist of more than one statement, then
commas (not semicolons) separate the statements.  (The C comma
operator is not available in any other context in Yorick.)  The two
semicolons separate the start, condition, and step clauses and must
always be present, but any or all of the three clauses themselves may
be blank.  For example, in order to increment two variables @kbd{i}
and @kbd{j}, a loop might look like this:

@example
for (i=1000, j=1 ; i>j ; i+=1000, j*=2);
@end example

This example also illustrates that the @kbd{body_statements} may be
omitted; the point of this loop is merely to compute the first @kbd{i}
and @kbd{j} for which the condition is not satisfied.  The trailing
semicolon is necessary in this case, since otherwise the line would
be continued (on the assumption that the loop body was to follow).


The @kbd{+=} and @kbd{*=} are special forms of the @kbd{=} operator;
@kbd{i=i+1000} and @kbd{j=j*2} mean the same thing.  Any binary
operator may be used in this short-hand form in order to increment a
variable.  Like @kbd{++i} and @kbd{--i}, these are particularly useful
in loop increment expressions.


@node goto,  , for, Loops
@subsubsection Using @kbd{break}, @kbd{continue}, and @kbd{goto}

The @kbd{break} statement jumps out of the loop containing it.  The
@kbd{continue} statement jumps to the end of the loop body,
``continuing'' with the next pass through the loop (if any).  In
deeply nested loops --- which are extremely rare in Yorick programs
--- you can jump to more general places using the @kbd{goto}
statement.  For example, here is how to break out or continue from one
or both levels of a doubly nested loop:

@example
 for (i=1 ; i<=n ; ++i) @{
   for (j=1 ; j<=m ; ++j) @{
     if (skip_to_next_j_now) continue;
     if (skip_to_next_i_now) goto skip_i;
     if (break_out_of_inner_loop_now) break;
     if (break_out_of_both_loops_now) goto done;
     more_statements_A;
   @}
   more_statements_B;
  skip_i:
 @}
done:
 more_statements_C;
@end example

@noindent
The @kbd{continue} jumps just after @kbd{more_statements_A}, the
@kbd{break} just before @kbd{more_statements_B}.

Break and continue statements may be used to escape from @kbd{while}
or @kbd{do while} loops as well.

If @kbd{while} tests the condition before the loop body, and @kbd{do
while} checks after the loop body, you may have wondered what to do
when you need to check the condition in the middle of the loop body.
The answer is the ``do forever'' construction, plus the @kbd{break}
statement (note the inversion of the sense of the condition):

@example
for (;;) @{
  body_before_test;
  if (!condition) break;
  body_after_test;
@}
@end example


@node Scoping,  , Loops, Flow control
@subsection Variable scope

By default, dummy parameters and variables which are defined in a
function body before any other use are @emph{local} variables.
Variables (or functions) used before their definition are
@emph{external} variables (@pxref{Define procedure}, @pxref{Define
function}).  A variable (or function) defined outside of all functions
is just that --- external to all functions and local to none.

Whenever a function is called, Yorick remembers the external values of
all its local variables, then replaces them by their local values.
Thus, all of its local variables are potentially ``visible'' as external
variables to any function it calls.  When it returns, the function
replaces all its local variables by the values it remembered.  Neither
that function, nor any function it calls can affect these remembered
values; Yorick provides no means of ``unmasking'' a local variable.

The default rule for determining whether a variable should have local or
external scope fails in two cases: First, you may want to redefine an
external variable without looking at its value.  Second, a few
procedures set the values of their parameters when they return; it may
appear to Yorick's parser that such a variable has been used without
being defined, even though you intend it to be local to the function.
The @kbd{extern} and @kbd{local} statements solve these two problems,
respectively.

@menu
* extern::
* local::
@end menu

@node extern, local, Scoping, Scoping
@subsubsection @kbd{extern} statements

Suppose you want to write a function which sets the value of the
@kbd{theta} array used by all the variants of the @kbd{q_out}
function:

@example
func q_out_domain(start, stop, n)
@{
  extern theta;
  theta = span(start, stop, n);
@}
@end example

Without the @kbd{extern} statement, @kbd{theta} will be a local
variable, whose external value is restored when @kbd{q_out_domain}
returns (thus, the function would be a no-op).  With the @kbd{extern}
statement, @kbd{q_out_domain} has the intended side effect of setting
the value of @kbd{theta}.  The new @kbd{theta} would be used by
subsequent calls to @kbd{q_out}.

Any number of variables may appear in a single @kbd{extern} statement:

@example
extern var1, var2, var3, var4;
@end example


@node local,  , extern, Scoping
@subsubsection @kbd{local} statements

The @kbd{save} and @kbd{restore} functions store and retrieve Yorick
variables in self-descriptive binary files (called PDB files).  To
create a binary file @file{demo.pdb}, save the variables @kbd{theta},
@kbd{E}, @kbd{m}, and @kbd{c} in it, then close the file:

@example
f = createb("demo.pdb")
save, f, theta, E, m, c
close, f
@end example

To open this file and read back the @kbd{theta} variable:

@example
restore, openb("demo.pdb"), theta, c
@end example

For symmetry with @kbd{save}, the @kbd{restore} function has the unusual
property that it redefines its parameters (except the first).  Thus,
@kbd{theta} is redefined by this @kbd{restore} statement, just as it
would have been by a @kbd{theta=} statement.  The Yorick parser does not
understand this, which means that you must be careful when you place
such a function call inside a function:

@example
func q_out_file(infile, Q, outfile)
@{
  if (!is_stream(infile)) infile = openb(infile);
  local theta;
  restore, infile, theta;
  return q_out(Q, outfile);
@}
@end example

This variant of @kbd{q_out} uses the @kbd{theta} variable read from
@kbd{infile}, instead of an external value of @kbd{theta}.  Without the
@kbd{local} declaration, @kbd{restore} would clobber the external value
of @kbd{theta}.  This way, when @kbd{q_out_file} calls @kbd{q_out}, it
has remembered the external value of @kbd{theta}, but @kbd{q_out}
``sees'' the value of @kbd{theta} restored from @kbd{infile}.  When
@kbd{q_out_file} finally returns, it replaces the original value of
@kbd{theta} intact.

Any number of variables may appear in a single @kbd{local} statement:

@example
local var1, var2, var3, var4;
@end example


@node Environment,  , Flow control, Basics
@section The Interpreted Environment

The Yorick program accepts only complete input lines typed at your
keyboard.  Typing a command to Yorick presumes a ``command line
interface'' or a ``terminal emulator'' which is not a part of Yorick.
I designed Yorick on the assumption that you have a good terminal
emulator program.  In particular, Yorick is much easier to use if you
can recall and edit previous input lines; as in music, repitition with
variations is at the heart of programming.  My personal recommendation
is the yorick command in GNU Emacs (defined in yorick.el).

Therefore, Yorick inherits most of its ``look and feel'' from your
terminal emulator.  Yorick's distinctive prompts and error messages
are described later in this section.

Any significant Yorick program will be stored in a text file, called
an @dfn{include} file, after the command which reads it.  Use your
favorite text editor to create and modify your include files.  Again,
GNU Emacs is my favorite --- use the yorick-mode (defined in
yorick.el) to edit Yorick include files as well as C programs.  Just
as C source file names should end in @file{.c} or @file{.h}, and
Fortran source file names should end in @file{.f}, so Yorick include
file names should end in @file{.i}.

This section begins with additional stylistic suggestions concerning
include files.  In particular, Yorick's @kbd{help} command can find
documentation comments in your include files if you format them
properly.  All of the built-in Yorick functions, such as @kbd{sin},
@kbd{write}, or @kbd{plg}, come equipped with such comments.


@menu
* Starting::                    Starting, stopping, and interrupting Yorick.
* Include::                     How to read Yorick statements from a file.
* Help::                        Using the help command.
* Info::                        Getting information about a variable.
* Prompts::                     What Yorick prompts mean.
* Shell commands::              Issuing shell commands from within Yorick.
* Errors::                      What to do when Yorick detects an error.
@end menu

@node Starting, Include, Environment, Environment
@subsection Starting, stopping, and interrupting Yorick

Start Yorick by typing its name to your terminal emulator.  When you
are done, use the @kbd{quit} function to exit gracefully:

@example
% yorick
 Yorick ready.  For help type 'help'.
> quit
%
@end example

You can interrupt Yorick at any time by typing Control-C; this causes
an immediate runtime error (@pxref{Errors}).  (Your terminal emulator
and operating system must be able to send Yorick a SIGINT signal in
order for this to work.  On UNIX systems, you can set this character
using the @kbd{intr} option of the @kbd{stty} command; Control-C is the
usual setting.)


@node Include, Help, Starting, Environment
@subsection Include files

When you run Yorick in a window or Emacs environment, keep a window
containing the include file you are writing, in addition to the window
where Yorick is running.  When you want to test the latest changes to
your include file --- let's call it @file{damped.i} --- save the file,
then move to your Yorick window and type:

@example
#include "damped.i"
@end example

The special character @kbd{#} reminds you that the include directive
is not a Yorick statement, like all other inputs to Yorick.  Instead,
@kbd{#include} directs Yorick to switch its input stream from the
keyboard to the quoted file.  Yorick then treats each line of the file
exactly as if it were a line you had typed at the keyboard (including,
of course, additional @kbd{#include} lines).  When there are no more
lines in the file, the input stream switches back to the keyboard.

While you are debugging an include file, you will ordinarily include
it over and over again, until you get it right.  This does no harm,
since any functions or variables defined in the file will be replaced
by their new definitions the next time you include the file, just as
they would if you typed new definitions.

Ideally, your include files should consist of a series of function
definitions and variable initializations.  If you follow this
discipline, you can regard each include file as a library or package of
functions.  You type @kbd{#include} once in order to load this
library, making the functions defined there available, just like
Yorick's built in functions.  Many such libraries come with Yorick ---
Bessel functions, cubic spline interpolators, and other goodies.
Most of these functions are available without the need to explicitly
@kbd{#include} the required package, via yorick's @kbd{autoload}
function (@pxref{Customizing}).

@menu
* Sample include::
* Comments::
* DOCUMENT::                    The help command recognizes special comments.
* Include path::                Directories Yorick searches for include files.
* Customizing::                 How to execute Yorick statements at startup.
@end menu

@node Sample include, Comments, Include, Include
@subsubsection A sample include file

Here is @file{damped.i}, which defines the damped sine wave functions
we've been designing in this chapter:

@example
/* damped.i */

local damped;
/* DOCUMENT damped.i --- compute and output damped sine waves
   SEE ALSO: damped_wave, q_out
 */

func damped_wave(phase, Q)
/* DOCUMENT damped_wave(phase, Q)
     returns a damped sine wave evaluated at PHASE, for quality factor Q.
     (High Q means little damping.)  The PHASE is 2*pi after one period of
     the natural frequency of the oscillator.  PHASE and Q must be
     conformable arrays.

   SEE ALSO: q_out
 */
@{
  nu = 0.5/Q;
  omega = sqrt(1.-nu*nu);
  return sin(omega*phase)*exp(-nu*phase);  /* always zero at phase==0 */
@}

func q_out(Q, file)
/* DOCUMENT q_out, Q, file
         or q_out(Q, file)
     Write the damped sine wave of quality factor Q to the FILE.
     FILE may be either a filename, to create the file, or a file
     object returned by an earlier create or q_out operation.  If
     q_out is invoked as a function, it returns the file object.

     The external variable
          theta
     determines the phase points at which the damped sine wave is
     evaluated; q_out will write two header lines, followed by
     two columns, with one line for each element of the theta
     array.  The first column is theta; the second is
     damped_wave(theta, Q).

     If Q is an array, the two header lines and two columns will
     be repeated for each element of Q.

   SEE ALSO: damped_wave
 */
@{
  if (structof(file)==string) file = create(file);
  n = numberof(Q);
  for (i=1 ; i<=n ; ++i) @{ /* loop on elements of Q */
    write, file, "Q = "+pr1(Q(i));
    write, file, "   theta       amplitude";
    write, file, theta, damped_wave(theta,Q(i));
  @}
  return file;
@}
@end example

You would type

@example
#include "damped.i"
@end example

@noindent
to read the Yorick statements in the file.  The first thing you notice
is that there are comments --- the text enclosed by @kbd{/* */}.  If
you take the trouble to save a program in an include file, you will be
wise to make notes about what it's for and annotations of obscure
statements which might confuse you later.  Writing good comments is
arguably the most difficult skill in programming.  Practice it
diligently.


@node Comments, DOCUMENT, Sample include, Include
@subsubsection Comments

There are two ways to insert comments into a Yorick program.  These
are the C style @kbd{/* ... */} and the C++ style @kbd{//}:

@example
// C++ style comments begin with // (anywhere on a line)
// and end at the end of that line.

E = m*c^2;  /* C style comments begin with slash-star, and
               do not end until start-slash, even if that
               is several lines later.  */
/* C style comments need not annotate a single line.
 * You should pick a comment style which makes your
 * code attractive and easy to read.  */

F = m*a;         // Here is another C++ style comment...
divE = 4*pi*rho; /* ... and a final C style comment. */
@end example

I strongly recommend C++ style comments when you ``comment
out'' a sequence of Yorick statements.  C style comments do not nest
properly, so you can't comment out a series of lines which contain
comments:

@example
/*
E = m*c^2;   /* ERROR -- this ends the outer comment --> */
F = m*a
*/ <-- then this causes a syntax error
@end example

The C++ style not only works correctly; it also makes it more obvious
that the lines in question are comments:

@example
// E = m*c^2;   /* Any kind of comment could go here.  */
// F = m*a;
@end example

Yorick recognizes one special comment: If the first line of an include
file begins with @kbd{#!}, Yorick ignores that line.  This allows Yorick
include scripts to be executable on UNIX systems supporting the ``pound
bang'' convention:

@example
#!/usr/local/bin/yorick -batch
/* If this file has execute permission, UNIX will use Yorick to
 * execute it.  The Yorick function get_argv can be used to accept
 * command line arguments (see help, get_argv).  You might want
 * to use -i instead of -batch on the first line.  Read the help
 * on process_argv and batch for more information.  */
write, "The square root of pi is", sqrt(pi);
quit;
@end example


@node DOCUMENT, Include path, Comments, Include
@subsubsection @kbd{DOCUMENT} comments

Immediately after (within a few lines of) a @kbd{func}, @kbd{extern}, or
@kbd{local} statement (@pxref{Scoping}), you may put a comment which
begins with the eleven characters @kbd{/* DOCUMENT}.  Although Yorick
itself doesn't pay any more attention to a @kbd{DOCUMENT} comment than
to any other comment, there is a function called @kbd{help} which does.
What Yorick does do when it sees a @kbd{func}, @kbd{extern}, or
@kbd{local} (outside of any function body), is to record the include
file name and line number where that function or variable(s) are
defined.

Later, when you ask for help on some topic, the @kbd{help} function
asks Yorick whether it knows the file and line number where that topic
(a function or variable) was defined.  If so, it opens the file, goes
to the line number, and scans forward a few lines looking for a
@kbd{DOCUMENT} comment.  If it finds one, it prints the entire
comment.

The file @kbd{damped.i} has three @kbd{DOCUMENT} comments: one for a
fictitious variable called @kbd{damped}, so that someone who saw the
file but didn't know the names of the functions inside could find out,
and one each for the @kbd{damped_wave} and @kbd{q_out} functions.


Near the end of most @kbd{DOCUMENT} comments, you will find a
@kbd{SEE ALSO:} field which feeds the reader the names of related
functions or variables which also have @kbd{DOCUMENT} comments.  If
you use these wisely, you can lead someone (often an older self) to
all of the documentation she needs to be able to use your package.

This low-tech form of online documentation is surprisingly effective:
easy to create, maintain, and use.  As an automated step toward a more
formal document, the @kbd{mkdoc} function (in the Yorick library include
file @file{mkdoc.i}) collects all of the @kbd{DOCUMENT} comments in one
or more include files, alphabetizes them, adds a table of contents, and
writes them into a file suitable for printing.


@node Include path, Customizing, DOCUMENT, Include
@subsubsection Where Yorick looks for include files

You can specify a complete path name (including directories) for the
file in an @kbd{#include} directive.  More usually, you will use a
relative path name.  In that case, Yorick tries to find the file
relative to these four directories, in this order:

@enumerate
@item
Your current working directory.
@item
@file{~/yorick}, that is, the @file{yorick} subdirectory of your home
directory.
@item
@file{Y_SITE/i}, where @file{Y_SITE} is the directory where
Yorick was installed at your site (@kbd{help} will tell you where
this is).
@item
@file{Y_SITE/contrib}
@end enumerate

You can use the @kbd{set_path} command in your @file{custom.i} file
in order to change this path, but you should be very cautious if you
do this.

The @file{~/yorick} directory is where you put all of the Yorick
include files you frequently use, which have not been placed in the
include or contrib directories at your site.  You can also override an
include file in one of these places by placing a file of the same name
in your @file{~/yorick} directory.


@node Customizing,  , Include path, Include
@subsubsection The @file{custom.i} file and @file{i-start/} directory

When Yorick starts, the last thing it does before prompting you is to
include the file @file{custom.i}.  The default @file{custom.i} is in the
@file{Y_SITE/i/custom.i}.  (Yorick's @kbd{help} command will tell
you the actual name of the @file{Y_SITE} directory at your site.)

If you create the directory @file{~/yorick}, you can place your own
version of @kbd{custom.i} there to override the default.  Always begin
by copying the default file to your directory.  Then add your
customizations to the bottom of the default file.  Generally, these
customizations consist of @kbd{#include} directives for function libraries
you use heavily, plus commands like @kbd{pldefault} to set up your
plotting style preferences.

Immediately before @file{custom.i}, yorick includes all files with the
@file{.i} extension which are present in the @file{Y_SITE/i-start},
@file{Y_HOME/i-start}, and @file{~/yorick/i-start} directories.  These
files are intended to contain calls to the interpreted @kbd{autoload}
function, which tells yorick what file to include in order to get the
full definition of a function in some package not loaded by default.
Using the @file{~/yorick/i-start}, you can autoload your own private
functions, eliminating most of the reason for a @kbd{custom.i} file.
See @file{Y_SITE/i-start} for examples of how to do this.


@node Help, Info, Include, Environment
@subsection The @kbd{help} function

Use the @kbd{help} function to get online help.  Yorick will parrot
the associated @kbd{DOCUMENT} comment to your terminal:

@example
> #include "damped.i"
> help, damped
  DOCUMENT damped.i --- compute and output damped sine waves
  SEE ALSO: damped_wave, q_out
defined at:   LINE: 3  FILE: /home/icf/munro/damped.i
>
@end example

Every Yorick function (including @kbd{help}!) has a @kbd{DOCUMENT}
comment which describes what it does.  Most have @kbd{SEE ALSO:}
references to related help topics, so you can navigate your way
through a series of related topics.

Whenever you want more details about a Yorick function, the first
thing to do is to see what @kbd{help} says about it.  Note that, in
addition to the @kbd{DOCUMENT} comment, @kbd{help} also reports the
full file name and line number where the function is defined.  That
way, if the comment doesn't tell you what you need to know, you can
go to the file and read the complete definition of the function.

The help for @kbd{help} itself, which is the default topic, is of
particular interest, even to a Yorick expert: it tells you the
directory where you can find the include files for all of Yorick's
library functions.  Just type @kbd{help}, like it says when Yorick
starts.  One of the things you will find in Yorick's directory tree is
a @file{doc} directory, which contains not only the source for this
manual, but also alphabetized listings of all @kbd{DOCUMENT} comments.
Read the @kbd{README} file in the @file{doc} directory.


@node Info, Prompts, Help, Environment
@subsection The @kbd{info} function

If you want to know the data type and dimensions of a variable, use
the @kbd{info} function.  Unlike @kbd{help}, which is intended to tell
you what a thing @emph{means}, @kbd{info} simply tells what a thing
@emph{is}.

@example
> info, theta
 array(double,200)
> info, E
 array(double)
@end example

Here, @kbd{double} means ``double precision floating point number'',
which is the default data type for any real number.  The default
integer data type is called @kbd{long}.  (Both these names come from
the C language, which gives them a precise meaning.)

Notice that the scalar value @kbd{E} is, somewhat confusingly, called
an ``array''.  In fact, a scalar is a special case of an array with
zero dimensions.  The @kbd{info} function is designed to print a Yorick
expression which will create a variable of the same data type and
shape as its argument.  Thus, @kbd{array(double,200)} in an expression
would evaluate to an array of 200 real numbers, while
@kbd{array(double)} is a real scalar (the values are always
zero).

Using @kbd{info} on a non-numeric quantity (a file object, a function,
etc.) results in the same output as the @kbd{print} function.  If an
array of numbers might be large, try @kbd{info} before
@kbd{print}.


@node Prompts, Shell commands, Info, Environment
@subsection Prompts

Yorick occasionally prompts you with something other than @kbd{>}.
The usual @kbd{>} prompt tells you that Yorick is waiting for a new
input line.  The other prompts alert you to unusual situations:

@table @kbd
@item cont>
The previous input line was not complete; Yorick is waiting for its
continuation before it does anything.  Type several close brackets or
control-C if you don't want to complete the line.
@item dbug>
When an error occurs during the execution of a Yorick program, Yorick
offers you the choice of entering ``debug mode''.  In this mode, you
are ``inside'' the function where the error occurred.  You can type
any Yorick statement you wish, but until you explicitly exit from
debug mode by means of the @kbd{dbexit} function, Yorick will prompt
you with @kbd{dbug>} instead of its usual prompt.  (@pxref{Simple debugging})
@item quot>
It is possible to continue a long string constant across multiple lines.
If you do, this will be your prompt until you type the closing @kbd{"}.
Don't ever do this intentionally; use the string concatenation operator
@kbd{+} instead, and break long strings into pieces that will fit on
a single line.
@item comm>
For some reason, you began a comment on the previous line, and Yorick
is waiting for the @kbd{*/} to finish the comment.  I can't imagine why
you would put comments in what you were typing at the terminal.
@end table


@node Shell commands, Errors, Prompts, Environment
@subsection Shell commands, removing and renaming files

Yorick has a @kbd{system} function which you can use to invoke
operating system utilities.  For example, typing @kbd{ls *.txt} to a
UNIX shell will list all files in your current working directory
ending with @file{.txt}.  Similarly, @kbd{pwd} prints the name of your
working directory:

@example
> system, "pwd"
/home/icf/munro
> system, "ls *.txt"
   damped.txt      junk.txt
@end example

This is so useful that there is a special escape syntax which
automatically generates the system call, so you don't need to type the
name @kbd{system} or all of the punctuation.  The rule is, that if the
first character of a Yorick statement is @kbd{$} (dollar), the
remainder of that line becomes a quoted string which is passed to the
@kbd{system} function.  Hence, you really would have typed:

@example
> $pwd
/home/icf/munro
> $ls *.txt
   damped.txt      junk.txt
@end example

Note that you cannot use @kbd{;} to stack @kbd{$} escaped lines ---
the semicolon will be passed to @kbd{system}.  Obviously, this syntax
breaks all of the ordinary rules of Yorick's grammar.

On UNIX systems, the @kbd{system} function (which calls the ANSI C
standard library function of the same name) usually executes your
command in a Bourne shell.  If you are used to a different shell, you
might be surprised at some of the results.  If you have some
complicated shell commands for which you need, say, the C-shell
@kbd{csh}, just start a copy of that shell before you issue your
commands:

@example
> $csh
% ls *.txt
  damped.txt    junk.txt
% exit
>
@end example

As this example shows, you can start up an interactive program with
the @kbd{system} function.  When you exit that program, you return to
Yorick.

One system command which you cannot use is @kbd{cd} (or @kbd{pushd} or
@kbd{popd}) to change directories.  The working directory of the shell
you start under Yorick will change, but the change has no effect on
Yorick's own working directory.  Instead, use Yorick's own @kbd{cd}
function:

@example
> cd, "new/working/directory"
>
@end example

Also, if you write a Yorick program which manipulates temporary files,
you should not use the UNIX commands @kbd{mv} or @kbd{rm} to rename or
remove the files; any time you use the @kbd{system} command you are
restricting your program to a particular class of operating systems.
Instead, use Yorick's @kbd{rename} and @kbd{remove} functions, which
will work under any operating system.  Yorick also has @kbd{mkdir},
@kbd{rmdir}, @kbd{lsdir}, @kbd{get_cwd}, and @kbd{get_env}
functions.


@node Errors,  , Shell commands, Environment
@subsection Error Messages

When Yorick cannot decipher the meaning of a statement, you have made
a @emph{syntax} error.  Syntax errors are mostly simple typos, but a
mechanical language like Yorick can be exasperatingly picky:

@example
> th= span(0,2*pi,200)
> for (i=1 ; i<=5 ; ++i) @{ r=cos(i*th); plg,r*sin(th),r*cos(th) @}
SYNTAX: parse error near @}
>
@end example

@noindent
The only mistake here is that Yorick wants a semicolon (or newline)
before the close curly brace completing a compound statement.  If you
think ``fixing'' this type of behavior would be simple, I suggest you
study parsers.  Every programming language has its quirks.

When Yorick detects a syntax error, the error message always begins with
@kbd{SYNTAX:}.  The entire block containing the error will be discarded;
no statements will be executed which might lead to the execution of the
statement containing the error.  If the error is in a function body, the
function will not be defined.  However, if the sytax error occurs
reading an include file, Yorick continues to parse the file looking for
additional syntax errors.  After about a dozen syntax errors in a single
file, Yorick gives up and waits for keyboard input.  Therefore, you may
be able to repair several syntax errors before you re-include the file.

All other errors are @emph{runtime} errors.

@menu
* Runtime errors::
* Simple debugging::
@end menu

@node Runtime errors, Simple debugging, Errors, Errors
@subsubsection Runtime errors

Runtime errors in Yorick are often simple typos, like syntax errors:

@example
> theta = span(0,2*pi,200)
> wave = sin(thta)*exp(-0.5*theta)
ERROR (*main*) expecting numeric argument
WARNING source code unavailable (try dbdis function)
now at pc= 1 (of 23), failed at pc= 5
 To enter debug mode, type <RETURN> now (then dbexit to get out)
>
@end example

@noindent
Many errors of this sort would be detected as syntax errors if you had
to declare Yorick variables.  Yorick's free-and-easy attitude toward
declaration of variables is particularly annoying when the offending
statement is in a conditional branch which is very rarely executed.
When a bug like that ambushes you, be philosophical: Minutely declared
languages will just ambush you in more subtle ways.

Other runtime errors are more interesting; often such a bug will teach
you about the algorithm or even about the physical problem:

@example
> #include "damped.i"
> theta = span(0, 6*pi, 300)
> amplitude = damped_wave(theta, 0.25)
ERROR (damped_wave) math library exception handler called
  LINE: 19  FILE: /home/icf/munro/damped.i
 To enter debug mode, type <RETURN> now (then dbexit to get out)
>
@end example

@noindent
What is an oscillator with a Q of less than one half?  Maybe you don't
care about the so-called @emph{overdamped} case --- you really wanted
Q to be 2.5, not 0.25.  On the other hand, maybe you need to modify
the @kbd{damped_wave} function to handle the overdamped case.


@node Simple debugging,  , Runtime errors, Errors
@subsubsection How to respond to a runtime error

When Yorick stops with a runtime error, you have a choice: You can
either type the next statment you want to execute, or you can type a
carriage return (that is, a blank line) to enter debug mode.  The two
possibilities would look like this:

@example
ERROR (damped_wave) math library exception handler called
  LINE: 19  FILE: /home/icf/munro/damped.i
 To enter debug mode, type <RETURN> now (then dbexit to get out)
> amplitude = damped_wave(theta, 2.5)
>
@end example

@example
ERROR (damped_wave) math library exception handler called
  LINE: 19  FILE: /home/icf/munro/damped.i
 To enter debug mode, type <RETURN> now (then dbexit to get out)
>
dbug>
@end example

In the second case, you have entered debug mode, and the @kbd{dbug>}
prompt appears.  In debug mode, Yorick leaves the function which was
executing and its entire calling chain intact.  You can type any Yorick
statement; usually you will print some values or plot some arrays to
try to determine what went wrong.  When you reference or modify a
variable which is local to the function, you will ``see'' its local
value:

@example
dbug> nu; 1-nu*nu
2
-3
dbug>
@end example

As soon as possible, you should escape from debug mode using the
@kbd{dbexit} function:

@example
dbug> dbexit
>
@end example

You may also be able to repair the function's local variables and
resume execution.  To modify the value of a variable, simply redefine
it with an ordinary Yorick statement.  The @kbd{dbcont} function
continues execution, beginning by re-executing the statement which
failed to complete.  Use the @kbd{help} function (@pxref{Help}) to learn
about the other debugging functions; the help for the @kbd{dbexit}
function describes them all.



@comment -----------------------------------------------------------------

@node    Arrays, Graphics, Basics, Top
@chapter Using Array Syntax

Most Yorick statements look like algebraic formulas.  A variable name is
a string like @kbd{Var_1} --- upper or lower case characters (case
matters), digits, or underscores in any combination except that the
first character may not be a digit.  Expressions consist of the usual
arithmetic operations @kbd{+ - * /}, with parentheses to indicate the
order of operations (when that order is different than or unclear from
the ordinary rules of precedence in algebra).  Elementary mathematical
functions such as @kbd{exp(x)}, @kbd{cos(x)}, or @kbd{atan(x)} look just
like that.

Usually, a Yorick variable is a parametric representation of a
mathematical function.  The variable is an array of numbers which are
values of the function at a number of points; few points to represent
the function coarsely, more for an accurate rendition.  The parameters
of the function are the indices into the array, which rarely make an
explicit appearance in Yorick programs.  Thus,

@example
theta = span(0.0, 2*pi, 100)
@end example

@noindent
defines a variable @kbd{theta} consisting of 100 evenly spaced values
starting with @kbd{0.0} and ending with @kbd{2*pi}.

Now that @kbd{theta} has been defined as a list of 100 numbers, any
function of @kbd{theta} has a concrete representation as a list of 100
numbers --- namely the values of the function at the 100 particular
values of @kbd{theta}.  Hence, variables @kbd{x} and @kbd{y}
representing coordinates of the unit circle are defined with:

@example
x = cos(theta);   y = sin(theta)
@end example

Here, @kbd{cos} and @kbd{sin} are built-in Yorick functions.  Like most
Yorick functions, they operate on an entire array of numbers, returning
an array of like shape.  Hence both @kbd{x} and @kbd{y} are now lists of
100 numbers -- the cosines and sines of the 100 numbers
@kbd{theta}.

The semicolon marks the end of a Yorick statement, allowing several
statements to share a single line.  The end of a line (i.e.- a newline)
can also mark the end of a Yorick statement.  However, if any
parentheses are open, or if a binary operator or a comma is the last
token on the line, then the newline is treated like a space or a tab
character and does not terminate the Yorick statement.

If a line ends with backslash, the following newline will never
terminate the Yorick statement.  (That is, backslash is the continuation
character in Yorick.)  I recommend that you never use a backslash -- end
the line to be continued with a binary operator, or leave the comma
separating subroutine arguments at the end of the line, or split a
parenthetic expression across the line, and it will be continued
automatically.

@menu
* Creating Arrays::             How to originate arrays.
* Interpolating::               Interpolation functions.
* Indexing::                    How to reference array elements.
* Sorting::                     How to sort an array.
* Transposing::                 How to change the order of array dimensions.
* Broadcasting::                Making arrays conformable.
* Dimension Lists::
@end menu

@node    Creating Arrays, Interpolating, Arrays, Arrays
@section Creating Arrays

Including the @kbd{span} function introduced in the previous section,
there are five common ways to originate arrays --- that is, to make an
array out of scalar values:

@table @kbd
@item [val1, val2, val3, ...]
The most primitive way to build an array is to enumerate the successive
elements of the array in square brackets.  The length of the array is
the number of values in the list.  An exceptional case is @kbd{[]},
which is called ``nil''.  A Yorick variable has the value nil before
it has been defined, and @kbd{[]} is a way to write this special value.


@item array(value, dimlist)
Use @kbd{array} to ``broadcast'' (@pxref{Broadcasting}) a value into an
array, increasing its number of dimensions.  A @kbd{dimlist} is a
standard list of arguments used by several Yorick functions
(@pxref{Dimension Lists}).  Often, the pseudo-index range function
(@pxref{Pseudo-Index}) is a clearer alternative to the @kbd{array}
function.

@item span(start, stop, number)
Use @kbd{span} to generate a list of equally spaced values.

@item indgen(n)
@itemx indgen(lower:upper)
@itemx indgen(start:stop:step)
Use @kbd{indgen} (``index generator'') instead of @kbd{span} to generate
equally spaced integer values.  By default, the list starts with 1.

@item spanl(start, stop, number)
Use @kbd{spanl} (``span logarithmically'') to generate a list of numbers
which increase or decrease by a constant ratio.
@end table

@node    Interpolating, Indexing, Creating Arrays, Arrays
@section Interpolating

As I have said, a Yorick array often represents the values of a
continuous function at a number of discrete points.  In order to find
the values of the function at other points, you need to know how it
varies between (or beyond) the given points.  In general, to interpolate
(or extrapolate), you need a detailed understanding of how the function
was discretized in the first place.  However, when the list of values
accurately represents the function, linear interpolation between the
known points will suffice.  A function which is linear between
successive points is called ``piecewise linear''.

The @kbd{interp} function is a mechanism for converting a list of
function values at discrete points into a piecewise linear function
which can be evaluated at any point.

@example
theta = span(0, pi, 100);
x_circle = cos(theta);
y_circle = sin(theta);
x = span(-2, 2, 64);
y = interp(y_circle, x_circle, x);
@end example

This code fragment produces a @kbd{y} array with the same number of
points as @kbd{x} (64), with the values of the piecewise linear function
defined by the points @kbd{(x_circle, y_circle)}.  Outside the range
covered by @kbd{x_circle}, the piecewise linear function remains
constant -- the simplest possible extrapolation rule.

Regarded as a function of its third argument, @kbd{interp} behaves just
like the @kbd{sin} or @kbd{cos} function -- its first two arguments are
really parameters specifying which piecewise linear function
@kbd{interp} will evaluate.

The @kbd{integ} function works just like @kbd{interp}, except that it
returns the integral of the piecewise linear function.  The integration
constant is chosen so that @kbd{integ} returns zero at the first point
of the piecewise linear function.  (This point will actually have the
maximum value of x if the x array is decreasing.)  Thus, the integral
of the piecewise linear approximation to the semicircle and the exact
integral of the semicircle can be computed by:

@example
yi = integ(y_circle, x_circle, x);
yi_exact = 0.5*(acos(max(min(x,1),-1)) - x*sqrt(1-min(x^2,1)));
@end example

Again, the piecewise linear function is assumed to remain constant
beyond the first and last points specified.  Hence, @kbd{integ} is a
linear function when extrapolating, and piecewise parabolic when
interpolating.

Use @kbd{integ} only when you need the indefinite integral of a
piecewise linear function.  Yorick has more efficient ways to compute
definite integrals.  Again, think of @kbd{integ}, like @kbd{interp}, as
a continuous function of its third argument; the first two arguments are
parameters specifying which function.

Neither @kbd{interp} nor @kbd{integ} makes sense unless its second
argument is either increasing or decreasing.  There is no way to decide
which branch of a multi-valued function should be returned.

Internally, both @kbd{interp} and @kbd{integ} need a lookup function --
that is, a function which finds the index of the point in x_circle just
beyond each of the x values.  This lookup function can also be called
directly; its name is @kbd{digitize}.



@node    Indexing, Sorting, Interpolating, Arrays
@section Indexing

Yorick has a bewildering variety of different ways to refer to
individual array elements or subsets of array elements.  In order to
master the language, you must learn to use them all.  Nearly all of the
examples later in this manual use one or more of these indexing
techniques, so trust me to show you how to use them later:

@itemize @bullet
@item
A scalar integer index refers to a specific array element.  In a
multi-dimensional array, this ``element'' is itself an array -- the
specified row, column, or hyperplane.

@item
An index which is nil or omitted refers to the entire range of the
corresponding array dimension.

@item
An index range of the form @kbd{start:stop} or @kbd{start:stop:step}
refers to an evenly spaced subset of the corresponding array
dimension.

@item
An array of integers refers to an arbitrary subset, possibly including
repeated values, of the corresponding array dimension.

@item
The special symbol @kbd{-}, or @kbd{-:start:stop}, is the pseudo-index.
It inserts an index in the result which was not present in the array
being indexed.  Although this sounds recondite, the pseudo-index is
often necessary in order to make two arrays conformable for a binary
operation.

@item
The special symbol @kbd{..} or @kbd{*} is a rubber-index.  A
rubber-index stands for zero or more dimensions of the array being
indexed.  This syntax allows you to specify a value for the final index
of an array, even when you don't know how many dimensions the array has.
The @kbd{..} version leaves the original dimension count intact; the
@kbd{*} version collapses all dimensions into a single long dimension in
the result.

@item
The special symbol @kbd{+} marks an index for matrix multiplication.

@item
Many statistical functions, such as @kbd{sum}, @kbd{avg}, and @kbd{max}
can be applied along a specific dimension of an array.  These reduce the
rank of the resulting array, just like a scalar integer index value.

@item
Several finite difference functions, such as @kbd{zcen} (zone center)
and @kbd{cum} (cumulative sums) can be applied along a specific array
dimension.  You can use these to construct numerical analogues to the
operations of differential and integral calculus.
@end itemize

@menu
* Scalar Index::                Scalar indices and array order.
* Index Range::                 Selecting a range of indices.
* Nil Index::                   Nil index refers to an entire dimension.
* Index List::                  Selecting an arbitrary list of indices.
* Pseudo-Index::                Creating a pseudo-index.
* Negative Index::              Numbering a dimension from its last element.
* Rubber-Index::                Using a rubber index.
* Matrix Multiply::             Marking an index for matrix multiplication.
* Statistical::                 Rank reducing (statistical) range functions.
* Finite Difference::           Rank preserving (finite difference) range functions.
@end menu

@node    Scalar Index, Index Range, Indexing, Indexing
@subsection Scalar indices and array order

An array of objects is stored in consecutive locations in memory (where
each location is big enough to hold one of the objects).  An array
@kbd{x} of three numbers is stored in the order @kbd{[x(1), x(2), x(3)]}
in three consecutive slots in memory.  A three-by-two array @kbd{y}
means nothing more than an array of two arrays of three numbers each.
Thus, the six numbers are stored in two contiguous blocks of three
numbers each: @kbd{[[x(1,1), x(2,1), x(3,1)], [x(1,2), x(2,2),
x(3,2)]]}.

A multi-dimensional array may be referenced using fewer indices than its
number of dimensions.  Hence, in the previous example, @kbd{x(5)} is the
same as @kbd{x(2,2)}, since the latter element is stored fifth.

Although most of Yorick's syntax follows the C language, array indexing
is designed to resemble FORTRAN array indexing.  In Yorick, as in
FORTRAN, the first (leftmost) dimension of an array is always the index
which varies fastest in memory.  Furthermore, the first element along
any dimension is at index 1, so that a dimension of length three can be
referenced by index 1 (the first element), index 2 (the second element),
or index 3 (the third element).

If this inconsistency bothers you, here is why Yorick indexing is like
FORTRAN indexing: In C, an array of three numbers, for example, is a
data type on the same footing as the data type of each of its three
members; by this trick C sidesteps the issue of multi-dimensional arrays
--- they are singly arrays of objects of an array data type.  While this
picture accurately reflects the way the multi-dimensional array is
stored in memory, it does not reflect the way a multi-dimensional array
is used in a scientific computer program.

In such a program, the fact that the array is stored with one or the
other index varying fastest is irrelevent --- you are equally likely to
want to consider as a ``data type'' a slice at a constant value of the
first dimension as of the second.  Furthermore, the length of every
dimension varies as you vary the resolution of the calculation in the
corresponding physical direction.



@node    Index Range, Nil Index, Scalar Index, Indexing
@subsection Selecting a range of indices

You can refer to several consecutive array elements by an index range:
@kbd{x(3:6)} means the four element subarray @kbd{[x(3), x(4), x(5),
x(6)]}.

Occasionally, you also want to refer to a sparse subset of an array; you
can add an increment to an index range by means of a second colon:
@kbd{x(3:7:2)} means the three element subarray @kbd{[x(3), x(5),
x(7)]}.

A negative increment reverses the order of the elements: @kbd{x(7:3:-2)}
represents the same three elements, but in the opposite order
@kbd{[x(7), x(5), x(3)]}.  The second element mentioned in the index
range may not actually be present in the resulting subset, for example,
@kbd{x(7:2:-2)} is the same as @kbd{x(7:3:-2)}, and @kbd{x(3:6:2)}
represents the two element subarray @kbd{[x(3), x(5)]}.

Just as the increment defaults to 1 if it is omitted, the start and stop
elements of an index range also have default values, namely the first
and last possible index values.  Hence, if @kbd{x} is a one-dimensional
array with 10 elements, @kbd{x(8:)} is the same as @kbd{x(8:10)}.  With
a negative increment, the defaults are reversed, so that @kbd{x(:8:-1)}
is the same as @kbd{x(10:8:-1)}.

A useful special case of the index range default rules is @kbd{x(::-1)},
which represents the array @kbd{x} in reverse order.

Beware of a minor subtlety: @kbd{x(3:3)} is not the same thing as
@kbd{x(3)}.  An index range always represents an array of values, while
a scalar index represents a single value.  Hence, @kbd{x(3:3)} is an
array with a single element, @kbd{[x(3)]}.



@node    Nil Index, Index List, Index Range, Indexing
@subsection Nil index refers to an entire dimension

When you index a multi-dimensional array, very often you want to let
one or more dimensions be ``spectators''.  In Yorick, you accomplish this
by leaving the corresponding index blank:

@example
x(3,)
x(,5:7)
y(,::-1,)
x(,)
@end example

In these examples, @kbd{x} is a 2-D array, and @kbd{y} is a 3-D array.
The first example, @kbd{x(3,)}, represents the 1-D array of all the
elements of @kbd{x} with first index 3.  The second represents a 2-D
array of all of the elements of @kbd{x} whose second indices are 5, 6,
or 7.  In the third example, @kbd{y(,::-1,)} is the @kbd{y} array with
the elements in reverse order along its middle index.  The fourth
expression, @kbd{x(,)}, means the entire 2-D array @kbd{x}, unchanged.





@node    Index List, Pseudo-Index, Nil Index, Indexing
@subsection Selecting an arbitrary list of indices

An index list is an array of index values.  Use an index list to specify
an arbitrary subset of an array: @kbd{x([5,1,2,1])} means the four
element array @kbd{[x(5), x(1), x(2), x(1)]}.  The @kbd{where} function
returns an index list:

@example
list= where(x > 3.5)
y= x(list)
@end example

@noindent
These lines define the array @kbd{y} to be the subset of the @kbd{x}
array, consisting of the elements greater than 3.5.

Like the result of an index range, the result of an index list is itself
an array.  However, the index list follows a more general rule: The
dimensionality of the result is the same as the dimensionality of the
index list.  Hence, @kbd{x([[5, 1], [2, 1]])} refers to the two
dimensional array @kbd{[[x(5), x(1)], [x(2), x(1)]]}.  The general rule
for index lists is:

@display
Dimensions from the dimensions of the index list; values from the array
being indexed.
@end display

@noindent
Note that the scalar index value is a special case of an index list
according to this rule.

The rule applies to multi-dimensional arrays as well: If @kbd{x} is a
five-by-nine array, then @kbd{x(, [[5, 1], [2, 1]])} is a
five-by-two-by-two array.  And @kbd{x([[5, 1], [2, 1]], 3:6)} is a
two-by-two-by-four array.



@node    Pseudo-Index, Negative Index, Index List, Indexing
@subsection Creating a pseudo-index

A binary operation applied between two arrays of numbers yields an array
of results in Yorick -- the operation is performed once for each
corresponding pair of elements of the operands.  Hence, if @kbd{rho} and
@kbd{vol} are each four-by-three arrays, the expression @kbd{rho*vol}
will be a four-by-three array of products, starting with
@kbd{rho(1,1)*vol(1,1)}.

This extension of binary operations to array operands is not always
appropriate.  Instead of operating on the corresponding elements of
arrays of the same shape, you may want to perform the operation between
all pairs of elements.  The most common example is the outer product
of two vectors @kbd{x} and @kbd{y}:

@example
outer= x*y(-,);
@end example

@noindent
Here, if @kbd{x} were a four element vector, and @kbd{y} were a three
element vector, @kbd{outer} would be the four-by-three array
@kbd{[[x(1)*y(1), x(2)*y(1), x(3)*y(1), x(4)*y(1)], [x(1)*y(2),
x(2)*y(2), x(3)*y(2), x(4)*y(2)], [x(1)*y(3), x(2)*y(3), x(3)*y(3),
x(4)*y(3)]]}.

In Yorick, this type of multiplication is still commutative.  That is,
@kbd{x*y(-,)} is the same as @kbd{y(-,)*x}.  To produce the three-by-four
transpose of the array @kbd{outer}, you would write @kbd{x(-,)*y}.

I call the @kbd{-} sign, when used as an index, a pseudo-index because
it actually inserts an additional dimension into the result array which
was not present in the array being indexed.  By itself, the expression
@kbd{y(-,)} is a one-by-three array (with the same three values as the
three element vector @kbd{y}).  You may insert as many pseudo-indices
into a list of subscripts as you like, at any location you like relative
to the actual dimensions of the array you are indexing.  Hence,
@kbd{outer(-,-,,-,)} would be a one-by-one-by-four-by-one-by-three array.


By default, a pseudo-index produces a result dimension of length one.
However, by appending an index range to the @kbd{-} sign, separated by
a colon, you can produce a new dimension of any convenient length:

@example
x = span(-10, 10, 100)(,-:1:50);
y = span(-5, 5, 50)(-:1:100,);
gauss2d = exp(-0.5*(x^2+y^2))/(2.0*pi);
@end example

@noindent
computes a normalized 2-D Gaussian function on a 100-by-50 rectangular
grid of points in the xy-plane.  The pseudo-index @kbd{-:1:50} has 50
elements, and @kbd{-:1:100} has 100.  For a pseudo-index of non-unit
length, the values along the actual dimensions are simply copied, so
that @kbd{span(-10, 10, 100)(,-:1:50)} is a 100-by-50 array consisting
of 50 copies of @kbd{span(-10, 10, 100)}.

If only the result @kbd{gauss2d} were required, a single default
pseudo-index would have sufficed:

@example
gauss2d = exp(-0.5*( span(-10,10,100)^2 +
                     span(-5,5,50)(-,)^2 )) / (2.0*pi);
@end example

However, the rectangular grid of points (@kbd{x},@kbd{y}) is often
required -- as input to plotting routines for example.



@node    Negative Index, Rubber-Index, Pseudo-Index, Indexing
@subsection Numbering a dimension from its last element

Array index values are subtly asymmetric: An index of 1 represents the
first element, 2 represents the second element, 3 the third, and so on.
In order to refer to the last, or next to last, or any element relative
to the final element, you apparently need to find out the length of the
dimension.

In order to remedy this asymmetry, Yorick interprets numbers less than 1
relative to the final element of an array.  Hence, @kbd{x(1)} and
@kbd{x(2)} are the first and second elements of @kbd{x}, while
@kbd{x(0)} and @kbd{x(-1)} are the last and next to last elements, and
so on.

With this convention for negative indices, many Yorick programs can
be written without the need to determine the length of a dimension:

@example
deriv = (f(3:0)-f(1:-2)) / (x(3:0)-x(1:-2));
@end example

@noindent
computes a point-centered estimate of the derivative of a function
@kbd{f} with values known at points @kbd{x}.  (A better way to compute
this derivative is to use the @kbd{pcen} and @kbd{dif} range functions
described below.  @xref{Finite Difference, , Rank preserving (finite
difference) range functions}.)

In this example, the extra effort required to compute the array length
would be slight:

@example
n = numberof(f);
deriv = (f(3:n)-f(1:n-2)) / (x(3:n)-x(1:n-2));
@end example

@noindent
However, using the negative index convention produces faster code, and
generalizes to multi-dimensional cases in an obvious way.

The negative index convention works for scalar index values and for the
start or stop field of an index range (as in the example).  Dealing with
negative indices in an index list would slow the code down too much, so
the values in an index list may not be zero or negative.



@node    Rubber-Index, Matrix Multiply, Negative Index, Indexing
@subsection Using a rubber index

Many Yorick functions must work on arrays with an unknown number of
dimensions.  Consider a filter response function, which takes as input a
spectrum (brightness as a function of photon energy), and returns the
response of a detector.  Such a function could be passed a spectrum and
return a scalar value.  Or, it might be passed a two dimensional array
of spectra for each of a list of rays, and be expected to return the
corresponding list of responses.  Or a three dimensional array of
spectra at each pixel of a two dimensional image, returning a two
dimensional array of response values.

In order to write such a function, you need a way to say, ``and all other
indices this array might have''.  Yorick's rubber-index, @kbd{..}, stands
for zero or more actual indices of the array being indexed.  Any indices
preceding a rubber-index are ``left justified'', and any following it are
``right justified''.  Using this syntax, you can easily index an array, as
long as you know that the dimension (or dimensions) you are interested
in will always be first, or last -- even if you don't know how many
spectator dimensions might be present in addition to the one your
routine processes.

Thus, as long as the spectral dimension is always the final dimension
of the input @kbd{brightness} array,

@example
brightness(..,i)
@end example

@noindent
will always place the @kbd{i} index in the spectral dimension, whether
@kbd{brightness} itself is a 1-D, 2-D, or 3-D array.

Similarly, @kbd{x(i,..)} selects a value of the first index of @kbd{x},
leaving intact all following dimensions, if any.  Constructions such as
@kbd{x(i,j,..,k,l)} are also legal, albeit rarely necessary.

A second form of rubber-index collapses zero or more dimensions into a
single index.  The length of the collapsed dimension will be the product
of the lengths of all of the dimensions it replaces (or 1, if it
replaces zero dimensions).  The symbol for this type of rubber index is
an asterisk @kbd{*}.  For example, if @kbd{x} were a
five-by-three-by-four-by-two array, then @kbd{x(*)} would be a 1-D array
of 120 elements, while @kbd{x(,*,)} would be a five-by-twelve-by-two
array.

If the last actual index in a subscripted array is nil, and if this
index does not correspond to the final actual dimension of the array,
Yorick will append a @kbd{..} rubber-index to the end of the subscript
list.  Hence, in the previous example, @kbd{x()} is equivalent to
@kbd{x(,..)}, which is equivalent to @kbd{x(..)}, which is equivalent to
simply @kbd{x}.  This rule is the only rogue in Yorick's array
subscripting stable, and I am mightily tempted to remove it on grounds
of linguistic purity.  When you mean, ``and any other dimensions which
might be present,'' use the @kbd{..} rubber-index, not a nil index.  Use
a trailing nil index only when you mean, ``and the single remaining
dimension (which I know is present).''



@node    Matrix Multiply, Statistical, Rubber-Index, Indexing
@subsection Marking an index for matrix multiplication

Yorick has a special syntax for matrix multiplication, or, more
generally, inner product:

@example
A(,+)*B(+,)
B(+,)*A(,+)
x(+)*y(+)
P(,+,,)*Q(,,+)
@end example

@noindent
In the first example, @kbd{A} would be an L-by-M array, @kbd{B} would be
an M-by-N array, and the result would be the L-by-N matrix product.  In
the second example, the result would be the N-by-L transpose of the
first result.  The general rule is that all of the ``spectator''
dimensions of the left operand precede the spectator dimensions of the
right operand in the result.

The third example shows how to form the inner product of two vectors
@kbd{x} and @kbd{y} of equal length.  The fourth example shows how to
contract the second dimension of a 4-D array @kbd{P} with the third
dimension of the 3-D array @kbd{Q}.  If @kbd{P} were 2-by-3-by-4-by-5
and @kbd{Q} were 6-by-7-by-3, the result array would be
2-by-4-by-5-by-6-by-7.

Unlike all of the other special subscript symbols (nil, @kbd{-},
@kbd{..}, and @kbd{*} so far), the @kbd{+} sign marking an index for use
in an inner product is actually treated specially by the Yorick parser.
The @kbd{+} subscript is a parse error unless the array (or expression)
being subscripted is the left or right operand of a binary @kbd{*}
operator, which is then parsed as matrix multiplication instead of
Yorick's usual element-by-element multiplication.  A parse error will
also result if only one of the operands has a dimension marked by
@kbd{+}.  Both operands must have exactly one marked dimension, and the
marked dimensions must turn out to be of equal length at run
time.



@node    Statistical, Finite Difference, Matrix Multiply, Indexing
@subsection Rank reducing (statistical) range functions

The beauty of Yorick's matrix multiplication syntax is that you ``point''
to the dimension which is to be contracted by placing the @kbd{+} marker
in the corresponding subscript.  In this section and the following
section, I introduce Yorick's range functions, which share the ``this
dimension right here'' syntax with matrix multiplication.  The topic in
this section is the statistical range functions.  These functions reduce
the rank of an array, as if they were a simple scalar index, but instead
of selecting a particular element along the dimension, a statistical
range function selects a value based on an examination of all of the
elements along the selected dimension.  The statistical function is
repeated separately for each value of any spectator dimensions.  The
available functions are:

@table @kbd
@item min
@itemx max
returns the minimum (or maximum) value.  These are also available as
ordinary functions.

@item sum
returns the sum of all the values.  This is also available as an
ordinary function.

@item avg
returns the arithmetic mean of all the values.  The result will be a
real number, even if the input is an integer array.  This is also
available as an ordinary function.

@item ptp
returns the peak-to-peak difference (difference between the maximum
and the minimum) among all the values.  The result will be positive
if the maximum occurs at a larger index than the minimum, otherwise
the result will be negative.

@item rms
returns the root mean square deviation from the arithmetic mean of the
values.  The result will be a real number, even if the input is an
integer array.

@item mnx
@itemx mxx
returns the index of the element with the smallest (or largest) value.
The result is always an integer index value, independent of the data
type of the array being subscripted.  If more than one element reaches
the extreme value, the result will be the smallest index.
@end table

The @kbd{min}, @kbd{max}, @kbd{sum}, and @kbd{avg} functions may also
be applied using ordinary function syntax, which is preferred if you
want the function to be applied across all the dimensions of an array
to yield a single scalar result.

Given the @kbd{brightness} array representing the spectrum incident on a
detector or set of detectors, the @kbd{mxx} function can be used to find
the photon energy at which the incident light is brightest.  Assume that
the final dimension of @kbd{brightness} is always the spectral
dimension, and that the 1-D array @kbd{gav} of photon energies (with the
same length as the final dimension of @kbd{brightness}) is also
given:

@example
max_index_list = brightness(.., mxx);
gav_at_max = gav(max_index_list);
@end example

@noindent
Note that @kbd{gav_at_max} would be a scalar if @kbd{brightness} were
a 1-D spectrum for a single detector, a 2-D array if @kbd{brightness}
were a 3-D array of spectra for each point of an image, and so on.


An arbitrary index range (@kbd{start:stop} or @kbd{start:stop:step}) may
be specified for any range function, by separating the function name
from the range by another colon.  For example, to select only a relative
maximum of @kbd{brightness} for photon energies above 1.0, ignoring
possible larger values at smaller energies, you could use:


@example
i = min(where(gav > 1.0));
max_index_list = brightness(.., mxx:i:0);
gav_at_max = gav(max_index_list);
@end example

@noindent
Note the use of @kbd{min} invoked as an ordinary function in the first
line of this example.  (Recall that @kbd{where} returns a list of
indices where some conditional expression is true.)  In the second line,
@kbd{mxx:i:0} is equivalent to @kbd{mxx:i:}.  Because of the details of
Yorick's current implementation, the former executes slightly faster.


More than one range function may appear in a single subscript list.  If
so, they are computed from left to right.  In order to execute them in
another order, you must explicitly subscript the expression resulting
from the first application:

@example
x = [[1, 3, 2], [8, 0, 9]];
max_min = x(max, min);
min_max = x(, min)(max);
@end example

@noindent
The value of @kbd{max_min} is 3; the value of @kbd{min_max} is 2.



@node    Finite Difference,  , Statistical, Indexing
@subsection Rank preserving (finite difference) range functions

Because Yorick arrays almost invariably represent function values,
Yorick provides numercal equivalents to the common operations of
differential and integral calculus.  In order to handle functions of
several variables in a straightforward manner, these operators are
implemented as range functions.  Unlike the statistical range functions,
which return a scalar result, the finite difference range functions do
not reduce the rank of the subscripted array.  Instead, they preserve
rank, in the same way that a simple index range @kbd{start:stop}
preserves rank.  The available finite difference functions are:


@table @kbd
@item cum
returns the cumulative sums (sum of all values at smaller index, the
first index of the result having a value of 0).  The result dimension
has a length one greater than the dimension in the subscripted array.

@item psum
returns the partial sum (sum of all values at equal or smaller index)
up to each element.  The result dimension has the same length as the
dimension in the subscripted array.  This is the same as @kbd{cum},
except that the leading 0 value in the result is omitted.

@item dif
returns the pairwise difference between successive elements, positive
when the function values increase along the dimension, negative when
they decrease.  The result dimension has a length one less than the
dimension of the subscripted array.  The @kbd{dif} function is the
inverse of the @kbd{cum} function.

@item zcen
returns the pairwise average between successive elements.  The result
dimension has a length one less than the dimension of the subscripted
array.  The name is short for ``zone center''.

@item pcen
returns the pairwise average between successive elements.  However, the
two endpoints are copied unchanged, so that the result dimension has a
length one greater than the dimension of the subscripted array.  The
name is short for ``point center''.

@item uncp
returns the inverse of the @kbd{pcen} operation.  (Gibberish will be the
result of applying this function to an array dimension which was not
produced by the @kbd{pcen} operation.)  The result dimension has a
length one less than the dimension of the subscripted array.  The name
is short for ``uncenter point centered''.
@end table

The derivative @i{dy/dx} of a function @i{y(x)}, where @i{y} and @i{x}
are represented by 1-D arrays @kbd{y} and @kbd{x} of equal length is:

@example
deriv = y(dif)/x(dif);
@end example

@noindent
Note that @kbd{deriv} has one fewer element than either @kbd{y} or
@kbd{x}.  The derivative is computed as if @i{y(x)} were the piecewise
linear function passing through the given points @kbd{(x,y)}; there is
one fewer line segment (slope) than point.

The values @kbd{x} and @kbd{y} can be called ``point-centered'', while the
values @kbd{deriv} can be called ``zone-centered''.  The @kbd{zcen} and
@kbd{pcen} functions provide a simple mechanism for moving back and
forth between point-centered and zone-centered quantities.  Usually,
there will be several reasonable ways to point-center zone centered data
and vice-versa.  For example:

@example
deriv_pc1 = deriv(pcen);
deriv_pc2 = y(dif)(pcen)/x(dif)(pcen);
deriv_pc3 = y(pcen)(dif)/x(pcen)(dif);
@end example

@noindent
For a well-resolved function, the differences among these three arrays
will be negligible.  That is, the differences are second order in
@kbd{x(dif)}, which is often the order of the errors in the calculation
that produced @kbd{x} and @kbd{y} in the first place.  If @kbd{x} and
@kbd{y} represent @i{y(x)} more accurately that this, then you must know
a better model of the shape of @i{y(x)} than the simple piecewise linear
model, and you should use that model to select @kbd{deriv_pc1},
@kbd{deriv_pc2}, and @kbd{deriv_pc3}, or some other expression.

An indefinite integral may be estimated using the trapezoidal rule:

@example
indef_integ = (y(zcen)*x(dif))(psum);
@end example

@noindent
Once again, @kbd{indef_integ} has one fewer point than @kbd{x} or
@kbd{y}, because there is one fewer trapezoid than point.  This time,
however, @kbd{indef_integ} is not zone centered.  Instead,
@kbd{indef_integ} represents values at the upper (or lower) boundaries
@kbd{x(2:)} (or @kbd{x(:-1)}).  Often, you want to think of the integral
of @i{y(x)} as a point centered array of definite integrals from
@kbd{x(1)} up to each of the @kbd{x(i)}.  In this case (which actually
arises more frequently), use the @kbd{cum} function instead of
@kbd{psum} in order to produce a result @kbd{def_integs} with the same
number of points as the @kbd{x} and @kbd{y} arrays:

@example
def_integs = (y(zcen)*x(dif))(cum);
@end example

For single definite integrals, the matrix multiply syntax can be used in
conjunction with the @kbd{dif} range function.  For example, suppose you
know the transmission fraction of the filter, @kbd{ff}, at several
photon energies @kbd{ef}.  That is, @kbd{ff} and @kbd{ef} are 1-D arrays
of equal length, specifying a filter transmission function as the
piecewise linear function connecting the given points.  The final
dimension of an array @kbd{brightness} represents an incident spectrum
(any other dimensions represent different rays, say one ray per pixel of
an imaging detector).  The 1-D array @kbd{gb} represents the group
boundary energies -- that is, the photon energies at the boundaries of
the spectral groups represented in @kbd{brightness}.  The following
Yorick statements compute the detector response:

@example
filter = integ(ff, ef, gb)(dif);
response = brightness(..,+)*filter(+);
@end example

For this application, the correct interpolation routine is @kbd{integ}.
The integrated transmission function is evaluated at the boundaries
@kbd{gb} of the groups; the effective transmission fraction for each
group is the difference between the integral at the upper bin boundary
and the lower bin boundary.  The @kbd{dif} range function computes these
pairwise differences.  Since the @kbd{gb} array naturally has one more
element than the final dimension of @kbd{brightness} (there is one more
group boundary energy than group), and since @kbd{dif} reduces the
length of a dimension by one, the @kbd{filter} array has the same length
as the final dimension of @kbd{brightness}.

Note that the @kbd{cum} function cannot be used to integrate
@kbd{ff(ef)}, because the points @kbd{gb} at which the integral must be
evaluated are not the same as the points @kbd{ef} at which the integrand
is known.  Whenever the points at which the integral is required are the
same (or a subset of) the points at which the integrand is known, you
should perform the integral using the @kbd{zcen}, @kbd{dif}, and
@kbd{cum} index functions, instead of the more general @kbd{integ}
interpolator.




@node    Sorting, Transposing, Indexing, Arrays
@section Sorting

The function @kbd{sort} returns an index list:

@example
list = sort(x);
@end example

The list has the same length as the input vector @kbd{x}, and list(1) is
the index of the smallest element of @kbd{x}, list(2) the index of the
next smallest, and so on.  Thus, @kbd{x(list)} will be @kbd{x} sorted
into ascending order.

By returning an index list instead of the sorted array, @kbd{sort}
simplifies the co-sorting of other arrays.  For example, consider a
series of elaborate experiments.  On each experiment, thermonuclear
yield and laser input energy are measured, as well as fuel mass.  These
might reasonably be stored in three vectors @kbd{yield}, @kbd{input},
and @kbd{mass}, each of which is a 1D array with as many elements as
experiments performed.  The order of the elements in the arrays maight
naturally be the order in which the experiments were performed, so that
@kbd{yield(3)}, @kbd{input(3)}, and @kbd{mass(3)} represent the third
experiment.  The experiments can be sorted into order of increasing
gain (yield per unit input energy) as follows:

@example
list = sort(yield/input);
yield = yield(list);
input = input(list);
mass = mass(list);
@end example

The inverse list, which will return them to their orginal order, is:

@example
invlist = list;  /* faster than  array(0, dimsof(list)) */
invlist(list) = indgen(numberof(list));
@end example

The @kbd{sort} function actually sorts along only one index of a
multi-dimensional array.  The dimensions of the returned list are the
same as the dimensions of the input array; the values are indices
relative to the beginning of the entire array, not indices for the
dimension being sorted.  Thus, @kbd{x(sort(x))} is the array @kbd{x}
sorted so that the elements along its first dimension are in ascending
order.  In order to sort along another dimension, pass @kbd{sort} a
second argument -- @kbd{x(sort(x,2))} will be @kbd{x} sorted into
increasing order along its second dimension, @kbd{x(sort(x,3))} along
its third dimension, and so on.

A related function is @kbd{median}.  It takes one or two arguments, just
like @kbd{sort}, but its result -- which is, of course, the median
values along the dimension being sorted -- has one fewer dimension than
its input.



@node    Transposing, Broadcasting, Sorting, Arrays
@section Transposing

In a carefully designed Yorick program, array dimensions usually wind up
where you need them.  However, you may occasionally wind up with a
five-by-seven array, instead of the seven-by-five array you wanted.
Yorick has a very general @kbd{transpose} function:

@example
x623451 = transpose(x123456);
x561234 = transpose(x123456, 3);
x345612 = transpose(x123456, 5);
x153426 = transpose(x123456, [2,5]);
x145326 = transpose(x123456, [2,5,3,4]);
x653124 = transpose(x123456, [2,5], [1,4,6]);
@end example

Here, @kbd{x123456} represents a six-dimensional array (hopefully these
will be rare).  The same array with its first and last dimensions
transposed is called @kbd{x623451}; this is the default result of
@kbd{transpose}.  The @kbd{transpose} function can take any number of
additional arguments to describe an arbitrary permutation of the indices
of its first argument -- the array to be transposed.

A scalar integer value, as in the second and third lines, represents a
cyclic permutaion of all the dimensions; the first dimension of the
input becomes the Nth dimension of the result, for an argument value of
N.

An array of integer values represents a cyclic permutation of the
specified dimensions.  Hence, in the fifth example, @kbd{[2,5,3,4]}
means that the second dimension becomes the fifth, the fifth becomes the
third, the third becomes the fourth, and the fourth becomes the second.
An arbitrary permutation may be built up of a number of cyclic
permutations, as shown in the sixth example.

One additional problem can arise in Yorick -- you may not know how many
dimensions the array to be transposed has.  In order to deal with this
possibility, either a scalar argument or any of the numbers in a cyclic
permutation list may be zero or negative to count from the last
dimension.  That is, 0 represents the last dimension, -1 the next to
last, -2 the one before that, and so on.

Typical transposing tasks are: (1) Move the first dimension to be last,
and all the others back one cyclically.  (2) Move the last dimension
first, and all the others forward one cyclically.  (3) Transpose the
first two dimensions, leaving all others fixed.  (4) Transpose the final
two dimensions, leaving all others fixed.  These would be accomplished,
in order, by the following four lines:

@example
x234561 = transpose(x123456, 0);
x612345 = transpose(x123456, 2);
x213456 = transpose(x123456, [1,2]);
x123465 = transpose(x123456, [0,-1]);
@end example



@node    Broadcasting, Dimension Lists, Transposing, Arrays
@section Broadcasting and conformability

Two arrays need not have identical shape in order for a binary operation
between them to make perfect sense:

@example
y = a*x^3 + b*x^2 + c*x + d;
@end example

The obvious intent of this assignment statement is that @kbd{y} should
have the same shape as @kbd{x}, each value of @kbd{y} being the value of
the polynomial at the corresponding @kbd{y}.

Alternatively, array valued coefficients @kbd{a}, @kbd{b}, ...,
represent an array of several polynomials -- perhaps Legendre
polynomials.  Then, if @kbd{x} is a scalar value, the meaning is again
obvious; @kbd{y} should have the same shape as the coefficient arrays,
each @kbd{y} being the value of the corresponding polynomial.

A binary operation is performed once for each element, so that
@kbd{a+b}, say, means

@example
a(i,j,k,l) + b(i,j,k,l)
@end example

for every @kbd{i, j, k, l} (taking @kbd{a} and @kbd{b} to be four
dimensional).  The lengths of corresponding indices must match in order
for this procedure to make sense; Yorick signals a ``conformability
error'' when the shapes of binary operands do not match.

However, if @kbd{a} had only three dimensions, @kbd{a+b} still makes
sense as:

@example
a(i,j,k) + b(i,j,k,l)
@end example

This sense extends to two dimensional, one dimensional, and finally
to scalar @kbd{a}:

@example
a + b(i,j,k,l)
@end example

which is how Yorick interpreted the monomial @kbd{x^3} in the first
example of this section.  The shapes of @kbd{a} and @kbd{b} are
conformable as long as the dimensions which they have in common all
have the same lengths; the shorter operand simply repeats its values
for every index of a dimension it doesn't have.  This repitition is
called ``broadcasting''.

Broadcasting is the key to Yorick's array syntax.  In practical
situations, it is just as likely for the @kbd{a} array to be missing the
second (@kbd{j}) dimension of the @kbd{b} array as its last (@kbd{l})
dimension.  To handle this case, Yorick will broadcast any unit-length
dimension in addition to a missing final dimension.  Hence, if the
@kbd{a} array has a second dimension of length one, @kbd{a+b}
means:

@example
a(i,1,k,l) + b(i,j,k,l)
@end example

for every @kbd{i, j, k, l}.  The pseudo-index can be used to generate
such unit length indices when necessary (@pxref{Pseudo-Index}).



@node    Dimension Lists,  , Broadcasting, Arrays
@section Dimension Lists

You should strive to write Yorick programs in such a way that you never
need to refer to the lengths of array dimensions.  Array dimensions are
usually of no direct significance in a calculation; your programs will
tend to be clearer if only the arrays themselves appear.

In practice, unfortunately, you can't always get by without mentioning
dimension lengths.  Two functions are important:

@table @kbd
@item numberof(x)
returns the total number of items in the array @kbd{x}.

@item dimsof(x)
returns a list consisting of the number of dimensions of the array @kbd{x},
followed by the length of each of those dimensions.  Thus, if @kbd{x} were
a nine-by-two-by-six array, @kbd{dimsof(x)} would return @kbd{[3,9,2,6]},
while if @kbd{x} were a scalar value, @kbd{dimsof(x)} would return
@kbd{[0]}.

@item dimsof(x, y, z, ...)
With multiple arguments, the @kbd{dimsof} function returns the dimension
list of the array @kbd{x+y+z+...}, or @kbd{[]} if the input arrays are not
conformable.
@end table

The @kbd{array} function (@pxref{Creating Arrays}), and the more arcane
functions @kbd{add_variable}, @kbd{add_member}, and @kbd{reshape} all
have parameter lists ending with one or more ``dimension list''
parameters.  Each parameter in a dimension list can be either a scalar
integer value, representing the length of a single dimension, or a
list of integers in the format returned by @kbd{dimsof} to represent
zero or more dimensions.  Several arguments can be used to build up
a complicated dimension list:

@example
x1 = array(0.0, 9, 2, 6);     /* 9-by-2-by-6 array of 0.0 */
x2 = array(0.0, [3,9,2,6]);  /* another 9-by-2-by-6 array */
x3 = array(0.0, 9, [0], [2,2,6]);   /* ...and yet another */

flux= array(0.0, 3, dimsof(z), numberof(groups));
@end example

In the final example, the @kbd{flux} might represent the three
components of a flux vector, at each of a number of positions @kbd{z},
and for each of a number of photon energies @kbd{groups}.  The first
dimension of @kbd{flux} has length three, corresponding to the three
components of each flux vector.  The last dimension has the same length
as the @kbd{groups} array.  In between are zero or more dimensions --
whatever the dimensions of the array of positions @kbd{z}.

By using the rubber index syntax (@pxref{Rubber-Index}), you can extract
meaningful slices of the @kbd{flux} array without ever needing to know
how many dimensions @kbd{z} had, let alone their lengths.




@comment -----------------------------------------------------------------

@node Graphics, Embedding, Arrays, Top
@chapter Graphics

Yorick's graphics functions produce most of the generic kinds of
pictures you see in scientific publications.  However, providing the
perfect graphics interface for every user is not a realistic design
goal.  Instead, my aim has been to provide the simplest possible
graphics model and the most basic plotting functions.  If you want more,
I expect you to build your own ``perfect'' interface from the parts I
supply.

My dream is to eventually supply several interfaces as interpreted code
in the Yorick distribution.  Currently, the best example of this
strategy is the @file{pl3d.i} interface, which I describe at the end of
this chapter.  Not every new graphics interface needs to be a major
production like @file{pl3d.i}, however.  Modest little functions are
arguably more useful; the plh function discussed below is an
example.

As you will see, the simplest possible graphics model is still very
complicated.  Unfortunately, I don't see any easy remedies, but I can
promise that careful study pays off.  I recommend the books ``The Visual
Display of Quantitative Information'' and ``Envisioning Information'' by
Edward Tufte for learning the fundamentals of scientific graphics.

@menu
* Plotting primitives::         The basic drawing functions.
* Plot limits::                 Setting plot limits, log scaling, etc.
* Display list::                The display list model.
* Hardcopy::                    How to get it.
* Graphics style::              How to change it.
* Query and edit::              Seeing legends and making minor changes.
* pldefault::                   Setting (non-default) defaults.
* Custom plot functions::       Combining the plotting primitives.
* Animation::                   Spielberg look out.
* 3D graphics::                 An experimental interface.
@end menu

@node Plotting primitives, Plot limits, Graphics, Graphics
@section Primitive plotting functions

Yorick features nine primitive plotting commands: plg plots polylines
or polymarkers, pldj plots disjoint line segments, plm, plc, and plf
plot mesh lines, contours, and filled (colored) meshes, respectively,
for quadrilateral meshes, pli plots images, plfp plots filled
polygons, plv plots vector darts, and plt plots text strings.  You can
write additional plotting functions by combining these primitives.

@menu
* plg::                         Plot graph.
* pldj::                        Plot disjoint lines.
* plm::                         Plot quadrilateral mesh.
* plc::                         Plot contours.
* plf::                         Plot filled quadrilateral mesh.
* pli::                         Plot image.
* plfp::                        Plot filled polygons.
* plv::                         Plot vectors.
* plt::                         Plot text.
@end menu

@node plg, pldj, Plotting primitives, Plotting primitives
@subsection plg

Yorick's plg command plots a one dimensional array of y values
as a function of a corresponding array of x values.  To be more
precise,

@example
plg, y, x
@end example

@noindent
plots a sequence of line segments from @kbd{(x(1),y(1))} to
@kbd{(x(2),y(2))} to @kbd{(x(3),y(3))}, and so on until
@kbd{(x(N),y(N))}, when x and y are N element arrays.

The ``backwards'' order of the arguments to plg (y,x instead of x,y)
allows for a default value of x.  Namely,

@example
plg, y
@end example

@noindent
plots y against 1, 2, 3, ..., N, or @kbd{indgen(numberof(y))} in Yorick
parlance.  You often want a plot of an array y with the horizontal
axis (y is plotted vertically) merely indicating the sequence of
values in the array.

Optional keyword arguments adjust line type (solid, dashed, etc.), line
color, markers placed along the line, whether to connect the last point
to the first to make a closed polygon, whether to draw direction arrows,
and other variations on the basic connect-the-dots theme.

Specifying line type 0 or "none" by means of the type= keyword causes
plg to plot markers at the points themselves, rather than a polyline
connecting the points.  Here is how you make a ``scatter plot'':

@example
plg, type=0, y, x
@end example

For a polymarker plot, x and y may be scalars or unit length arrays.  If
you need to specify your own marker shapes (perhaps to plot experimental
data points), you may want to use the plmk function -- use the help
function to find out how.

@node pldj, plm, plg, Plotting primitives
@subsection pldj

The pldj command also connects points, but as disjoint line segments:

@example
pldj, x0, y0, x1, y1
@end example

@noindent
connects @kbd{(x0(1),y0(1))} to @kbd{(x1(1),y1(1))}, then
@kbd{(x0(2),y0(2))} to @kbd{(x1(2),y1(2))}, and so on.  Unlike plg,
where y and x are one dimensional arrays, the four arguments to pldj may
have any dimensionality, as long as all four are the same shape.

@node plm, plc, pldj, Plotting primitives
@subsection plm

The plm command plots a quadrilateral mesh.  Two dimensional arrays x
and y specify the mesh.  The x array holds the x coordinates of the
nodes of the mesh, the y array the y coordinates.  If x and y are M by N
arrays, the mesh will consist of @kbd{(M-1)*(N-1)} quadrilateral zones.
The four nodes @kbd{(x(1,1),y(1,1))}, @kbd{(x(2,1),y(2,1))},
@kbd{(x(1,2),y(1,2))}, and @kbd{(x(2,2),y(2,2))} bound the first zone --
nodes @kbd{(1,1)}, @kbd{(2,1)}, @kbd{(1,2)}, and @kbd{(2,2)} for short.
This corner zone has two edge-sharing neighbors -- one with nodes
@kbd{(2,1)}, @kbd{(3,1)}, @kbd{(2,2)}, and @kbd{(3,2)}, and the other
with nodes @kbd{(1,2)}, @kbd{(2,2)}, @kbd{(1,3)}, and @kbd{(2,3)}.  Most
zones share edges four neighbors, one sharing each edge of the
quadrilateral.  The plm command:

@example
plm, y, x
@end example

@noindent
draws all @kbd{(M-1)*N+M*(N-1)} edges of the quadrilateral mesh.  Optional
keywords allow for separate color and linetype adjustments for both
families of lines (those with constant first or second index).

An optional third argument is an existence map -- not all
@kbd{(M-1)*(N-1)} zones need actually be drawn.  Logically, the
existence map is an @kbd{(M-1)} by @kbd{(N-1)} array of truth values,
telling whether a zone exists or not.  For historical reasons, plm
instead requires an M by N array called ireg, where @kbd{ireg(1,)} and
@kbd{ireg(,1)} (the first row and column) are all 0, the value for zones
which do not exist.  Furthermore, for zones which do exist, ireg can
take any positive value; the value of @kbd{ireg(i,j)} is the ``region
number'' (non-existent zones belong to region zero) of the zone bounded
by mesh nodes @kbd{(i-1,j-1)}, @kbd{(i,j-1)}, @kbd{(i-1,j)}, and
@kbd{(i,j)}.  The @kbd{boundary=} keyword causes plm to draw only the
edges which are boundaries of a single region (the @kbd{region=} keyword
value).

As a simple example, here is how you can draw a four by four zone
mesh missing its central two by two zones:

@example
x = span(-2, 2, 5)(,-:1:5);
y = transpose(x);
ireg = array(0, dimsof(x));
ireg(2:5,2:5) = 1;
ireg(3:4,3:4) = 0;
plm, y, x, ireg;
@end example

The plc, plf, and plfc commands plot functions on a quadrilateral mesh.

@node plc, plf, plm, Plotting primitives
@subsection plc and plfc

The plc command plots contours of a function @kbd{z(x,y)}.  If z is a two
dimensional array of the same shape as the x and y mesh coordinate
arrays, then

@example
plc, z, y, x
@end example

@noindent
plots contours of z.  The @kbd{levs=} keyword specifies particular contour
levels, that is, the values of z at which contours are drawn; by
default you get eight equally spaced contours spanning the full range
of z.  Each contour is actually a set of polylines.  Here is the
algorithm Yorick uses to find contour polylines:

Each edge of the mesh has a z value specified at either end.  For
those edges with one end above and the other below the desired contour
level, linearly interpolate z along the edge to find the (x,y) point
on the edge where z has the level value.

Start at any such point, choose one of the two zones containing that
edge, and move to the point on another edge of that zone, stepping
across the new edge to the zone on its opposite side.  Continue until
you reach the boundary of the mesh or the point at which you started.
If any points remain, repeat the process to get another disjoint
contour.  If you start at boundary points as long as any remain, you
will first walk all open contours -- those which run from one point on
the mesh boundary to another -- then all closed contours.  Note that
each contour level may consist of several polylines.

One additional complication can arise: Some zones might have two
diagonal corners above the contour value and the two other corners
below, so all four edges have points on the contour.  In a sense, your
mesh does not resolve this contour -- it could represent a true
X-point where contour lines cross, in which case you want to move
directly across the zone to the point on the opposite edge.
Unfortunately, if you connect the opposite edges, you will make very
ugly contour plots.  (Even if you disagree on my taste in other
places, trust me on this one.)

So the question is, which adjacent edge do you pick?  If you don't
make the same choice for every contour level you plot, contours at
different levels cross (and so will your eyes when you try to
interpret the picture).  By default, plc will use a sort of minimum
curvature algorithm: it turns the opposite direction that it turned
crossing the previous zone.  The @kbd{triangulate=} keyword to plc can be
used both to force a particular triangulation of any or all zones in
the mesh, and to return automatic triangulation decisions made during
the course of the plc command.  Again, if triangulation decisions
become important to you, your mesh is probably not fine enough to
resolve the contour you are trying to draw.

Making a good contour plot is an art, not an algorithm.  Contours work
best if used sparingly; more than a half dozen levels is likely to
result in an unintelligible picture.  I suggest you try drawing a very
few levels on top of a filled mesh plot (draw with plf).  You will need
to select a contrasting color, and you may want to call plc once per
level to get a different line type for each level.

If you need lots of levels, you should try the plfc function, which
fills the regions between successive contour levels with different
colors.  Beyond about six levels, contour lines drawn with plc will be
unintelligible, but with plfc you can get perhaps as many as twenty
distinguishable levels.

@example
plfc, z, y, x, levs=span(zmin,zmax,101);
@end example

@node plf, pli, plc, Plotting primitives
@subsection plf

The plf command plots a filled mesh, that is, it gives a solid color
to each quadrilateral zone in a two dimensional mesh.  The color is
taken from a continuous list of colors called a palette.  Different
colors represent different function values.  A palette could be a
scale of gray values from black to white, or a spectrum of colors from
red to violet.

Unlike plc, for which the z array had the same shape as x and y, the z
array in a plf command must be an M-1 by N-1 array if x and y are M by
N.  That is, there is one z value or color for each zone of the mesh
instead of one z value per node:

@example
plf, z, y, x
@end example

A separate palette command determines the sequence of colors which will
represent z values.  Keywords to plf determine how z will be scaled onto
the palette (by default, the minimum z value will get the first color in
the palette, and the maximum z the last color), and whether the edges of
the zones are drawn in addition to coloring the interior.  When the x
and y coordinates are projections of a two dimensional surface in three
dimensions, the projected mesh may overlap itself, in which case the
order plf draws the zones becomes important -- at a given (x,y), you
will only see the color of the last-drawn zone containing that point.
The drawing order is the same as the storage order of the z array,
namely @kbd{(1,1)}, @kbd{(2,1)}, @kbd{(3,1)}, ..., @kbd{(1,2)},
@kbd{(2,2)}, @kbd{(3,2)}, ..., @kbd{(1,3)}, @kbd{(2,3)}, @kbd{(3,3)},
...

One or two contours plotted in a contrasting color on top of a filled
mesh is one of the least puzzling ways to present a function of two
variables.  The fill colors give a better feel for the smooth
variation of the function than many contour lines, but the
correspondence between color and function value is completely
arbitrary.  One or two contour lines solves this visual puzzle nicely,
especially if drawn at one or two particularly important levels that
satisfy a viewer's natural curiosity.

As of yorick 1.5, z may also be a 3 by M-1 by N-1 array of type char in
order to make a true color filled mesh.  The first index of z is (red,
green, blue), with 0 minimum intensity and 255 maximum.

@node pli, plfp, plf, Plotting primitives
@subsection pli

Digitized images are usually specified as a two dimensional array of
values, assuming that these values represent colors of an array of
square or rectangular pixels.  By making appropriate x and y mesh
arrays, you could plot such images using the plf function, but the pli
command is dramatically faster and more efficient:

@example
pli, z, x0, y0, x1, y1
@end example

@noindent
plots the image with @kbd{(x0,y0)} as the corner nearest @kbd{z(1,1)}
and @kbd{(x1,y1)} the corner nearest @kbd{z(M,N)}, assuming z is an M by
N array of image pixel values.  The optional x0, y0, x1, y1 arguments
default to 0, 0, M, N.

As of yorick 1.5, z may also be a 3 by M by N array of type char in
order to make a true color filled mesh.  The first index of z is (red,
green, blue), with 0 minimum intensity and 255 maximum.

@node plfp, plv, pli, Plotting primitives
@subsection plfp

A third variant of the plf command is plfp, which plots an arbitrary
list of filled polygons; it is not limited to quadrilaterals.  While
pli is a special case of plf, plfp is a generalization of plf:

@example
plfp, z, y, x, n
@end example

Here z is the list of colors, and x and y the coordinates of the corners
of the polygons.  The fourth argument n is a list of the number of
corners (or sides) for each successive polygon in the list.  All four
arguments are now one dimensional arrays; the length of z and n is the
number of polygons, while the length of x and y is the total number of
corners, which is @kbd{sum(n)}.  Again, plfp draws the polygons in the
order of the z (or n) array.

As a special case, if all of the lengths n after the first are 1, the
first polygon coordinates are taken to be in NDC units, and the
remaining single points are used as offsets to plot @kbd{numberof(n)-1}
copies of this polygon.  This arcane feature is necessary for the plmk
function.

As of yorick 1.5, z may also be a 3 by @kbd{sum(n)} array of type char
in order to make a true color filled mesh.  The first index of z is
(red, green, blue), with 0 minimum intensity and 255 maximum.

@node plv, plt, plfp, Plotting primitives
@subsection plv

While plc, plf, pli, and plfp plot representations of a single valued
function of two variables, the plv command plots a 2D vector at each
of a number of (x,y) points.  The vector actually looks more like a
dart -- it is an isoceles triangle with a much narrower base than
height, with its altitude equal to the vector (u,v), in both magnitude
and direction, and its centroid at the point (x,y):

@example
plv, v, u, y, x
@end example

Making a good vector plot is very tricky.  Not only must you find a
nice looking length scale for your (u,v) vectors -- the longest should
be something like the spacing between your (x,y) points -- but also
you must sprinkle the (x,y) points themselves rather uniformly
throughout the region of your plot.  The time you spend overcoming
these artistic difficulties usually isn't worth the effort.

@node plt,  , plv, Plotting primitives
@subsection plt

The final plotting command, plt, plots text rather than geometrical
figures:

@example
plt, text, x, y
@end example

Optional keywords determine the font, size, color, and orientation of
the characters, and the precise meaning of the coordinates x and y --
what coordinate system they are given in, and how the text is justified
relative to the given point.

Unlike the other plotting primitives, by default the (x,y) coordinates
in the plt command do not refer to the same (x,y) scales as your data.
Instead, they are so-called normalized device coordinates, which are
keyed to the sheet of paper, should you print a hardcopy of your
picture.  To make (x,y) refer to the so-called world coordinates of
your data (what planet is your data from?), you must use the @kbd{tosys=1}
keyword.  If you do locate text in your world coordinate system, only
its position will follow your data as you zoom and pan through it;
don't expect text size to grow as you zoom in, or your characters to
become hideously distorted when you switch to log axis scaling.

Text may be rotated by multiples of 90 degrees by means of the
@kbd{orient=} keyword.  Arbitrary rotation angles are not supported, and
the speed that rotated text is rendered on your screen may be
dramatically slower than ordinary unrotated text.

You can get superscripts, subscripts, and symbol characters by means
of escape sequences in the text.  Yorick is not a typesetting program,
and these features will not be the highest possible quality.  Neither
will what you see on the screen be absolutely identical to your
printed hardcopy (that is never true, actually, but superscripts and
subscripts are noticeably different).  With those caveats, the escape
feature is still quite useful.

To get a symbol character (assuming you are a font other than symbol),
precede that character by an exclamation point -- for example,
@kbd{"!p"} will be plotted as the Greek letter pi.  There are four
exceptions: @kbd{"!!"}, @kbd{"!^"}, and @kbd{"!_"} escape to the
non-symbol characters exclamation point, caret, and underscore,
respectively.  And @kbd{"!]"} escapes to caret in the symbol font, which
is the symbol for perpendicular.  The exclamation point, underscore, and
right bracket characters are themselves in the symbol font, and
shouldn't be necessary as escaped symbols.  If the last character
in the text is an exclamation point, it has no special meaning; you
do not need to escape a trailing exclamation point.

Caret @kbd{"^"} introduces superscripts and underscore @kbd{"_"}
introduces subscripts.  There are no multiple levels of superscripting;
every character in the text string is either ordinary, a superscript, or
a subscript.  A caret switches from ordinary or subscript characters to
superscript, or from superscript to ordinary.  An underscore switches
from ordinary or superscript characters to subscript, or from subscript
back to ordinary.

If the text has multiple lines (separated by newline @kbd{"\n"}
characters), plt will plot it in multiple lines, with each line
justified according to the @kbd{justify=} keyword, and with the vertical
justification applied to the whole block.  You should always use the
appropriate text justification, since the size of the text varies
from one output device to another -- the size of the text you see on
your screen is only approximately the size in hardcopy.  In multiline
text, the superscript and subscript state is reset to ordinary at
the beginning of each line.

Here is an example of escape sequences:

@example
text = "The area of a circle is !pr^2\n"+
  "Einstein's field equations are G_!s!n_=8!pT_!s!n";
plt, text, .4,.7,justify="CH";
@end example


@node Plot limits, Display list, Plotting primitives, Graphics
@section Plot limits and relatives

A sequence of plotting primitives only partly determines your picture.
You will often want to specify the plot limits -- the range x and y
values you want to see.  You may also want log-log or semi-log scales
instead of linear scales, or grid lines that extend all the way across
your graph instead of just tick marks around the edges.  Finally, you
need to be able to specify the color palette used for pseudocoloring
any plf, pli, or plfp primitives.

The limits, logxy, gridxy, and palette functions are the interface
routines you need.  If you are really fussy, you can also control the
appearance of your picture in much more detail -- for example, the
thickness of the tick marks, the font of the labels, or the size and
shape of the plotting region.  The section on graphics styles explains
how to take complete control over the appearance of your graphics.
This section sticks with the functions you are likely to use
frequently and interactively.

@menu
* limits::                      Set plot limits.
* logxy::                       Set log axis scaling.
* gridxy::                      Set grid lines.
* palette::                     Set color palette.
* Color model::                 More about color.
@end menu

@node limits, logxy, Plot limits, Plot limits
@subsection limits

There are several ways to change the plot limits: The limits command,
the range command, and mouse clicks or drags.  Also, the unzoom
command undoes all mouse zooming operations.

The syntax of the limits command is:

@example
limits, xmin, xmax, ymin, ymax
@end example

Each of the specified limits can be a number, the string "e" to signal
the corresponding extreme value of the data, or nil to leave the limit
unchanged.  For example, to set the plot limits to run from 0.0 to the
maximum x in the data plotted, and from the current minimum y value to
the maximum y in the data, you would use:

@example
limits, 0.0, "e", , "e"
@end example

If both xmin and xmax (or ymin and ymax) are numbers, you can put xmin
greater than xmax (or ymin greater than ymax) to get a scale that
increases to the right (or down) instead of the more conventional
default scale increasing to the left (or up).

As a special case, limits with no arguments is the same as setting all
four limits to their extreme values (rather than the no-op of leaving
all four limits unchanged).  Hence, if you can't see what you just
plotted, a very simple way to guarantee that you'll be able to see
everything in the current picture is to type:

@example
limits
@end example

If you just want to change the x axis limits, type:

@example
limits, xmin, xmax
@end example

As a convenience, if you just want to change the y axis limits, you
can use the range function (instead of typing three consecutive commas
after the limits command):

@example
range, ymin, ymax
@end example

@menu
* mouse zooming::               How to zoom by mouse clicks.
* saving limits::               Save and restore plot limits.
* square limits::               Assure that circles are not ellipses.
@end menu

@node mouse zooming, saving limits, limits, limits
@subsubsection Zooming with the mouse

To zoom with using the mouse, put the mouse on the point you want to
zoom around.  Click the left button to zoom in on this point, or the
right button to zoom out.  If you drag the mouse between pressing and
releasing the button, the point under the mouse when you pressed the
button will scroll to the point where you release the button.  The
middle mouse button does not zoom, but it will scroll if you drag
while pressing it.  Hence, the left button zooms in, the middle button
pans, and the right button zooms out.

If you click just outside the edges of the plot, near the tick marks
around the edges of the plot, the zoom and pan operations will involve
only the axis you click on.  In this way you can zoom in on a region
of x (or y) without changing the magnification in y (or x).  Or with
the middle button, pan along one direction without having to worry
about accidentally changing the limits in the other direction
slightly.

An alternative way to scroll using the mouse is to hold the shift key
and press the left mouse button at one corner of the region you want
to expand to fill the screen.  Holding the button down, drag the mouse
to the opposite corner of your rectangle, then release the button to
perform the zoom.  This zoom operation is more difficult to control,
but it provides single step zooming with unequal x and y zoom factors.
(The inverse operation -- mapping the current full screen to fill the
rectangle you drag out with the mouse -- is available with shifted
right button.  This turns out to be unusably non-intuitive.)

After any mouse zoom function, all four limits are set to fixed
values, even if they were extreme values before the zoom.  You can
restore the pre-zoom limits, including any extreme value settings,
by typing:

@example
unzoom
@end example

@node saving limits, square limits, mouse zooming, limits
@subsubsection Saving plot limits

You can also invoke the limits command as a function, in which case it
returns the current value of [xmin,xmax,ymin,ymax,flags].  The flags
are bits that determine which of the limits (if any) were computed as
extreme values of the data, and a few other optional features (to be
mentioned momentarily).  The value returned by limits as the argument
to a later limits command restores the limits to a prior condition,
including the settings of extreme value flags.  Thus,

@example
         // mouse zooms here locate an interesting feature
detail1 = limits()
unzoom   // remove effects of mouse zooms
         // mouse zooms here locate second interesting feature
detail2 = limits()
limits, detail1    // look at first feature again
limits, detail2    // look at second feature again
limits   // return to extreme values
@end example

@node square limits,  , saving limits, limits
@subsubsection Forcing square limits

The square keyword chooses any extreme limit to force the x and y
scales to have identical units -- so that a circle will always look
round, not elliptical, for instance.  Of course, if you set all four
limits explicitly, the square keyword will have no effect.  For
example,

@example
limits, square=1, 0., "e", 0., "e"
@end example

forces the lower limits of both x and y to zero, while the upper
limits are chosen to both show all of the data in the first quadrant,
and to keep the units of x and y the same (normally, this means xmax
and ymax are equal, but if the viewport were not square, xmax and ymax
would be in the same ratio as the sides of the viewport).  The square
keyword remains in effect even if the units are changed, hence, for
example,

@example
limits
@end example

@noindent
will reset all limits to extreme values, but now one axis may extend
beyond the limits of the data in that direction to keep the x and y
units equal.  To return to the default behavior, you must explicitly
call the limits command with the square keyword again, for example,

@example
limits, square=0
@end example

@node logxy, gridxy, limits, Plot limits
@subsection logxy

You may want to choose among log, semi-log, or ordinary linear scale
axes.  Use the logxy command:

@example
logxy, 1, 1       // log-log plot
logxy, 1, 0       // semi-log plot, x logarithmic
logxy, 0, 1       // semi-log plot, y logarithmic
logxy, 0, 0       // linear axis scales
@end example

You can also omit the x or y flag to leave the scaling of that axis
unchanged:

@example
logxy, , 1
@end example

@noindent
changes the y axis to a log scale, leaving the x axis scaling
unchanged.

The flags returned by the limits function include the current logxy
settings, if you need them in a program.

Zero or negative values in your data have no catastrophic effects with
log axis scaling; Yorick takes the absolute value of your data and
adds a very small offset before applying the logarithm function.  As a
side effect, you lose any indication of the sign of your data in a log
plot.  If you insist you need a way to get log axis scaling for
oscillating data, write a function to treat negative points specially.
For example, to draw positive-y portions solid and negative-y portions
dashed, you might use a function like this:

@example
func logplg(y, x)
@{
  s = sum(y>=0.);  /* number of positive-y points */
  n = numberof(y);
  if (s) plg, max(y,0.), x, type="solid";
  if (s<n) plg, min(y,0.), x, type="dash";
@}
@end example

You always have the option of plotting the logarithm of a function,
instead of using log axes:

@example
logxy,0,1; plg,y,x
@end example

@noindent
plots the same curve as

@example
logxy,0,0; plg,log(y),x
@end example

The log axis scaling merely changes the position of the ticks and
their labels; the y axis ticks will look like a slide rule scale in
the first case, but like an ordinary ruler in the second.

@node gridxy, palette, logxy, Plot limits
@subsection gridxy

The gridxy command changes the appearance of your plot in a more
subjective way than logxy or limits.  Ordinarily, Yorick draws tick
marks resembling ruler scales around the edges of your plot.  With
gridxy, you can cause the major ticks to extend all the way across the
middle of your plot like the lines on a sheet of graph paper.  Like
logxy, with gridxy you get separate control over horizontal and
vertical grid lines.  However, with gridxy, if you supply only one
argument, both axes are affected (usually you want grid lines for
neither axis, or grid lines for both axes).  For example,

@example
gridxy, 1      // full grid, like graph paper
gridxy, 0, 1   // only y=const grid lines
gridxy, , 0    // turn off y=const lines, x unchanged
gridxy, 1,     // turn on x=const lines, y unchanged
gridxy, 0      // ticks only, no grid lines
@end example

If the flag value is 2 instead of 1, then instead of a full grid, only
a single grid line is drawn at the ``roundest number'' in the range
(zero if it is present).  If you need a reference line at y = 0, but
you don't want your graph cluttered by a complete set of grid lines,
try this:

@example
gridxy, 0, 2
@end example

The appearance of the ticks and labels is actually part of the
graphics style.  You will want to change other details of your
graphics style far less frequently than you want to turn grid lines on
or off, which explains the separate gridxy function controlling this
one aspect of graphics style.

The gridxy command also accepts keyword arguments which change the
default algorithm for computing tick locations, so that ticks can appear
at multiples of 30.  This makes axes representing degrees look better
sometimes.  By default, ticks can appear only at ``decimal'' locations,
whose last non-zero digit is 1, 2, or 5.  See the online help for gridxy
for a more complete description of these options.

@node palette, Color model, gridxy, Plot limits
@subsection palette

The plf, pli, and plfp commands require a color scale or palette,
which is a continuum of colors to represent the continuous values of a
variable.  Actually, a palette consists of a finite number of (red,
green, blue) triples, which represent a color for each of a finite
list of values.  A Yorick palette can never have more than 256 colors,
so that a type char variable (one byte per item) can hold any index
into a palette.  Because many screens can display only 256 colors
simultaneously, however, you shouldn't have more than about 200 colors
in a palette; that is the size of all of Yorick's predefined palettes.

The palette command allows you to change palettes.  The Yorick
distribution comes with predefined palettes called @file{viridis.gp}
(the default palette), @file{inferno.gp}, @file{magma.gp},
@file{plasma.gp}, @file{coolwarm.gp}, @file{earth.gp}, @file{gray.gp},
@file{yarg.gp}, @file{heat.gp}, @file{stern.gp}, and @file{rainbow.gp}.
To load the gray palette, you would type:

@example
palette, "gray.gp"
@end example

These palettes tend to start with dark colors and progress toward
lighter colors.  The exceptions are @file{yarg.gp}, which is a reversed
version of @file{gray.gp} (your picture looks like the photographic
negative of the way it looks with the @file{gray.gp} palette), and
@file{rainbow.gp}, which runs through the colors in spectral order at
nearly constant intensity.  Besides @file{gray.gp} (or @file{yarg.gp})
and @file{rainbow.gp}, it's tough to find color sequences that people
have been trained to think have an order.  The @file{heat.gp} palette is
a red-orange scale resembling the colors of an iron bar as it grows
hotter.  The default @file{earth.gp} is loosely based on mapmaker's
colors from dark blue deep ocean to green lowlands to brown highlands to
white mountains.

Instead of a file name, you may pass palette three arrays of numbers
ranging from 0 to 255, which are relative intensities of red, green,
and blue.  For example,

@example
scale = bytscl(indgen(200),top=255);
palette, scale,scale,scale;
@end example

@noindent
produces the same palette as @file{gray.gp}, but by direct specification
of RGB values, rather than by reading a palette file.

Yorick internally uses the bytscl function to map the z values in a
plf, pli, or plfp command into a (0-origin) index into the palette.
Occasionally, as here, you will also want to call bytscl explicitly.

The predefined palette files are in the directory @kbd{Y_SITE+"gist"};
you should be able to figure out their format easily if you want to
produce your own.  If you create a directory @file{~/Gist} and put your
custom palette files there, Yorick will find them no matter what its
current working directory.  The library include file @file{color.i}
includes functions to help you construct palettes, and a dump_palette
function which writes a palette in Yorick's standard format.

You can use the @kbd{query=} keyword to retrieve the RGB values for the
currently installed palette:

@example
local r,g,b;
palette,query=1, r,g,b;
@end example

There is also a @kbd{private=} keyword to palette, which you should
investigate if you are interested in color table animation, or if
other programs steal all your colors.

@node Color model,  , palette, Plot limits
@subsection Color model

The line drawing primitives plg, pldj, plm, plc, and plv accept a
@kbd{color=} keyword.  You can use an index into the current palette, in
which case the color of the line becomes dependent on the palette.
However, you can also choose one of ten colors which are always
defined, and which do not change when the palette changes.  For
example,

@example
plg, sin(theta),cos(theta), color="red"
@end example

@noindent
draws a red circle (assuming theta runs from zero to two pi).  The
colors you can specify in this way are: @kbd{"black"}, @kbd{"white"},
@kbd{"red"}, @kbd{"green"}, @kbd{"blue"}, @kbd{"cyan"}, @kbd{"magenta"},
@kbd{"yellow"}, @kbd{"fg"}, and @kbd{"bg"}.

Black and white, and the primary and secondary colors need no further
explanation.  But @kbd{"fg"} stands for foreground and @kbd{"bg"} for
background.  The default color of all curves is @kbd{"fg"}; the color of
the blank page or screen is @kbd{"bg"}.  These colors are intentionally
indefinite; unlike the others, they may differ on your screen and in
print.  By default, @kbd{"fg"} is black and @kbd{"bg"} is white.  If you
use the X window system, you can change these defaults by setting X
resources.  For example, set

@example
Gist*foreground: white
Gist*background: black
@end example

@noindent
to make Yorick draw in reversed polarity on your screen (white lines
on a black background).  (The X utility xrdb is the easiest way to set
X resources for your screen.)

As of yorick 1.5, the color may also be an array of three values (red,
green, blue), with 0 minimum intensity and 255 maximum.  If you specify
a true color on a display which only supports pseudocolor, that window
will switch irreversibly to a 5x9x5 color cube, which causes a
significant degradation in rendering quality with smooth palettes.

@node Display list, Hardcopy, Plot limits, Graphics
@section Managing a display list

Most graphics are overlays produced by several calls to the primitive
plotting functions.  For example, to compare of three calculations of,
say, temperature as a function of time, type this:

@example
plg, temp1, time1
plg, temp2, time2
plg, temp3, time3
@end example

All three curves will appear on a single plot; as you type each plg
command, you see that curve appear.

The basic paradigm of Yorick's interactive graphics package is that as
you type the command to add each curve to the plot, you see that curve
appear.  You may need to open a file or perform a calculation in order
to generate the next curve.  Often, an interesting comparison will
occur to you only after you have seen the first few curves.

Even if you are not changing the data in a plot, you often will want
to change plot limits, or see how your data looks with log or semi-log
axes.  You can call limits, logxy, or gridxy any time you want; the
changes you request take place immediately.

Yorick maintains a display list -- a list of the primitive plotting
commands you have issued to display your data -- to enable you to make
these kinds of changes to your picture.  The primitive functions just
add items to the display list; you don't need access to the actual
rendering routines.  You can change secondary features like plot
limits or log axis scaling changing the plotted data.  In order to
actually produce a picture, Yorick must ``walk'' its display list,
sending appropriate plotting commands to some sort of graphics output
device.

If the picture changes, Yorick automatically walks its display list
just before pausing to wait for your next input line.  Thus, while you
are thinking about what to do next, you are always looking at the
current state of the display list.

A command like limits does not actually force an immediate replot; it
just raises a flag that tells Yorick it needs to walk its display
list.  Therefore, feel free to call limits or logxy several times if a
Yorick program is clearer that way -- the effects will be cumulative,
but you will not have to wait for Yorick to redraw your screen each
time -- the replot only happens when Yorick pauses to walk the display
list.

Usually, the display list paradigm does just what you expect.
However, the indirection can produce some subtle and surprising
effects, especially when you write a Yorick program that produces
graphical output (rather than just typing plotting commands directly).

@menu
* fma::                         Frame advance (begin next picture).
* multiple windows::            How to get them.
@end menu

@node fma, multiple windows, Display list, Display list
@subsection fma and redraw

The fma command (mnemonic for frame advance) performs two functions:
First, if the current display list has changed, Yorick re-walks it,
ensuring that your screen is up to date.  Afterwards, fma clears the
display list.

In normal interactive use, you think of the first function -- making
sure what you see on your screen matches the display list -- as a side
effect.  Your aim is to clear the display list so you can begin a
fresh picture.  However, in a Yorick program that is making a movie,
the point is to draw the current frame of the movie, and clearing the
display list is the side effect which prepares for the next movie
frame.

Another side effect of fma is more subtle: Since Yorick no longer
remembers the sequence of commands required to draw the screen,
attempts to change plot limits (zoom or pan), rescale axes, or any
other operation requiring a redraw will not change your screen after
an fma, even though you may still see the picture on your screen.

Once set, limits are persistent across frame advances.  That is, after
an fma, the next plot will inherit the limits of the previous plot.  I
concede that this is not always desirable, but forgetting the previous
limits can also be very annoying.  Another possibility would be to
provide a means for setting the initial limits for new plots; I judged
this confusing.  You get used to typing

@example
fma; limits
@end example

@noindent
when you know the old limits are incorrect for the new plot.  The axis
scaling set by logxy, and grid lines set by gridxy are also persistent
across frame advances.

On rare occasions, Yorick may not update your screen properly.  The
redraw command erases your screen and walks the entire display list.
You can also use this in Yorick programs to force the current picture
to appear on your screen immediately, without the side effect of
clearing the display list with fma.  (Perhaps you want to start
looking at a rapidly completed part of a picture while a more lengthy
calculation of the rest of the picture is in progress.)

Note that redraw unconditionally walks the entire display list, while
fma only walks the part that hasn't been previously walked or damaged
by operations (like limits changes) that have occurred since the last
walk.

@node multiple windows,  , fma, Display list
@subsection Multiple graphics windows

You can have up to eight graphics windows simultaneously.  Each window
has its own display list, so you can build a picture in one window,
switch to a second window for another plot, then return to the first
window.  The windows are numbered 0 through 7.  The window command
switches windows:

@example
window, 1       // switch to window number 1
window, 0       // switch back to default window
@end example

If you switch to a window you have never used before, a new graphics
window will appear on your screen.  (If you do not issue an explicit
window command before your first graphics command, Yorick
automatically creates window number 0 for you.)

The window command takes several keywords.  The @kbd{wait=1} keyword
pauses Yorick until the new window actually appears on your screen (if
the window command creates a new window).  This is important under the X
window system, which cannot draw anything until a newly created window
actually appears.  If you write a program (such as Yorick's demo
programs) which will produce graphical output, but you are not sure
whether there have been any previous graphics commands, you should start
with the statement:

@example
window, wait=1;
@end example

The @kbd{style=} keyword to window specifies a particular graphics style
for the window.  I'll return to graphics styles later.

The @kbd{dpi=100} keyword specifies a 100 dot per inch window, rather
than the default 75 dot per inch window.  The window you see on your
screen is a replica of a portion of an 8.5 by 11 inch sheet of paper,
and the dpi (either 75 or 100) refers to the number of screen pixels
which will correspond to one inch on that sheet of paper if you were to
tell Yorick to print the picture (see hcp).  Owing to the fonts
available with X11R4, Yorick permits only the two scale factors.

Initially, a Yorick window is a six inch by six inch window on the 8.5
by 11 sheet, centered at the center of the upper 8.5 by 8.5 inch square
for portrait orientation, or at the center of the rectangular sheet for
landscape orientation.  (Portrait or landscape orientation is a function
of the graphics style you choose.  Six inches is 450 pixels at 75 dpi or
600 pixels at 100 dpi, so @kbd{dpi=75} makes a small window, while
@kbd{dpi=100} makes a large window.)  You can resize the window (with
your window manager), but Yorick will always use either the 75 or 100
dot per inch scaling, so resizing a Yorick window is not very useful
unless you want to use more than the six by six inch square that is
initially visible.

The @kbd{display=} keyword specifies a display other than your default
display.  For example, if you have two screens you might use

@example
window, 0
window, 1, display="zaphod:0.1"
@end example

@noindent
(if zaphod is the name of the machine running your X server) in order
to create window number 0 on your default screen and window number 1
on your second screen.

You will rarely need the other keywords to the window command, which
allow for a private colormap (for color table animation), turn the
legends off in hardcopy output, and create a private hardcopy file for
the sole use of that one window.  With the latter capability, you can
write commands like eps that print the current picture, without
affecting the main hardcopy file.

@node Hardcopy, Graphics style, Display list, Graphics
@section Getting hardcopy

Type:

@example
hcp
@end example

@noindent
to add the picture you see on your screen to the current hardcopy file.
Yorick walks the current display list, rendering to the hardcopy file
instead of to your screen.  To close the hardcopy file and send it to
the printer, use:

@example
hcp_out
@end example

Normally, hcp_out destroys the file after it has been printed; you
can save the file in addition to printing it with:

@example
hcp_out, keep=1
@end example

If you want to close the hardcopy file without printing it, you can
call hcp_finish instead of hcp_out; if you invoke hcp_finish as a
function (with a nil argument), it returns the name of the file it
just closed.

After you call hcp_out or hcp_finish, the next call to hcp creates a
new hardcopy file.  You can call hcp_file to set the name of the next
hardcopy file.  For example,

@example
hcp_file, "myfile.ps"
@end example

@noindent
means that the next time Yorick creates hardcopy file, its name will
be @file{myfile.ps}.  The name of the file also implicitly determines
the file type; names of the form @file{*.ps} will be PostScript files,
while a file name of the form @file{*.cgm} will be a binary CGM.  If
you do not specify a file name, Yorick chooses a name will not clobber
any existing file; if you specify a name and that file already exists,
Yorick will silently overwrite the existing file.

@menu
* Color hardcopy::              Dumping palettes into hardcopy files.
* CGM hardcopy::                Caveats about binary CGM format.
* EPS hardcopy::                Encapsulated PostScript output.
@end menu

@node Color hardcopy, CGM hardcopy, Hardcopy, Hardcopy
@subsection Color hardcopy

When you produce a color picture (with plf or pli, for example) on your
screen, you need to save not only the picture, but also the palette you
used to draw that picture.  Since the palette might change for each
frame in a hardcopy file, the file must contain a complete palette as a
sort of preamble to each frame.  However, you usually want less
expensive, higher quality black and white output, so the palette
information is usually unnecessary.  Yorick optionally converts colors
to a standard gray scale before output to a hardcopy file, saving a
little space in the output file.

If you are going to use a black and white printer, you can turn off
palette dumping by means of the @kbd{dump=} keyword:

@example
hcp_file, dump=0
@end example

@noindent
(you could also specify the name for the next hardcopy file).  Unlike
the name of the hardcopy file, the palette dumping flag persists across
any number of hardcopy files.  To turn on dumping again, you need to
call hcp_file with @kbd{dump=1}.

All Yorick PostScript files will print on any PostScript printer, black
and white or color.  However, if you set @kbd{dump=0}, the palette was
not dumped to the file, and it will print gray even on a color printer.
If you print a @kbd{dump=1} file on a black and white printer, the
result will be very similar to @kbd{dump=0}; your colors will be
translated to grays.  You can put the call to hcp_file setting
@kbd{dump=0} in your @file{custom.i} file if you rarely want to dump the
palette.

@node CGM hardcopy, EPS hardcopy, Color hardcopy, Hardcopy
@subsection Binary CGM caveats

A Yorick binary CGM conforms to all the recommendations of the ANSI
Computer Graphics Metafile standard.  Unfortunately, this standard is
very outdated.  Unlike PostScript, the CGM standard does not specify
where the ink has to go on the page, so that every program which
interprets a CGM draws a somewhat different picture.  Fonts and line
types vary a great deal, as does the absolute scale of the picture.
The Yorick distribution includes a binary CGM browser, gist (Gist is
the name of Yorick's graphics package).  The gist browser can convert
a binary CGM generated by Yorick into exactly the same PostScript file
that Yorick would have produced.  No other CGM reader can do the
same.

Yorick includes CGM support for historical reasons.  Unless you like
using the gist browser program, you should write PostScript files
directly.  If you want to archive lots of pictures, you may be
concerned that the PostScript file is much larger than the equivalent
CGM.  However, standard compression software (such as gzip from
project GNU) works much better on the text of a PostScript file than
on the binary CGM data, which erases most of the size discrepancy.

If you are serious about archiving pictures, you should strongly
consider archiving the raw data plus the Yorick program you used to
create the pictures, rather than any graphics hardcopy files.  Often
this is even more compact than the graphics file, and it has the huge
advantage that you can later recover the actual data you plotted,
making it easy to plot it differently or overlay data from other
sources.

@node EPS hardcopy,  , CGM hardcopy, Hardcopy
@subsection Encapsulated PostScript

If your system includes the ghostscript program from project GNU, you
can use it to make Encapsulated PostScript (EPS) files from Yorick
PostScript output.  Yorick's eps command will do this, although it
forks Yorick in order to start ghostscript, which is a bad idea if
Yorick has grown to a large size.  (The gist browser which is part of
the Yorick distribution can make an inferior EPS file; you can change
eps to use gist if you need to.)  When you see the picture you want
on your screen, type:

@example
eps, "yowza"
@end example

to create an EPS file called @file{yowza.eps} containing that picture.
If you don't want to fork Yorick, you can also use the hcps command,
which creates a PostScript file containing just the current picture; you
can then run ps2epsi by hand to convert this to an EPS file.

Page layout programs such as FrameMaker or xfig can import EPS files
generated by Yorick.  You can then use the page layout program to
resize the Yorick graphic, place other text or graphics around it (as
for a viewgraph), and even add arrows to call out features of your
plot.  This is the easiest way to make finished, publication quality
graphics from Yorick output.  You may be able to produce satisfactory
results using Yorick alone, but the page layout programs will always
have fancier graphical user interfaces -- they are and will remain
better for page layout than Yorick.

My recommendation to set up for high quality output is this:

@example
func hqg(onoff)
/* DOCUMENT hqg
            hqg, 0
     turn on and off high quality graphics output.  You should
     refrain from using plot or axis titles; add them with your
     page layout program.  You will also need to distinguish
     curves by means of line type as hq turns off alphabetic
     curve markers.
   SEE ALSO: eps
 */
@{
  if (is_void(onoff) || onoff) @{
    window, style="vg.gs", legends=0;
    pldefault, marks=0, width=4;
  @} else @{
    window, style="work.gs", legends=0;
    pldefault, marks=1, width=0;
  @}
@}
@end example

@node Graphics style, Query and edit, Hardcopy, Graphics
@section Graphics style

Yorick formats each page of graphics output according to a model
regulated by several dozen numeric parameters.  Most of these change
seldom -- you tend to find a graphics style that suits you, then stick
with it for many plots.  Yorick demotes the stylistic parameters to a
lower rank; they are intentionally harder to access and change than
more urgent items like plot limits or log scaling.  (The gridxy
function is an exception -- it changes the stylistic parameters that
determine how Yorick draws ticks, but has a relatively simple
interface.)

Designing a new graphics style requires patience, hard work, and
multiple iterations.  Use the @file{*.gs} files which come with the
Yorick distribution (installed in @kbd{Y_SITE+"gist"}) as examples and
starting points for your own designs.  The following sections should
help you to understand the meaning of the various style parameters, but
I am not going to attempt to lead you through an actual design.
Yorick's predefined graphics styles usually suffice.

@menu
* style keyword::               Accessing predefined graphics styles.
* style.i::                     Bypassing predefined graphics styles.
* plsys::                       Multiple coordinate systems.
* ticks and labels::            How to change them.
@end menu

@node style keyword, style.i, Graphics style, Graphics style
@subsection Style keyword

Graphics styles are ordinarily read from files, which you load by
means of the style= keyword to the window command.  The Yorick
distribution comes with several style files:

The @file{work.gs} file is the default style and my own preference.
Many people prefer the @file{boxed.gs} style, which looks more like
other graphics packages (baaa baaa).  The @file{vg.gs} file stands for
``viewgraph graphics style'', which I recommend as the starting point
for high quality graphics.  There is a @file{vgbox.gs} style as well.
The @file{axes.gs} style has coordinate axes with ticks running through
the middle of the viewport, similar to the style of many elementary math
textbooks.  The @file{nobox.gs} style has no tick marks, labels, or
other distractions; use it for drawing geometrical figures or imitating
photographs.  The @file{work2.gs} and @file{boxed2.gs} styles are
variants which allow you to put a second set of tick marks and labels on
the right hand side of the plot, independent of the ones on the left
side.  (This is a little cumbersome in Yorick, but if you really need
it, that's how you're supposed get it.)  Finally, @file{l_nobox.gs} is a
variant of the nobox style in landscape orientation; all the other
styles give portrait orientation.

For example,

@example
window, style="nobox.gs"
@end example


switches to the nobox style.  This clears the current display list;
the new style will take effect with your next plot.  If you have
several different windows, they may have different styles; each window
has its own style, just as it has its own display list.

The style files are in the @kbd{Y_SITE+"gist"} directory; the file
@file{work.gs} in that directory contains enough comments for you to see
how to write your own variants of the distribution style files.  You can
put variants into either that public directory, or into a private
directory called @file{~/Gist}, or in Yorick's current working directory
to enable the style= keyword to find them.

@node style.i, plsys, style keyword, Graphics style
@subsection @file{style.i} functions

If you need to control details of the graphics style from a Yorick
program, you can include the library file @file{style.i}:

@example
#include "style.i"
@end example

This defines functions get_style and set_style which enable you to
determine and change every detail of Yorick's graphics style model.
Functions read_style and write_style read from or write to a text file
in the format used by the @kbd{style=} keyword.

@node plsys, ticks and labels, style.i, Graphics style
@subsection Coordinate systems

Most Yorick graphics styles define only a single coordinate system.
In other words, there is a single transformation between the world
coordinates of your data, and the normalized device coordinates (NDC)
on the sheet of paper or the screen.

The world-to-NDC transformation maps the x-y limits of your data to a
rectangle in NDC space, called the viewport.  Normally, Yorick will
not draw any part of your data lying outside the viewport.  However,
you can tell Yorick to use raw NDC coordinates without any
transformation by means of the plsys command:

@example
plsys, 0
@end example

@noindent
tells Yorick not to transform data from subsequent plotting
primitives.  When you are plotting to ``system 0'', your data will not
be clipped to the viewport.  In NDC coordinates, the bottom left
corner of the 8.5 by 11 sheet of paper is always (0,0), and 0.0013 NDC
unit is exactly one printer's point, or 1/72.27 inch.

You can switch back to the default ``system 1'' plotting with:

@example
plsys, 1
@end example

The fma command also switches you back to ``system 1'' as a side effect.
Switching windows with the window command puts you in the coordinate
system where you most recently plotted.  Note that plsys returns the old
system number if you call it as a function; together with the
current_window function you can write new functions which do something
elsewhere, then return to wherever you were before.

Remember that there is a special relation between the window on your
screen and an 8.5 by 11 inch sheet of paper; the bottom left corner of
the visible screen window is not usually at (0,0) in NDC.  Instead,
the NDC point (0.3993,0.6342) in portrait mode -- or (0.5167,0.3993)
in landscape mode -- remains at the center of your screen window no
matter how you resize it.  (The points make more sense in inches:
(4.25,6.75) -- centered horizontally and an equal distance down from
the top of the page for portrait orientation, and (5.50,4.25) --
centered on the page for landscape orientation.)  Furthermore, the
screen scale is always 75 dpi or 100 dpi, that is, either 798.29
pixels per NDC unit or 1064.38 pixels per NDC unit.

(I apologize to European users for not making explicit provision for
A4 paper.  If you use Yorick's eps command for serious, high quality
output, the page positioning is not an issue.  For rough draft or
working plots, I hope the fact that Yorick produces off-center
pictures on A4 paper will not hurt you too much.  An inch,
incidentally, is 2.54 cm.)

You can also create a graphics style with multiple world coordinate
systems.  For example, you could define a style with four viewports
in a single window, in a two by two array.  If your style has more
than one coordinate system, you use

@example
plsys, i
@end example

@noindent
to switch among them, where i can be 0, 1, 2, up to the number of
coordinate systems in your style.  Each coordinate system has its own
limits, logxy, and gridxy parameters.  However, all systems in a
window must share a single palette (many screens don't have more
colors than what you need for one picture anyway).  The limits, logxy,
and gridxy commands are meaningless in ``system 0'', since there is no
coordinate transformation there.  If invoked as a function, plsys will
return the current coordinate system, which allows you to do something
in another system, then reset to the original system.

For interactive use, I find multiple coordinate systems too confusing.
Instead, I pop up multiple windows when I need to see several pictures
simultaneously.  For non-interactive use, you should consider making
separate Yorick pictures, saving them with the eps command, then
importing the separate pieces into a page layout program to combine
them.

@node ticks and labels,  , plsys, Graphics style
@subsection Ticks and labels

The style parameters regulating the location of the viewport,
appearance of the tick marks, and so on are fairly straightforward:
You can specify which of the four sides of the viewport will have
tick marks, or you can select ``axis'' style, where coordinate axes
and ticks run through the middle of the viewport.  Tick marks can
extend into or out of the viewport, or both (beware that tick marks
which project into the viewport often overlay a portion of your data,
making it harder to interpret).  You can draw a frame line around the
edge of the viewport, or just let the tick marks frame your picture.
You can specify which, if any, of the four sides of the viewport will
have numeric labels for the ticks, the distance from the ticks to the
labels, the font and size of the labels, and so forth.

An elaborate artificial intelligence (or stupidity) algorithm
determines tick and label positions.  Four parameters control this
algorithm: nMajor, nMinor, logAdjMajor, and logAdjMinor.  Tick marks
have a hierarchy of lengths.  The longest I call major ticks; these
are the tick marks which get a numeric label (if you turn on labels).
Like the scale on a ruler, there is a hierarchy of progressively
shorter ticks.  Each level in the hierarchy divides the intervals
between the next higher level into two or five subdivisions -- two if
the interval above had a width whose least significant decimal digit
of 1 or 2, five if that digit was 5.

Only two questions remain: What interval should I use for the major
ticks, and how many levels should I proceed down the hierarchy?  The
nMajor and nMinor parameters answer these questions.  They specify the
upper limit of the density of the largest and smallest ticks in the
hierarchy, respectively, as follows: The tick interval is the smallest
interval such that the ratio of the full interval plotted to that
interval is less than nMajor (or nMinor).  Only major tick intervals
whose widths have least significant decimal digit 1, 2, or 5 are
considered (otherwise, the rest of the hierarchy algorithm fails).

Log scaling adds two twists: First, if there are any decades, I make
decade ticks longer than any others, even if the first level of
subdecade ticks are far enough apart to get numeric labels.  If there
will be ticks within each decade, I divide the decade into three
subintervals -- from 1 to 2, 2 to 5, and 5 to 10 -- then use the
linear scale algorithm for selecting tick intervals at the specified
density within each of the three subintervals.  Since the tick density
changes by at most a factor of 2.5 within each subinterval, the linear
algorithm works pretty well.  You expect closer ticks on a log scale
than you see on a linear scale, so I multiply the nMajor and nMinor
densities by logAdjMajor and logAdjMinor before using them for the
subintervals.

A final caveat: On the horizontal axis, long numeric labels can
overlap each other if the tick density gets too high.  This depends on
the font size and the maximum number of digits you allow for those
labels, both of which are style parameters.  Designing a graphics
style which avoids this evil is not easy.

@node Query and edit, pldefault, Graphics style, Graphics
@section Queries, edits, and legends

Yorick allows you to query the current display list, and to edit
parameters such as line types, widths, and the like.  I usually choose
to simply make a new picture with corrected values; the existing low
level editing command is more suited to be the basis for a higher
level menu driven editing system.

@menu
* legends::                     Setting plot legends.
* plq and pledit::              The plot query and edit functions.
@end menu

@node legends, plq and pledit, Query and edit, Query and edit
@subsection Legends

Every plotting primitive function accepts a legend= keyword to allow
you to set a legend string describing that object.  If you do not
supply a legend= keyword, Yorick supplies a default by repeating a
portion of the command line.  For example,

@example
plg, cos(x), x
@end example

will have the default legend @kbd{"A: plg, cos(x), x"}, assuming that
the curve marker for this curve is @kbd{"A"}.  You can specify a more
descriptive legend with the @kbd{legend=} keyword:

@example
plg, cos(x), x, legend="oscillating driving force"
@end example

If you want the legend to have the curve marker prepended, so it is
@kbd{"A: oscillating driving force"} if the curve marker is @kbd{"A"},
but @kbd{"F: oscillating driving force"} if the curve marker is
@kbd{"F"}, you can begin the legend string with the special character
@kbd{"\1"}:

@example
plg, cos(x), x, legend="\1: oscillating driving force"
@end example

Like legends, you can specify a curve marker letter with the marker=
keyword, but if you don't, Yorick picks a value based on how many curves
have been plotted.  By default, Yorick draws the marker letter on top of
the curve every once in a while -- so A's mark curve A, B's mark curve
B, and so on.  This is only relevant for the plg and plc commands.  This
default style is ugly; use it for working plots, not polished graphics.
You should turn the markers off by means of the @kbd{marks=0} keyword
for high quality plots, and distinguish your curves by line type.  For
example,

@example
plg, cos(x), x, marks=0, type="dash",
  legend="oscillating driving force (dash)"
@end example

In order to conserve screen space, legends never appear on your screen;
they only appear in hardcopy files.  Furthermore, depending on the
graphics style, legends may not appear in hardcopy either.  In
particular, the @file{vg.gs} and @file{nobox.gs} styles have no legends.
This is because legends are ugly.  Legends take the place of a proper
figure caption in working plots.  For high quality output, I expect you
to take the trouble to add a proper caption.  You can use the
@kbd{legends=0} keyword to the window command in order to eliminate the
legends even from those graphics styles where they normally appear.

@node plq and pledit,  , legends, Query and edit
@subsection plq and pledit

Any time you want to see your legends, you can type:

@example
plq
@end example

For any plotting primitive you have issued, you can find all its
keyword values by specifying its index in the plq list (the basic plq
command prints this index):

@example
plq, 3
@end example

For the plc command, each contour level is a polyline; you can find
out about each individual level by adding a second index argument to
plq.

You can also invoke plq as a function in order to return
representations of the current display list elements for use by a
Yorick program.

Just as plq allows you query the current display list, pledit allows
you to edit it.  You specify an index as in the plq command (or two
indices for a specific contour level), then one or more keywords that
you want to change.  This is occasionally useful for changing a line
type or width.  One could build a more user friendly interface
combining plq and pledit with some sort of menu or dialogue.

@node pldefault, Custom plot functions, Query and edit, Graphics
@section Defaults for keywords

With the pldefault command, you can set default values for many of the
keywords in the window command and the primitive plotting commands.
These keywords control the line type, width, and color for the plg,
plc, plm, and pldj commands, the style and dpi of newly created
windows, and other properties of graphics windows and objects.

You can use pldefault interactively to set new, but temporary, default
values before embarking on a series of plots.  Or place a pldefault
command in your @file{~/yorick/custom.i} file if your personal
preferences differ from Yorick's defaults.

You must use the hcp_file command to set values for the @kbd{dump=} or
@kbd{ps=} keywords (which determine whether to dump the palette to
hardcopy files and whether automatically created hardcopy files are
PostScript or binary CGM), and these settings remain in effect until you
explicitly change them.

For example, if you want to use the @file{vg.gs} style, with unmarked
wide lines, and you want PostScript hardcopy files (by default)
including the palette for each frame, you can put the following two
lines in your @file{custom.i} file:

@example
pldefault, style="vg.gs", marks=0, width=4;
hcp_file, dump=1, ps=1;
@end example

The values you set using pldefault are true default values -- you can
override them on each separate call to window or a plotting primitive,
but the default value remains in effect unless you reset it with a
second call to pldefault.  Conversely, you can't temporarily override a
@kbd{dump=} or @kbd{ps=} value you set with hcp_file -- you just have to
set them to a new value.

@node Custom plot functions, Animation, pldefault, Graphics
@section Writing new plotting functions

You may want to plot something not directly supported by one of the
plotting primitive functions.  Usually, you can write your own
plotting function to perform such a task.  As an example, suppose you
want to plot a histogram -- that is, instead of a smooth curve
connecting a series of (x,y) points, you want to draw a line
consisting of a horizontal segment at each value of y, joined by
vertical segments (this is sometimes called a Manhattan plot for its
resemblance to the New York City skyline).

You quickly realize that to draw a histogram, you really need the x
values at the vertical segments.  This function works:

@example
func plh(y,x)
@{
  yy = xx = array(0.0, 2*numberof(y));
  yy(1:-1:2) = yy(2:0:2) = y;
  xx(2:-2:2) = xx(3:-1:2) = x(2:-1);
  xx(1) = x(1);
  xx(0) = x(0);
  plg, yy, xx;
@}
@end example

Notice that the x array must have one more element that the y array;
otherwise the assignment operations to xx will fail for lack of
conformability.

A more sophistocated version would include the possibility for the
caller to pass plh the keywords accepted by the plg function.  Also,
you might want to allow y to have one more element than x instead of x
one more than y, in order to start and end the plot with a vertical
segment instead of a horizontal segment.  Here is a more complete
version of plh:

@example
func plh(y,x,marks=,color=,type=,width=)
/* DOCUMENT plh, y, x
     plot a histogram (Manhattan plot) of Y versus X.  That is,
     the result of a plh is a set of horizontal segments at the Y
     values connected by vertical segments at the X values.  If X
     has one more element than Y, the plot will begin and end with
     a horizontal segment; if Y has one more element than X, the
     plot will begin and end with a vertical segment.  The keywords
     are a subset of those for plg.
   KEYWORDS: marks, color, type, width
   SEE ALSO: plg
 */
@{
  swap = numberof(x)<numberof(y);
  if (swap) @{ yy = y; y = x; x = yy; @}
  yy = xx = array(0.0, 2*min(numberof(y),numberof(x)));
  yy(1:-1:2) = yy(2:0:2) = y;
  xx(2:-2:2) = xx(3:-1:2) = x(2:-1);
  xx(1) = x(1);
  xx(0) = x(0);
  if (swap) @{ y = yy; yy = xx; xx = y @}
  plg, yy, xx, marks=marks, color=color, type=type, width=width;
@}
@end example

The point of an interpreted language is to allow you to easily alter
the user interface to suit your own needs.  Designing an interface for
a wide variety of users is much harder than designing one for your own
use.  (The first version of plh might be adequate for your own use; I
wouldn't release less than the second version to a larger public.)
Linking your routines to the Yorick help command via a document
comment is useful even if never anticipate anyone other than yourself
will use them.  Public interface routines should always have a
document comment.

@node Animation, 3D graphics, Custom plot functions, Graphics
@section Animation

Yorick's ordinary drawing commands make poor movies.  In order to make
a good movie on your screen, you need to use more resources -- the
idea is to draw the picture into offscreen memory first, then use a
fast copy operation to make the entire frame appear instantly.

The animate command creates an offscreen pixmap, and redirects the
rendering routines there, and modifies the fma command to perform the
copy from offscreen to onscreen.  A second animate command returns to
the ordinary mode of operation.

You can't use animate mode for ordinary interactive work, because you
don't get to see your picture until the fma.  Typically, you need to
write a Yorick function which enters animate mode, loops producing
frames of the movie, then exits animate mode.  Use the high level
movie interface by including

@example
#include "movie.i"
@end example

After this include, you can use the movie function, and let it manage
the low level animate function for you.  You only need to write a
function (passed as an argument to movie) that draws the n-th frame of
your movie and returns a flag saying whether there are any more frames.
You can easily build and test this function a frame at a time in
ordinary non-animation mode.  Study the @file{demo2.i}, @file{demo3.i},
and @file{demo5.i} demonstration programs that come with the Yorick
distribution to learn more about making movies.

@node 3D graphics,  , Animation, Graphics
@section 3D graphics interfaces

You can combine Yorick's graphical primitives to create your own
graphics interfaces.  The Yorick distribution includes a simple 3D
graphics interface built in just this way.  Study the @file{pl3d.i} library
file if you need an example of how to do this.

Before you can access the 3D functions, you need to include either
@file{plwf.i} or @file{slice3.i}, which provide the highest level 3D
interfaces.  The @file{plwf.i} interface lets you plot scalar functions
of two variables (like plc or plf).  The @file{slice3.i} interface
allows you to plot isosurfaces and slices of scalar functions of three
variables.  The @file{demo5.i} demonstration program in the Yorick
distribution has examples of both; study it to see how they work in
detail.

@menu
* 3D mapping::                  Changing your viewpoint.
* 3D lighting::                 The 3D lighting model.
* 3D gnomon::                   Gnomon indicates axis orientation.
* plwf::                        The plot wire frame interface.
* slice3::                      The slice and isosurface interface.
@end menu

@node 3D mapping, 3D lighting, 3D graphics, 3D graphics
@subsection Coordinate mapping

Since the picture you are viewing is always two dimensional, you need
to project three dimensional objects onto your screen.  I designed the
3D interface to allow you to change this projective mapping without
re-issuing plotting commands, by analogy with the 2D limits command.

Instead of limits, however, you need to specify a projection.  Imagine
that your screen is the image plane of a camera.  This camera carries
a coordinate system in which +x is rightward, +y is upward, and +z is
out of the screen toward your face.  Your world coordinate system --
the coordinates of your data -- has some orientation relative to these
camera coordinates.  You can rotate your world coordinates -- and see
the object you have plotted rotate -- using either the rot3 or orient3
commands.

The three arguments to rot3 are the angles (in radians) to rotate your
object about the camera x, y, and z axes.  These three rotations are
applied in order, first the x angle, then the y angle, then the z
angle.  Omitted or nil arguments are the same as zero arguments.  Note
that 3D rotation is not a commutative operation, so if you want the z
rotation to come first, followed by an x rotation, you need to issue
two rot3 commands (like limits or logxy changes, rot3 changes
accumulate so Yorick will only redraw your screen once):

@example
rot3,,,zangle; rot3,xangle
@end example

The orient3 operation is less general than rot3, and therefore easier
to use.  With orient3, the z axis of your world coordinates always
projects parallel to the camera y axis, so z is plotted upward on your
screen.  The first argument to orient3 is the angle about the world z
axis that the world x axis should be rotated relative to the camera x
axis.  The optional second argument to orient3 is the angle from the
camera y to the world z -- the angle at which you are looking down at
the scene.  Once set, that downlook angle persists until you reset it.
With no arguments, orient3 returns to a sane standard orientation.
Hence, you can just type:

@example
orient3
@end example

@noindent
to return to a standard view of your scene.

Unlike rot3, the results of an orient3 are not cumulative.  With each
orient3 call, you completely specify the orientation of your object.
Thus, if you set the orientation of your object by orient3:

@example
orient3, -.75*pi
@end example

@noindent
and you want to twist it just a little, you would issue a slightly
different orient3 command:

@example
orient3, -.70*pi
@end example

While if your rot3 command

@example
rot3,, -.75*pi
@end example

@noindent
were not quite right, you'd twist it a little more by specifying a
small angle:

@example
rot3,, .05*pi
@end example

Unfortunately, the projection operation is more complicated than
merely the three angles of an arbitrary rotation matrix.  The distance
from your camera (or eye) to the object also enters.  By default, you
are infinitely far away, looking through an infinite magnification
lens.  This is called isometric projection, because equal length
parallel segments in your world coordinates have equal length on your
screen, no matter how close or far from the camera.  If you want a
perspective picture, you can use the setz3 command to set the z
value in the camera coordinate system where your camera (or eye) is
located.

Don't bother attempting to put the camera inside your data; it's not
edifying.  Extreme fisheye perspectives won't make your scientific
data more intelligible.  Keep the camera well outside your object.

Your camera always looks toward -z.  Initially, the aim point is the
origin of the camera coordinate system, but you can change it to any
other point by means of the aim3 function.  If you want to specify an
aim point in your world coordinate system instead, use the mov3
function, which is analogous to rot3.

In general, you should use the nobox.gs graphics style, since the 2D
axes are not useful.  For an isometric view, the 2D (x,y) axes are
simple projections of your (x,y,z) world coordinates.  For a
perspective view (after you have called setz3), the 2D (x,y)
coordinates are the tangents of ray angles between the camera z axis
and points on your object.  The 2D limits command is useful for
cropping or adjusting your view, but you should not turn off the
@kbd{square=1} keyword.

The functions save3 and restore3 save and later restore the entire 3D
coordinate mapping, including the effects of rot3 or orient3, mov3,
aim3, and setz3.  You can also undo the effects of any number of these
commands by means of the undo3 function.

@node 3D lighting, 3D gnomon, 3D mapping, 3D graphics
@subsection Lighting

Yorick's pseudocolor model cannot describe a shaded color object.  (You
need to be able to specify arbitrary colors for that.)  Therefore, if
you want shading to suggest a 3D shape, you must settle for black and
white (or any single color -- the @file{heat.gp} palette works about as
well as @file{gray.gp} for shading).

You also need to worry about more variables -- namely the position and
relative brightness of any lights illuminating your object.  You call
the light3 function to set up light sources.

The 3D surfaces comprising your object (plotted by either plf or plfp
for the @file{plwf.i} or @file{slice3.i} interfaces) consist of a number
of polygonal facets.  Yorick assigns a 3D normal vector to each facet
(for non-planar polygons, the direction is somewhat arbitrary).  The
lighting model, which you set up by light3 calls, maps normal directions
to surface brightness.  If the facet is oriented so that it reflects one
of your light sources directly toward your camera, that facet will
appear very bright.

The functional form of this specular reflection model is
@kbd{(1+cos(alpha))^n}, where alpha is the angle between your camera and
the light source, as seen from the center of the facet.  You can adjust
the power n.  Large n gives a polished metallic look; small n gives a
matte look.  You can specify as many light sources as you want.
However, the light sources are all infinitely far away.  Also, although
you can specify a different n for each light source, you cannot get
dfferent values of n for different surfaces or parts of surfaces.

Instead of or in addition to these specular light sources, you can
also get diffuse lighting.  In this model, the brightness of a surface
is simply proportional to the cosine of its angle relative to your
viewing direction.  Thus, a surface you view face on is brightest,
while one viewed edge on is dimmest (the effect is called limb
darkening).

All of these effects are controlled by keywords to the light3
function: specular (brightness of specular light sources), sdir
(directions to specular light sources), spower (the powers n for the
specular light sources), diffuse (brightness of diffuse light source),
and ambient (an overall additive brightness).  The light3 function
also returns a value, which you can pass back to it in a later call
in order to restore a previous lighting specification.

These lighting models -- which are actually the standard models you
find in most 3D graphics packages -- bear little relationship to the
appearance of real world scenes.  There are no shadows, and surfaces
have no intrinsic texture.  Your goal is excitement, not realism.

@node 3D gnomon, plwf, 3D lighting, 3D graphics
@subsection gnomon

Use the gnomon function to turn the gnomon on or off.  The gnomon is a
projection of three orthgonal segments parallel to the world x, y, and
z axes.  The letters x, y, and z appear near the ends of the segments
-- a black on white letter means the segment is pointing out of the
screen; white on black means into the screen.

A related function is cage3, which turns 3D axes on or off.  The cage is
a rectangular box surrounding your object, with tick marks on its edges
to make your plot a 3D analogue of a typical 2D plot.  (Of course,
unlike the 2D plot, there is no way you can actually use the ticks
around the cage to figure out the coordinates of a point in your
picture.)  The faces of the cage are at planes set by the limit3
function.  The plwf function automatically calls limit3 with reasonable
extreme values, but for the pl3tree function you will need to call
limit3 yourself.  Unlike the 2D limits function, limit3 allows you to
set the aspect ratio of the cage; by default cage3 scales your x, y, and
z world coordinates to make the edges of the cage appear as equal
lengths.

@node plwf, slice3, 3D gnomon, 3D graphics
@subsection plwf interface

A plwf plot contains the same information as a plf or plc plot; a plwf
is like looking at terrain from a mountain peak, while plf or plc is
like looking at a map.  (Have you ever stood at a vista point, and
used a map to identify and understand the ``real'' scene you are
seeing?  Would you rather have a photograph of the vista or a map to
guide you through unfamiliar terrain?)

Like plc, plwf needs point centered z values on a quadrilateral mesh.
A scale factor relating the units of z to the units of x or y may be
specified via keyword; if not, plwf chooses a scale factor to make the
range of z values half of the larger of the ranges of x and y.  Other
keywords allow you to choose whether or not the mesh lines will be
drawn, and whether the surface itself will be shaded according to the
scene lighting, or simply left the background color.  (The latter
choice results in a wire frame plot, which is the wf in plwf.)

@node slice3,  , plwf, 3D graphics
@subsection slice3 interface

The slice3 function, in contrast, makes planar slices or isosurface
contours of functions of three variables.  You can pseudocolor the
planar slices, or show projections of the isosurfaces, or combine
slices with isosurfaces.  The plfp function makes the actual plots,
but you call a higher level function, pl3tree.

Unlike plwf, you must invoke at least two functions in order to plot
anything.  First, you call to slice3 returns either a planar slice
through your 3D data, or an isosurface of some scalar function on your
3D mesh.  Next, you call pl3tree in order to add the slice or
isosurface to your 3D display list.

Again, Yorick's pseudocolor model does not permit partially transparent
surfaces.  Often, you need to slice open closed isosurfaces like a melon
in order to view the interior.  Use the slice2 and slice2x functions to
make slices like this.  Unfortunately, when you remove slices to make
interior surfaces visible, you will only be able to see through your
slice from a restricted range of viewing directions.  That is, the best
slicing planes will depend on your viewing direction.  Study the
@file{demo5.i} for an example of how to make this type of picture.

A trick allows you to both color your slices (like a 2D plf or pli), and
shade your isosurfaces, despite the limitations of the pseudocolor
model.  The idea is to split your palette into a gray scale you can use
to shade isosurfaces, followed by a color palette you can use to
pseudocolor slices.  The split_palette routine in @file{slice3.i}
generates such a split palette.  Again, read @file{demo5.i} for
details.




@comment -----------------------------------------------------------------

@comment @node Embedding, Concept Index, Graphics, Top
@node Embedding, , Graphics, Top
@chapter Embedding Compiled Routines Inside Yorick

You can create a custom version of Yorick containing your own C or
Fortran compiled code.  If you are careful, your custom version will be
easily portable to any site where Yorick has been installed.  You will
considerably ease portability problems (read ``future hassles for
yourself'') by writing in ANSI C, which is a portable language, as
opposed to Fortran, which is not.

My experience with C++ is that its portability is intermediate between
ANSI C and Fortran.  You should be able to write C++ packages for Yorick
by using the @kbd{extern "C"} statement for the interface routines
called by the interpreter.  I don't encourage this, however, since the
interpreted language removes many of the motives for programming in C++
in the first place.

If you do choose Fortran, stick to a strict subset of ANSI Fortran 77.
Do not attempt to pass character variables into or out of interface
routines, nor put them in common blocks.  Also, try not to use common
blocks to pass inputs to or receive outputs from your interface
routines.  (This is possible; if you enjoy Fortran programming,
presumably you'll enjoy figuring out a portable way to do this.)

Whether you write in Fortran or C, @emph{do not} attempt to do I/O of
any sort in your compiled code.  The whole idea of embedding routines
inside Yorick is to let Yorick handle all I/O -- text and graphics.
Another way to say this is, that you want the interface between the
interpreted and compiled code to be arrays of numbers.  Let the
interpreter handle reading text to produce the input arrays, or
writing graphics to display the output arrays.

@menu
* Mechanics of embedding::      How to use the make utility.
* API for embedding::           Using the yapi.h interface.
* Embedding example::           Three ways to implement a function.
@end menu

@node Mechanics of embedding, API for embedding, Embedding, Embedding
@section Mechanics of embedding

In the following discussion, I refer to three directories:
@file{Y_SITE} is the directory where the architecture independent
parts of Yorick reside; everything Yorick needs at runtime is here.
@file{Y_HOME} is the directory where the libraries and executables you
need to build custom versions are stored.  @file{Y_LAUNCH} is the
directory containing the executable for your version of Yorick.  When
you run Yorick, the names of all three directories are available as
variables; for example, start yorick and type @kbd{Y_HOME} to print
the name of that directory.  Also, I will only discuss the UNIX
program development environment; the same system works tolerably well
under the MinGW or Cygwin environments under Windows, which you can
use with the commerical MSVC++ compiler if you wish.

The paradigm for a compiled extension to yorick is called a ``compiled
package''.  The idea is that a compiled package should be
indistinguishable from an interpreted package (for example, a @kbd{.i}
file in the interpreted library that comes with the distribution).  In
the case of a purely interpreted package, you write an include file
defining your interface.  So, too, you must lay out the interpreted
interface for a compiled package in an interpreted include file, which
we call here @file{mypkg.i}.  Including @file{mypkg.i} makes the
compiled package available to the interpreter.

The compiled package include file @file{mypkg.i} must contain one
statement, executed before any code that uses the compiled routines,
which distinguishes it from an interpreted package:

@example
plug_in, "mypkg";
@end example

Here, we assume the package name is ``mypkg'', matching the name of
the file @file{mypkg.i}, but you can choose any name for the package
you like.  In fact, you can have any number of include files defining
different pieces of the interface to a single compiled package, say
@file{mypkg1.i}, @file{mypkg2.i}, and so on, each of which would
contain the above @kbd{plug_in} call.  In a single compiled package
source directory, however, there can be at most one package name.  (If
you want to build more than one binary library, use more than one
directory to hold the source.)

An alternative form is permitted to make it possible to design package
include files which are compatible with pre-version-1.6 yorick:

@example
if (!is_void(plug_in)) plug_in, "mypkg";
@end example

@noindent
Unlike an interpreted package, before you can use a compiled package,
you naturally need to compile it!  This is done using the UNIX make
utility.  Yorick has tools to assist in this build process.  The first
is the (interpreted) package @kbd{make.i} which generates, or at least
provides a starting point for, the @file{Makefile}, which is the input
to the make utility.  The second is @kbd{codger}, a (compiled) yorick
utility which generates the C source code that bridges between your
compiled code and yorick's internal compiled code.

First, you need to put your C (and/or Fortran) source and header files
in the directory containing the @file{mypkg.i} interface definition
file(s).  Note that you need to design your C interface in an
``interpreter friendly'' fashion.  For example, it is worthless to
write a special function that computes a special function at a single
point, because the interpreter would need to call it in a loop to
compute a vector of values, most likely erasing the entire advantage
of writing your special function in compiled code.  At the very least,
therefore, you want your compiled functions to accept a (potentially
long) list of inputs an produce the corresponding list of outputs.

When all the @kbd{.i}, @kbd{.c}, and @kbd{.h} (and Fortran) source
files for your package have been collected in this package source
directory (and any unrelated source removed), execute the command
(in that directory):

@example
yorick -batch make.i
@end example

This creates a @file{Makefile} for your package.  In simple cases,
this @file{Makefile} will be all you need to build your compiled
package; in more complicated situations, you will need to edit it by
hand to fill in critical information the @file{make.i} package is
unable to determine by its examination of the source files.

The main reason you may need to edit @file{Makefile} by hand is to add
the loader flags (-l and -L) for dependencies, that is, third party
libraries, your compiled code calls.  You may also need to add
compiler flags (-I) to enable the compiler to find the header files
which declare the interfaces to such libraries.  The corresponding
macros in @file{Makefile} are @kbd{PKG_DEPLIBS} and @kbd{PKG_CFLAGS},
respectively.  In general, if your package needs modifications like
this and you wish it to be buildable by non-experts, you will need to
modify @file{Makefile} to pick up the results of a configure script,
which you will need to design and write.  This can be far more
difficult than writing the compiled package in the first place.  The
message is, do not rely on dependent libraries other than libc, libm,
and the utilities in yorick's own @file{play/} portability layer if
you can possibly avoid it.

Once you have the @file{Makefile}, use the make utility to actually
build your compiled package.  The @file{Makefile} invokes a more
extensive input file, @file{Y_HOME/Makepkg}, which provides a number
of useful ``targets'' you can build.  Here are several useful make
command lines; read @file{Y_HOME/Makepkg} to learn about others:

@example
make           # build release version
make install   # install it (may need to be root)
make debug     # build debuggable version
make clean     # delete all but source and Makefile
@end example

The release version of your package will be a dynamically loadable
yorick plugin, as long as yorick is able to handle plugins on your
platform.  The debuggable version, and the release version on
platforms which do not permit plugins, will be a complete yorick
executable, which statically loads your new package at startup.  (By
default, it will simply be called ``yorick'', and installing such a
release version it will overwrite the yorick you used to run
@kbd{-batch make.i}, so be careful.)

When you want to build your package on a different platform, you always
need to run:
@example
yorick -batch make.i
@end example
@noindent
before you rerun the make utility.

In addition to the @kbd{plug_in} command, your @file{mypkg.i} package
interface file must contain statements that allow the @kbd{codger}
utility to construct the bridge code.  During the build process,
@kbd{codger} examines @file{mypkg.i} package, looking for interpreted
@kbd{extern} statements outside any function body.  For each such
@kbd{extern} statement, codger generates an interpreted built-in
function, or an interpreted pointer to compiled global data.  When the
interpreter includes the file, these statements are no-ops, except
that they may include DOCUMENT comments that are added to the
interactive @kbd{help} system.  The @file{mypkg.i} file will usually
also contain purely interpreted wrapper code for these compiled
objects, in order to create a high-quality interpreted API.

For example, when @kbd{codger} finds this:

@example
extern my_func;
/* DOCUMENT my_func(input)
     returns the frobnostication of the array INPUT.
 */
@end example

@noindent
in the @file{mypkg.i} file, it connects the interpreted symbol
@kbd{my_func} to a compiled function @kbd{Y_my_func} with a specific
prototype, which you are responsible for writing, as described in the
following section.

Often you can avoid all these details; codger can generate
@kbd{Y_my_func} for you automatically.  To do this, put PROTOTYPE
comments in your startup include file:

@example
func my_func(input)
/* DOCUMENT my_func(input)
     returns the frobnostication of the array INPUT.
 */
@{
  return my_func_raw(input, numberof(input));
@}
extern my_func_raw;
/* PROTOTYPE
   double my_func_C_name(double array input, long length)
 */
@end example

This generates a wrapper for a C function which takes a single array as
input and returns a scalar result.  If the function had been Fortran,
it would have looked like this (Fortran passes all arguments by reference
-- that is, as if they were arrays):

@example
func my_func(input)
/* DOCUMENT my_func(input)
     returns the frobnostication of the array INPUT.
 */
@{
  return my_func_raw(input, numberof(input));
@}
extern my_func_raw;
/* PROTOTYPE FORTRAN
   double my_func_Fortran_name(double array input, long array length)
 */
@end example

Legal data types for the function return result in the @kbd{PROTOTYPE}
comment are: void (i.e.- a subroutine), char, short, int, long, float,
or double.

Legal data types for the function parameters in the @kbd{PROTOTYPE}
comment are: void (only if there are no other parameters), char, short,
int, long, float, double, string (char *, guaranteed 0-terminated), or
pointer (void *).  These may be followed by the word "array", which
becomes "*" in the C source code, to indicate an array of that type.
The parameter name is optional.

The @kbd{DOCUMENT} comment should start with @kbd{/* DOCUMENT}.  They
will be returned by the interpreted command @kbd{help, my_func}, and be
included in the poor- man's document produced by Yorick's "mkdoc"
command (see @file{Y_HOME/i/mkdoc.i}).

@example
extern my_global;
reshape, my_global, datatype;
@end example

attaches the interpreted variable @kbd{my_global} to a C-compiled global of
the same name, which has the data type @kbd{datatype} (this must have been
declared in a previous struct or be one of the primitive types).  If you
want @kbd{my_global} to be attached to a global variable of a different
name, use:

@example
extern my_global;
/* EXTERNAL my_global_C_name */
reshape, my_global, datatype;
@end example

To attach to a Fortran common block, say

@example
        double var1, var2, var3
        common /my_common/ var1, var2, var3
        save /my_common/
@end example

(note that this doesn't make sense unless the common block is saved outside
the scope of the functions in which it is used) use:

@example
struct my_common_type @{ double var1, var2, var3; @}
extern my_common;
/* EXTERNAL FORTRAN my_common */
reshape, my_common, my_common_type;
@end example

If you mix double, integer, and real data in a single common block, you
can ensure that you won't have any alignment difficulties by putting
all the doubles first, followed by integers and reals.  If you don't do
this, you're relying on the existence of a Fortran compiler switch which
forces proper data alignment -- some machine someday won't have this.

This covers all the things that @kbd{codger} can do for you.

The primary goal of this system is portability.  The basic idea is
that all the platform specific problems can be solved once in the
@file{Y_HOME/Makepkg} file, so that you can easily move your compiled
packages from one platform to another.  With the advent of plugins,
this system also must cope with packages which are either statically
or dynamically loaded, and the same source code must work for either
case.  The secondary goal is that the final users should do nothing
different to use a compiled package than they would to use an
interpreted package.

@node API for embedding, Embedding example, Mechanics of embedding, Embedding
@section API for embedding

Instead of letting @kbd{codger} generate simple wrapper functions
for you, you can omit the @kbd{PROTOTYPE} comment, and write
@kbd{Y_my_func} yourself.  Its prototype is

@example
extern void Y_my_func(int argc);
@end example

The @kbd{argc} argument is the count of the number of items on the stack
containing the arguments the interpreter is passing to @kbd{Y_my_func}.
This would be simply the number of arguments the caller is passing to
your function, were it not for the possibility of keyword arguments.
Each actual keyword argument takes two spaces on the stack -- one for
the name of the keyword and a second for its value.  If your function
does not accept keywords, you can ignore the distinction between
@kbd{argc} and the number of arguments being passed to your function.

When your function exits, whatever you leave on top of the stack becomes
its return value.  This section explains the programming interface (or
API) which permits you to get your arguments off the stack, push new
items onto the stack, set or get global variable values, and otherwise
interact with the yorick interpreter.  All of these functions are
declared in the @file{yapi.h} header, which your C (or C++) source file
should include.

The following examples do not illustrate all of the functions in
@file{yapi.h}; you need to read the header file itslef for a
comprehensive list.  For more examples of @file{yapi.h} usage, read
the yorick source file @file{yorick/ystr.c}.

@node Embedding example,  , API for embedding, Embedding
@section Embedding example

Suppose you want to write a function which returns
@kbd{sqrt(r^2-x^2)}, with some of the nice features of the @kbd{abs}
function, to be called @kbd{rtleg} for ``right triangle leg''.  Your
first option is to simply write an interpreted function:

@example
func rtleg(r,x,..)
@{
  rr = 1./r;
  x *= rr;  /* prevent overflow for very large or small r */
  y = 1. - x*x;
  while (!is_void((x=next_arg()))) @{ x *= rr; y -= x*x; @}
  return r*sqrt(y);
@}
@end example

This interpreted function is likely the best solution, because the
required function is simple and vectorizes naturally.  For the sake of
argument, however, suppose that you wanted to create a compiled
version of @kbd{rtleg} instead.  Now you need not only interpreted
code, in a file @file{rtleg.i}, but also compiled code in a file
@file{rtleg.c}.  Let's consider two cases: First, let's use the
@kbd{PROTOTYPE} comment and let codger generate the required wrapper
code.  Second, let's write the wrapper code using @file{yapi.h}
directly.

@menu
* Using PROTOTYPE::             Example of PROTOTYPE comment.
* Using yapi.h::                Example of @file{yapi.h} API.
@end menu

@node Using PROTOTYPE, Using yapi.h, , Embedding example
@subsection Using @kbd{PROTOTYPE}

Suppose you really didn't need the feature that @kbd{rtleg} accepts
more than two input arguments.  In that case, you can very quickly
write a C function to compute the required two argument result:

@example
#include <math.h>
extern void crtleg(double *r, double *x, long n);
void
crtleg(double *r, double *x, long n)
@{
  double y;
  while (n-- > 0) @{
    y = *(x++) / r[0];
    *(r++) *= sqrt(1. - y*y);  /* result overwrites r input */
  @}
@}
@end example

You can now use a @kbd{PROTOTYPE} comment to have @kbd{codger}
generate the C wrapper code to make your @kbd{crtleg} function
callable from interpreted code.  However, you also need to write an
interpreted wrapper that will fill in the correct length @kbd{n}, and
ensure that the @kbd{r} and @kbd{x} arguments really do have that
length, and are conformable arrays.  You do all of that in your
interface definition file, @file{rtleg.i}, which might look like this:

@example
plug_in, "rtleg";
func rtleg(r, x)
@{
  /* convert r to double, broadcast it to result shape */
  r = double(r) + array(0.0, dimsof(r, x));
  /* broadcast x to result shape, blow up if not conformable */
  x += 0.*r;
  _crtleg, r, x, numberof(r);
  return r;
@}
extern _crtleg;
/* PROTOTYPE
   void crtleg(double array r, double array x, long n)
 */
@end example

When you use this approach, your C source code makes no references at
all to yorick.  It is both an advantage and a disadvantage that you
need to write an interpreted wrapper to handle all of the bookkeeping
tasks to ensure that arguments have correct types, shapes, are
conformable, and the like.  The disadvantage is that you now have to
maintain both a compiled and an interpreted piece of code.  The
advantage is that, as you will see in the next section, the compiled
code for consistency and conformability checks is even more tedious
than the interpreted code.

@node Using yapi.h,  , Using PROTOTYPE, Embedding example
@subsection Using @file{yapi.h}

When you use the @kbd{PROTOTYPE} comment to generate C code wrappers,
you cannot take advantage of many of the features of the yorick
language.  In this example, there is no easy way to pass an
indeterminate number of arguments to your function.  When you write
your own C code wrapper, all such limitations disappear.  The price is
that you need to use the yorick-specific C functions declared in
@file{yapi.h} to interact with the intepreter.

Since you no longer need an interpreted wrapper (although you might
still choose to have one -- some tasks are easier, occasionally
dramatically easier, in interpreted code than in compiled code), your
inteface definition file @file{rtleg.i} becomes trivial:

@example
plug_in, "rtleg";
extern rtleg;
@end example

But your C source becomes much more complicated:

@example
#include "yapi.h"
#include <math.h>
extern void Y_rtleg(int argc);
void
Y_rtleg(int argc)
@{
  int iarg;
  long i, n, dimsr[Y_DIMSIZE], dims[Y_DIMSIZE];
  double *r, *rr, **px, *x, *a, xval;
  int *flag, flagr;
  if (argc < 2) y_error("rtleg needs at least two arguments");

  /* get some scratch space to deal with the indeterminate
   * argument count, this increases iarg for arguments by 1
   */
  px = ypush_scratch((argc-1)*(sizeof(double)+sizeof(int)), 0);
  flag = (int *)&px[argc];

  /* get r argument as double */
  r = ygeta_d(argc, 0, dimsr);

  /* first pass through remaining arguments checks they are
   * conformable, and builds the result dimension list
   */
  for (flagr=0, iarg=argc-1 ; iarg>0 ; iarg--) @{
    flagr |= flag[iarg-1] = yarg_dims(iarg, dims, dimsr);
    if (flag[iarg-1] < 0)
      y_errorn("rtleg argument %ld not conformable", argc-iarg);
  @}

  /* broadcast r argument to result shape if necessary */
  if (flagr & 1) yarg_bcast(argc, dimsr);

  /* second pass broadcasts all arguments to result shape
   * and converts to type double
   */
  for (iarg=argc-1 ; iarg>0 ; iarg--) @{
    if (flag[iarg-1] & 2) yarg_bcast(iarg, dimsr);
    px[iarg-1] = ygeta_d(iarg, &n, 0);
    flag[iarg-1] = yarg_scratch(iarg);
    if (flag[iarg-1]) flagr = iarg;
  @}

  /* push new array and fill with reciprocal of r */
  rr = ypush_d(dimsr);
  for (i=0 ; i<n ; i++)
    if (r[i] > 0.) rr[i] = 1./r[i];
    else if (r[i] < 0.) rr[i] = -1./r[i];
    else rr[i] = -1.;

  if (flagr) @{  /* use scratch x as accumulator */
    a = px[flagr-1];
    for (i=0 ; i<n ; i++) @{
      xval = a[i]*rr[i];
      a[i] = -xval*xval;
    @}
  @} else @{      /* need to create accumulator on stack */
    a = ypush_d(dimsr);
  @}

  /* compute 1 - sum((x/r)^2) */
  for (i=0 ; i<n ; i++) a[i] += (rr[i]<0.)? 0. : 1.;
  for (iarg=0 ; iarg<argc-1 ; iarg++) @{
    x = px[iarg];
    if (x == a) continue;
    for (i=0 ; i<n ; i++) @{
      xval = x[i]*rr[i];
      a[i] -= xval*xval;
    @}
  @}

  /* compute final result */
  for (i=0 ; i<n ; i++) a[i] = r[i]*sqrt(a[i]);

  /* discard overlying stack elements to leave result on top */
  if (flagr) yarg_drop(flagr+1);
@}
@end example

The @kbd{ypush_scratch} function enables you to create scratch space
that is safe against asynchronous interrupt.  You have to write every
compiled function assuming that the user will type Control-C and
interrupt your function.  If you used @kbd{malloc} to obtain scratch
space, and your function were asynchronously interrupted before the
matching @kbd{free}, your routine would leak memory.  In this case,
the scratch space is only required to cope with the fact that the
function accepts an indeterminate number of arguments.

The implementation unavoidably requires a temporary array to hold the
reciprocal of the first argument, which is used to normalize the
remaining arguments to prevent overflows or erronious underflows when
the other arguments are squared.  However, there is an opportunity to
reuse an argument array which happens to be a temporary (either
because it is the result of an expression, or because it had to be
type converted or broadcast) as the result.  Taking advantage of that
optimization also costs a few additional lines of source code towards
the end of the function.

What you get for this modest complexity is an @kbd{rtleg} function
that is equal in performance and features to the built in @kbd{abs}
function.

@comment -----------------------------------------------------------------

@comment @node    Concept Index,  , Embedding, Top
@comment @unnumbered Concept Index

@comment @printindex cp

@contents
@bye
